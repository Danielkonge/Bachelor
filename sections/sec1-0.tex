Let $L$ be a semisimple Lie algebra and let $L_k$ be a Lie subalgebra.
\begin{definition}\label{def:HarishChandra}
  An $L$-module $M$ is a Harish-Chandra module for the pair $(L,L_k)$ if, regarded as an $L_k$-module, it can be written as a direct sum
  \begin{align*}
    M &= \bigoplus_{i} M_i
  \end{align*}
  of finite dimensional simple\footnote{In \cite{indecompReprOfLorGr} the word irreducible is used instead of simple, but we will only use irreducible when talking about representations in this paper.} $L_k$-submodules $M_i$, where for each $M_{i_0}$ only finitely many $L_k$-submodules equivalent to $M_{i_0}$ occur in the decomposition of $M$. If $L$ and $L_k$ are clear from the context we will just call $M$ a Harish-Chandra module.

  A Harish-Chandra module $M$ is indecomposable if it cannot be decomposed into the direct sum of non-zero $L$-submodules.
\end{definition}

Our goal is to classify all indecomposable Harish-Chandra modules over $(L,L_k)$ for $L=\Lie{sl}(2,\C)\times \Lie{sl}(2,\C)$ and $L_k=\Set{(u,u)\given u\in\slc}$, where we by $\Lie{sl}(2,\C)\times \Lie{sl}(2,\C)$ mean the following:

For $L,L'$ Lie algebras over $F$, we consider $L\times L'=L\oplus L'$ as a Lie algebra over $F$ with pointwise addition, multiplication given by $\alpha(a,b)=(\alpha a,\alpha b)$ for $\alpha\in F,a\in L,b\in L'$, and with Lie bracket $[(a_1,b_1),(a_2,b_2)]=([a_1,a_2],[b_1,b_2])$ for $a_1,a_2\in L,b_1,b_2\in L'$.

\begin{remark}
  Note that $L\times 0$ and $0\times L'$ are ideals in $L\times L'$ as given above. Thus we see that $\Lie{sl}(2,\C)\times 0$ and $0\times \Lie{sl}(2,\C)$ are ideals in $\Lie{sl}(2,\C)\times \Lie{sl}(2,\C)$ with
  \begin{align*}
    (\slc\times 0) \oplus (0\times\slc) &= \slpr,
  \end{align*}
  so $\Lie{sl}(2,\C)\times \Lie{sl}(2,\C)$ is semisimple. 

  Now if we take $L=\slpr$ and $L_k=\Set{(u,u)\given u\in\slc}$ as a Lie subalgebra, it makes sense to talk about Harish-Chandra modules over $(L,L_k)$. Here $L_k$ is clearly a Lie subalgebra since it is a subspace and the Lie bracket on $\slpr$ preserves $L_k$ by the definition of the Lie bracket on a product. 
\end{remark}

We fix the following as a standard basis for $\Lie{sl}(2,\C)$:
\begin{align}\label{eq:xhy}
  x&=
     \begin{pmatrix}
       0 & 1 \\ 0 & 0
     \end{pmatrix}, \qquad h =
                    \begin{pmatrix}
                      1 & 0 \\ 0 & -1
                    \end{pmatrix}, \qquad y =
                                   \begin{pmatrix}
                                     0 & 0 \\ 1 & 0
                                   \end{pmatrix}.
\end{align}
Giving us the relations:
\begin{align}
  [x,y]&=h, & [h,x]&=2x, & [h,y]&=-2y, \label{eq:sl2rels}
\end{align}
cf.\ \cite[35]{jantzen} or \cite[6]{humphrey}.

We claim now that
\begin{align*}
  (x,x),& \quad (y,y), \quad \tfrac{1}{2}(h,h), \quad (ix,-ix), \quad (iy,-iy), \quad \tfrac{1}{2}(ih,-ih)
\end{align*}
is a basis of $\slpr$. This is clearly the case since $\dim_\C \slc = 3$, so $\dim_\C \slpr = 6$, and we see that the above elements span $\slpr$; we have $\tfrac{1}{2}(x,x)-\tfrac{i}{2}(ix,-ix)=(x,0)$ and $\tfrac{1}{2}(x,x)+\tfrac{i}{2}(ix,-ix)=(0,x)$ and likewise with $h$ and $y$. 

Putting
\begin{align*}
  h_+&=(x,x), & h_-&=(y,y), & h_3&=\tfrac{1}{2}(h,h), \\
  f_+&=(ix,-ix), & f_-&=(iy,-iy),&  f_3&=\tfrac{1}{2}(ih,-ih)
\end{align*}
we get the following commutation relations between these basis elements:
\begin{align}
  \begin{split}
    [h_+,h_3] &= \tfrac{1}{2}([x,h],[x,h]) = \tfrac{1}{2}(-2x,-2x) = -(x,x) = -h_+, \label{eq:lierels} \\
    [h_-,h_3] &= \tfrac{1}{2}([y,h],[y,h]) = \tfrac{1}{2}(2y,2y) = (y,y) = h_-, \\
    [h_+,h_-] &= ([x,y],[x,y]) = (h,h) = 2h_3, \\
    [h_+,f_+] &= ([x,ix],[x,-ix]) = 0, \\
    [h_-,f_-] &= ([y,iy],[y,-iy]) = 0, \\
    [h_3,f_3] &= \tfrac{1}{4}([h,ih],[h,-ih]) = 0, \\
    [h_+,f_3] &= \tfrac{1}{2}([x,ih],[x,-ih]) = \tfrac{1}{2}(-2ix,2ix) = -(ix,-ix) = -f_+, \\
    [h_-,f_3] &= \tfrac{1}{2}([y,ih],[y,-ih]) = \tfrac{1}{2}(2iy,-2iy) = (iy,-iy) = f_-, \\
    [h_+,f_-] &= ([x,iy],[x,-iy]) = (ih,-ih) = 2f_3, \\
    [h_3,f_-] &= \tfrac{1}{2}([h,iy],[h,-iy]) = \tfrac{1}{2}(-2iy,2iy) = -(iy,-iy) = -f_-, \\
    [h_-,f_+] &= ([y,ix],[y,-ix]) = (-ih,ih) = -(ih,-ih) = -2f_3, \\
    [h_3,f_+] &= \tfrac{1}{2}([h,ix],[h,-ix]) = \tfrac{1}{2}(2ix,-2ix) = (ix,-ix) = f_+, \\
    [f_+,f_3] &= \tfrac{1}{2}([ix,ih],[-ix,-ih]) = \tfrac{1}{2}(2x,2x) = (x,x) = h_+, \\
    [f_-,f_3] &= \tfrac{1}{2}([iy,ih],[-iy,-ih]) = \tfrac{1}{2}(-2y,-2y) = -(y,y) = -h_-, \\
    [f_+,f_-] &= ([ix,iy],[-ix,-iy]) = (-h,-h) = -(h,h) = -2h_3.
  \end{split}
\end{align}

\begin{remark}
  Note that these are the same relations as for the complexification of the Lie algebra $L$ of the proper Lorentz group in \cite[5]{indecompReprOfLorGr}, so $L$ is isomorphic to $\slpr$. This explains the equivalence of the work in this paper and the work in \cite{classifOfIndec,catOfHarChaMod,indecompReprOfLorGr}.
\end{remark}

Now in the rest of the paper let $L=\slpr$ and $L_k=\Set{(u,u)\given u\in\slc}$. Note that $L_k$ is the Lie subalgebra of $L$ with basis $h_+,h_-,h_3$, and that the above commutation relations gives us that 
\begin{align*}
  [h_+,h_-] &= 2h_3, & [2h_3,h_+] &= 2h_+, & [2h_3,h_-] &= -2h_-
\end{align*}
Comparing with \cref{eq:sl2rels} we see that we have an isomorphism
\begin{align}
  \slc\to L_k,\qquad u\mapsto (u,u), \label{eq:sl2iso}
\end{align}
or more explicitly $x\mapsto h_+$, $h_-\mapsto y$, and $h\mapsto 2h_3$, so we can use $\slc$-theory when we want to describe $L_k$-modules.

%%% Local Variables:
%%% mode: latex
%%% TeX-master: "../main"
%%% End:
