\section{Representations of \texorpdfstring{$L_k$}{L\_k}}

Let $V$ be a $\C$ vector space and $\rho\colon L_k \to \Lie{gl}(V)$ a representation of $L_k$. We will use the notation $\rho(a)=A$ for $a\in L_k$ switching to upper case letters when we talk about the representation corresponding to a given element. Note that we will switch freely between the language of representations of $L_k$ and the language of $L_k$-modules.

We will start out by describing the finite dimensional simple $L_k$-modules. Recall, cf.\ \cite[36]{jantzen}, that we know from $\slc$-theory that for integers $n\geq 0$ there exists a unique simple $\slc$-module $V(n)$ of dimension $n+1$, and $V(n)$ has a basis $(v_0,v_1,\dotsc,v_n)$ such that for all $i$, $0\leq i\leq n$
\begin{align}
  \begin{split}\label{eq:sl2modbasis}
    h.v_i &= (n-2i)v_i, \\
    x.v_i &=
    \begin{dcases*}
      (n-i+1)v_{i-1} & if $i>0$, \\
      0 & if $i=0$,
    \end{dcases*} \\
    y.v_i &=
    \begin{dcases*}
      (i+1)v_{i+1} & if $i<n$, \\
      0 & if $i=n$.
    \end{dcases*}
  \end{split}
\end{align}

Now using the isomorphism from \cref{eq:sl2iso} we see that for integers $n\geq 0$ there exists a unique simple $L_k$-module $M(n)$ of dimension $n+1$\footnote{We will use the notation $V(n)$ when talking about $\slc$-modules and $M(n)$ when talking about $L_k$-modules to clarify what kind of module we are talking about, but as vector spaces $V(n)$ and $M(n)$ are isomorphic.}, and $M(n)$ has a basis $(v_0,v_1,\dotsc,v_n)$ such that for all $i$, $0\leq i\leq n$
\begin{align}\label{eq:L_kbasis1}
  \begin{aligned}
    h_3.v_i &= (\tfrac{1}{2}n-i)v_i, \\
    h_+.v_i &=
    \begin{dcases*}
      (n-i+1)v_{i-1} & if $i>0$, \\
      0 & if $i=0$,
    \end{dcases*} \\
    h_-.v_i &=
    \begin{dcases*}
      (i+1)v_{i+1} & if $i<n$, \\
      0 & if $i=n$.
    \end{dcases*}
  \end{aligned}
\end{align}
Now consider $M(n)$ as an inner product space over $\C$ with inner product given by
\begin{align}\label{inprodstart}
  \inner{v_k}{v_j} = \delta_{jk}\binom{n}{k}.
\end{align}
We will switch to the orthonormal basis $(\overline v_0,\overline v_1,\dotsc,\overline v_n)$, where $\overline v_i = v_i/\norm{v_i}$. Here $\norm{}$ is given by $\norm{v}=\sqrt{\inner{v}{v}}$ as usually, and we note that
\begin{align*}
  \overline v_i &= \dfrac{1}{\sqrt{\binom{n}{i}}}v_i.
\end{align*}
Note furthermore that
\begin{align*}
  h_3.\overline v_i &= \dfrac{1}{\sqrt{\binom{n}{i}}}h_3.v_i = \dfrac{1}{\sqrt{\binom{n}{i}}}(\tfrac{1}{2}n-i)v_i = (\tfrac{1}{2}n-i)\overline v_i
\end{align*}
for all $i$, $0\leq i\leq n$, and clearly still
\begin{align*}
  h_+.\overline v_0 &= 0, \\
  h_-.\overline v_n &= 0.
\end{align*}
But for $i$, $0<i\leq n$
\begin{align*}
  h_+.\overline v_i &= \dfrac{1}{\sqrt{\binom{n}{i}}}h_+.v_i = \dfrac{1}{\sqrt{\binom{n}{i}}}(n-i+1)v_{i-1} \\
         &= \sqrt{\dfrac{\binom{n}{i-1}}{\binom{n}{i}}}(n-i+1)\dfrac{1}{\sqrt{\binom{n}{i-1}}}v_{i-1} \\  
  &= \sqrt{\dfrac{i}{n-i+1}}(n-i+1)\overline v_{i-1} = \sqrt{(n-i+1)i}\overline v_{i-1},
\end{align*}
and for $i$, $0\leq i<n$
\begin{align*}
  h_-.\overline v_i &= \dfrac{1}{\sqrt{\binom{n}{i}}}h_-.v_i = \dfrac{1}{\sqrt{\binom{n}{i}}}(i+1)v_{i+1} \\
         &= \sqrt{\dfrac{\binom{n}{i+1}}{\binom{n}{i}}}(i+1)\dfrac{1}{\sqrt{\binom{n}{i+1}}}v_{i+1} \\
  &= \sqrt{\dfrac{n-i}{i+1}}(i+1)\overline v_{i+1} = \sqrt{(n-i)(i+1)}\overline v_{i+1}.
\end{align*}

Finally write $\ell = \tfrac{1}{2}n$. We will re-index with $m=\tfrac{1}{2}(n-2i)=\ell-i$ by setting
\begin{align*}
  e_m &= \overline v_{\ell-m}
\end{align*}
for $m\in\Set{-\ell,-\ell+1,\dotsc,\ell-1,\ell}$. Thus we get
\begin{align*}
  h_3.e_m &= h_3.\overline v_{\ell-m} = (\ell-(\ell-m))\overline v_{\ell-m} = me_m,
\end{align*}
and since $e_{\ell} = \overline v_0$ and $e_{-\ell} = \overline v_n$ also
\begin{align*}
  h_+.e_{\ell} &= 0,\\
  h_-.e_{-\ell} &= 0.
\end{align*}
And for $m\in\Set{-\ell,-\ell+1,\dotsc,\ell-2,\ell-1}$ we get
\begin{align*}
  h_+.e_m &= h_+.\overline v_{\ell-m} = \sqrt{(n-(\ell-m)+1)(\ell-m)}\overline v_{\ell-m-1} \\
  &= \sqrt{(\ell+m+1)(\ell-m)}e_{m+1},
\end{align*}
while for $m\in\Set{-\ell+1,-\ell+2,\dotsc,\ell-1,\ell}$ we get
\begin{align*}
  h_-.e_m &= h_-.\overline v_{\ell-m} = \sqrt{(n-(\ell-m))(\ell-m+1)}\overline v_{\ell-m+1} \\
  &= \sqrt{(\ell+m)(\ell-m+1)}e_{m-1}.
\end{align*}

Thus we get the following Lemma:
\begin{lemma}\label{lem:modulebasis}
  Every simple finite dimensional $L_k$-module is uniquely given by a number $\ell\in\tfrac{1}{2}\Z_{\geq 0}$. For such $\ell$ the unique simple $L_k$-module $M(2\ell)$ has dimension $2\ell+1$, and $M(2\ell)$ has a basis $(e_{-\ell},e_{-\ell+1},\dotsc,e_{\ell-1},e_\ell)$ such that for all $m\in\Set{-\ell,-\ell+1,\dotsc,\ell-1,\ell}$ we have
  \begin{align}
    \begin{split}
      h_3.e_m &= me_m,\label{eq:modulerels} \\
      h_+.e_m &=
      \begin{dcases*}
        \sqrt{(\ell+m+1)(\ell-m)}e_{m+1} & if $m\neq \ell$, \\
        0 & if $m=\ell$,
      \end{dcases*} \\
      h_-.e_m &=
      \begin{dcases*}
        \sqrt{(\ell+m)(\ell-m+1)}e_{m-1} & if $m\neq -\ell$, \\
        0 & if $m=-\ell$.
      \end{dcases*}
    \end{split}
  \end{align}
\end{lemma}

\subsection{Formulae for the operators \texorpdfstring{$H_+,H_-,H_3,F_+,F_-,F_3$}{H\_+,H\_-,H\_3,F\_+,F\_-,F\_3}}\label{sec:formulae}

Let $M$ be a Harish-Chandra $L$-module. Then we have linear operators $H_+, H_-, H_3, F_+, F_-, F_3\colon M\to M$ given by the corresponding actions of $L$ satisfying commutation relations as in \cref{eq:lierels}, and we want to give expressions for these in terms of other linear operators $E_+,E_-,D_+,D_-,D_0\colon M\to M$. 

We will denote by $R_\ell$ a finite dimensional $L$-module which is a (finite) direct sum of $L_k$-modules $M(2\ell)$ for the same number $\ell\in\tfrac{1}{2}\Z_{\geq 0}$. Then $M$ is a direct sum of the subspaces $R_\ell$ since $M$ is Harish-Chandra, and from \Cref{lem:modulebasis} we know that $R_\ell$ can be written as the direct sum of subspaces $R_{\ell,m}$, where $R_{\ell,m}$ are eigenspaces for $H_3$ such that
\begin{align} \label{eq:H3eigen}
  H_3\xi &= m\xi
\end{align}
for $m\in\Set{-\ell,-\ell+1,\dotsc,\ell-1,\ell}$ and $\xi\in R_{l,m}$. We will use the decomposition
\begin{align*}
  M &= \bigoplus_{\substack{\ell\in\tfrac{1}{2}\Z_{\geq 0} \\ m\in\Set{-\ell,-\ell+1,\dotsc,\ell-1,\ell}}} \!\!\!\!\!\!\!\!\!\!\!\! R_{\ell,m} = \bigoplus_{\ell,m} R_{\ell,m}
\end{align*}
throughout this paper.

By \Cref{lem:modulebasis} we also have that $H_+$ and $H_-$ maps the $R_{\ell,m}$ into each other as follows:
\begin{align*}
  H_+\colon R_{\ell,m} &\to R_{\ell,m+1} && \mbox{if }-\ell\leq m<\ell, & H_+&\colon R_{\ell,\ell} \to 0, \\
  H_-\colon R_{\ell,m} &\to R_{\ell,m-1} && \mbox{if }-\ell< m\leq \ell, & H_-&\colon R_{\ell,-\ell} \to 0.
\end{align*}
Hence we have linear operators $H_+H_-,H_-H_+\colon R_{\ell,m}\to R_{\ell,m}$, and by \cref{eq:modulerels} we see that
\begin{align}
  \begin{split} \label{eq:H+H-}
    H_+H_-\xi &= \sqrt{(\ell+(m-1)+1)(\ell-(m-1))}\sqrt{(\ell+m)(\ell-m+1)} \xi \\
    &= (\ell+m)(\ell-m+1)\xi, \\
    H_-H_+\xi &= \sqrt{(\ell+(m+1))(\ell-(m+1)+1)}\sqrt{(\ell+m+1)(\ell-m)} \xi \\
    &= (\ell+m+1)(\ell-m)\xi.
  \end{split}
\end{align}
Note that this also covers the cases $m=\ell$ and $m=-\ell$. 

Now we define $E_+ \colon R_{\ell,m}\to R_{\ell,m+1}$ and $E_- \colon R_{\ell,m}\to R_{\ell,m-1}$ to be the linear maps satisfying 
\begin{align}
  \begin{aligned} \label{eq:E+andE-}
    H_+ \xi &= \sqrt{(\ell+m+1)(\ell-m)}E_+\xi && \mbox{if } m\neq \ell, \\
    E_+ \xi &= 0 && \mbox{if } m=\ell, \\
    H_- \xi &= \sqrt{(\ell+m)(\ell-m+1)}E_-\xi && \mbox{if } m\neq -\ell, \\
    E_-\xi &= 0 && \mbox{if } m=\ell,
  \end{aligned}
\end{align}
for $\xi\in R_{\ell,m}$. Comparing \cref{eq:E+andE-} and \cref{eq:H+H-} we see that
\begin{align*}
  E_+E_-\xi &= \xi && \mbox{if }m\neq-\ell \\
  E_-E_+\xi &= \xi && \mbox{if }m\neq\ell.
\end{align*}
Thus $E_+\colon R_{\ell,m}\to R_{\ell,m+1}$ and $E_-\colon R_{\ell,m+1}\to R_{\ell,m}$ are isomorphisms for $m\neq\ell$ and they are each others inverse.
\begin{remark}\label{rem:Edef}
  Note that the above definitions make sense more generally on $L_k$-modules $M$ with a direct sum decomposition $\bigoplus_\ell R_\ell$ such that each $R_\ell$ is a finite direct sum of simple $L_k$-modules isomorphic to $M(2\ell)$, i.e.\ we do not need the additional structure from $L$-modules.

  In particular note that on the $L_k$-module $M(2\ell)$ with basis $(e_{-\ell},e_{-\ell+1},\dotsc,\\ e_{\ell-1},e_\ell)$ as in \Cref{lem:modulebasis}, we have that $E_+ e_m = e_{m+1}$ for $m\neq\ell$ and $E_- e_m = e_{m-1}$ for $m\neq -\ell$.
\end{remark}

Now note that $H_+$, $H_-$, and $H_3$ are completely determined by \cref{eq:H3eigen} and \cref{eq:E+andE-}, so we just need to find maps to determine $F_+$, $F_-$, and $F_3$ now, while making sure that we get commutation relations as in \cref{eq:lierels}.

We already have that $L_k=\Span_\C(h_+,h_-,h_3)$, but now we will also consider $L_p=\Span_\C(f_+,f_-,f_3)$. We will show shortly that $u.R_\ell \subset R_{\ell-1}\oplus R_\ell \oplus R_{\ell+1}$ for all $u\in L_p$. This implies that there are maps $D_-^{u}\colon R_\ell\to R_{\ell-1}$, $D_0^{u}\colon R_\ell\to R_\ell$, and $D_+^{u}\colon R_\ell\to R_{\ell+1}$ such that $u.v=D_-^{u}(v)+D_0^{u}(v)+D_+^{u}(v)$ for all $u\in L_p$ and $v\in R_\ell$. In the following we will find maps $D_-$, $D_0$, and $D_+$ independent of $u$ such that we can express $D_-^{u}$, $D_0^{u}$, and $D_+^{u}$ in terms of these and the maps $E_-$ and $E_+$ from above, thus we will also be able to express $F_+$, $F_-$, and $F_3$ in terms of $D_-$, $D_0$, $D_+$, $E_-$, and $E_+$. To be more precise we will find maps $D_-$, $D_0$, and $D_+$ such that we can express $F_3$ in terms of just these (and multiplication by some constant), and then we can get $F_+$ and $F_-$ by the commutation relations. 

For reasons that will be clearer later, we want the maps $D_+$ to be defined on $M=\bigoplus_{\ell,m} R_{\ell,m}$, $D_0$ defined on the direct sum without the $R_{0,0}$ summand (if there is one), and $D_-$ defined on the direct sum without the summands $R_{\ell,\ell}$ and $R_{\ell,-\ell}$ to be such that $D_0 R_{\ell,m}\subset R_{\ell,m}$, $D_+ R_{\ell,m} \subset R_{\ell+1,m}$, and $D_- R_{\ell,m} \subset R_{\ell-1,m}$ and the diagrams
\begin{center}
  \begin{equation}\label{eq:Ddiagrams}
    \begin{tikzcd}[baseline=(current bounding box.center)]
      R_{\ell-1,m+1} & R_{\ell,m+1} \ar[l,"D_-",swap] & R_{\ell,m+1} \ar[r,"D_0"] & R_{\ell,m+1}  \\
      R_{\ell-1,m} \ar[u,"E_+"] & R_{\ell,m} \ar[l,"D_-"] \ar[u,"E_+",swap] & R_{\ell,m} \ar[r,"D_0",swap] \ar[u,"E_+"] & R_{\ell,m} \ar[u,"E_+",swap] \\
      & R_{\ell,m+1} \ar[r,"D_+"] & R_{\ell+1,m+1} \\
      & R_{\ell,m} \ar[u,"E_+"] \ar[r,"D_+",swap] & R_{\ell+1,m+1} \ar[u,"E_+",swap]
    \end{tikzcd}
  \end{equation}
\end{center}
commute, when $-\ell+1\leq m < \ell-1$ in the top left diagram, $-\ell\leq m<\ell$ in the other two diagrams. Also similar diagrams with $E_-$ replacing $E_+$ commute since $E_-\colon R_{\ell,m}\to R_{\ell,m-1}$ for $m\neq -\ell$ is inverse to $E_+\colon R_{\ell,m-1}\to R_{\ell,m}$. Before we can get to the final description of these maps we need quite a lot of work.\fancybreak{* \quad * \quad * \quad * \quad *}

Note that \cref{eq:lierels} gives us that $[L_k,L_p]\subset L_p$, so by the adjoint representation we can see $L_p$ as an $L_k$-module, and again by \cref{eq:lierels} we see that $L_p$ is a simple $L_k$-module: If $V$ is an $L_k$-submodule and we have a non-zero element $f=af_++bf_-+cf_3\in V$ for some $a,b,c\in\C$ not all zero, then
\begin{align*}
  [h_+,af_++bf_-+cf_3] &= 2bf_3-cf_+, \\
  [h_-,af_++bf_-+cf_3] &= -2af_3+cf_-, \\
  [h_3,af_++bf_-+cf_3] &= af_+-bf_-.
\end{align*}
If $c\neq0$, we get that
\begin{align*}
  [h_3,[h_+,f]] &= [h_3,2bf_3-cf_+] = -cf_+, \\
  [h_3,[h_-,f]] &= [h_3,-2af_3+cf_-] = -cf_-,
\end{align*}
so we see that $f_+,f_-\in V$, and thus also $[h_+,\tfrac{1}{2}f_-]=f_3\in V$, so $V=L_p$. If on the other hand $c=0$, then
\begin{align*}
  [h_-,f] &= -2af_3, \\
  [h_+,f] &= 2bf_3,
\end{align*}
so since either $a\neq 0$ or $b\neq 0$, we see that $f_3\in V$, and thus also $[h_+,-f_3]=f_+\in V$ and $[h_-,f_3]=f_-\in V$, so $V=L_p$. Hence $L_p$ is indeed a simple $L_k$-module. Now since $L_p$ is a simple finite dimensional $L_k$-module of dimension 3, we have that $L_p\simeq M(2)$ as $L_k$-modules.

In general given a Lie algebra $L_1$ and two $L_1$-modules $V$ and $W$, we consider the tensor product $V\otimes W$ over $\C$ of the underlying vector spaces as an $L_1$-module via the action
\begin{align}\label{eq:tensoraction}
  x.(v\otimes w) &= x.v\otimes w + v\otimes x.w,
\end{align}
for $x\in L_1$ and $v\otimes w\in V\otimes W$, cf.\ \cite[26]{humphrey}.

Now we are interested in the $L_k$-module $L_p\otimes M$, where $M$ is a Harish-Chandra $L$-module (and thus an $L_k$-module) as before. Specifically we will show that the linear map
\begin{align}
  \begin{split} \label{eq:psi}
    \psi \colon L_p\otimes M &\to M \\
    x\otimes v &\mapsto x.v
  \end{split}
\end{align}
is a homomorphism of $L_k$-modules. For $y\in L_k$ we see that
\begin{align*}
  y.(x\otimes v) &= y.x\otimes v + x\otimes y.v = [y,x]\otimes v + x\otimes y.v,
\end{align*}
for $x\otimes v\in L_p\otimes M$, since the action in $L_p$ is by the adjoint representation. So
\begin{align*}
  \psi(y.(x\otimes v)) &= \psi([y,x]\otimes v) + \psi(x\otimes y.v) = [y,x].v + x.(y.v) \\
  &= y.(x.v)-x.(y.v)+x.(y.v) = y.(x.v) = y.\psi(x\otimes v),
\end{align*}
i.e.\ $\psi$ is indeed a homomorphism of $L_k$-modules.

Now we note that $M=\bigoplus_{\ell} R_\ell$, so 
\begin{align*}
  L_p\otimes M &= L_p\otimes \bigl( \bigoplus_{\ell} R_\ell \bigr) \simeq \bigoplus_{\ell} (L_p\otimes R_\ell),
\end{align*}
as $L_k$-modules, and since $R_\ell$ is direct sum of finitely many copies of $M(2\ell)$, we see that
\begin{align*}
  L_p\otimes R_\ell &\simeq M(2)\otimes \bigl( M(2\ell)^1 \oplus M(2\ell)^2 \oplus \dotsb \oplus M(2\ell)^r \bigr) \\
  &\simeq (M(2)\otimes M(2\ell)^1) \oplus (M(2)\otimes M(2\ell)^2) \oplus \dotsb \oplus (M(2)\otimes M(2\ell)^r),
\end{align*}
as $L_k$-modules, since $L_p\simeq M(2)$. Here the superscripts are just indices for the different $M(2\ell)$. Thus we want to describe the $L_k$-modules $M(2)\otimes M(2\ell)$, which we will do by first describing the $\slc$-modules $V(2)\otimes V(2\ell)$ and then translating back to a solution to our problem.

\subsection{\texorpdfstring{Describing $V(2)\otimes V(n)$}{Describing V(2) tensor product V(n)}}
Let $2\ell=n\in\N$. We want to show that for $\slc$-modules we have that
\begin{align}\label{eq:V(2)tensorV(n)}
  V(2)\otimes V(n) &\simeq
                     \begin{dcases*}
                       V(n-2) \oplus V(n) \oplus V(n+2) & if $n\geq 2$,\\
                       V(3)\oplus V(1) & if $n=1$, \\
                       V(2) & if $n=0$.
                     \end{dcases*}
\end{align}
Note that in all cases there is a summand $V(n+2)$. We can show the above by considerations using formal characters. We will use the notation of \cite[Chapter~8]{jantzen}, specifically we will do calculations with the functions $e(\lambda)\colon H^*\to \Z$ for $\lambda\in H^*$. Firstly note that in general
\begin{align*}
  \ch_V &= \sum_{\lambda\in H^*} (\dim V_\lambda) e(\lambda),
\end{align*}
and use the notation $V(n)_k$ for $V(\lambda)_\mu$ and $e(n)$ for $e(\lambda)$ with $\lambda,\mu\in H^*$ such that $\lambda(h)=n$ and $\mu(h)=k$. We get that
\begin{align*}
  \ch_{V(2)} &= e(-2) + e(0) + e(2)
\end{align*}
and 
\begin{align*}
  \ch_{V(n)} &= \sum_{i=0}^n e(n-2i),
\end{align*}
since
\begin{align*}
  \dim V(n)_k &=
                \begin{dcases*}
                  1 & if $k=n-2i$ for some $i\in\Set{0,1,\dotsc,n}$, \\
                  0 & otherwise.
                \end{dcases*}
\end{align*}
Now since $e(\lambda)*e(\mu)=e(\lambda+\mu)$ in general cf.\ \cite[93]{jantzen}, we see that for $n\geq 2$
\begin{align*}
  \ch_{V(2)\otimes V(n)} &= \ch_{V(2)} * \ch_{V(n)} = e(-2)*\ch_{V(n)} + e(0)*\ch_{V(n)} + e(2)*\ch_{V(n)} \\
                         &= \sum_{i=0}^n e(n-2-2i) + \ch_{V(n)} + \sum_{i=0}^n e(n+2-2i) \\
                         &= e(-n-2)+e(-n) + \sum_{i=0}^{n-2} e(n-2-2i) + \ch_{V(n)} \\*
                         &\phantom{{}={}}+ \sum_{i=0}^n e(n+2-2i) \\
                         &= \ch_{V(n-2)} + \ch_{V(n)} + \sum_{i=0}^{n+2} e(n+2-2i) \\
  &= \ch_{V(n-2)} + \ch_{V(n)} + \ch_{V(n+2)} = \ch_{V(n-2)\oplus V(n)\oplus V(n+2)},
\end{align*}
where the first equality follows from the fact that $\ch_{V\otimes W}=\ch_V * \ch_W$ in general, cf.\ \cite[125]{humphrey}. Thus since two $\slc$-modules $V$ and $V'$ are isomorphic if and only if $\ch_V=\ch_{V'}$, cf.\ \cite[90]{jantzen}, we see that $V(2)\otimes V(n) \simeq V(n-2)\oplus V(n)\oplus V(n+2)$ if $n\geq 2$.

Likewise we see that
\begin{align*}
  \ch_{V(2)\otimes V(1)} &= \ch_{V(2)}*\ch_{V(1)} \\
  &= \bigl(e(-2)+e(0)+e(2)\bigr)*e(-1) + \bigl(e(-2)+e(0)+e(2)\bigr)*e(1) \\
                         &= e(-3)+e(-1)+e(1)+e(-1)+e(1)+e(3) \\
                         &= \bigl(e(-3)+e(-1)+e(1)+e(3)\bigr) + \bigl(e(-1)+e(1)\bigr) \\
  &= \ch_{V(3)} + \ch_{V(1)} = \ch_{V(3)\oplus V(1)}
\end{align*}
and
\begin{align*}
  \ch_{V(2)\otimes V(0)} &= \ch_{V(2)}*\ch_{V(0)} = \ch_{V(2)}*e(0) = \ch_{V(2)},
\end{align*}
so indeed $V(2)\otimes V(1)\simeq V(3)\oplus V(1)$ and $V(2)\otimes V(0)\simeq V(2)$. 

Now consider $(w_0,w_1,w_2)$ a basis for $V(2)$ and $(v_i \mid 0\leq i\leq n)$ a basis for $V(n)$ such that both satisfies the conditions from \cref{eq:sl2modbasis}. Then for $w_i\otimes v_j \in V(2)\otimes V(n)$ with $i\in\Set{0,1,2}$ and $j\in\Set{0,1,\dotsc,n}$ we see that
\begin{align}
  h.(w_i\otimes v_j) &= h.w_i\otimes v_j + w_i\otimes h.v_j = (2-2i)w_i\otimes v_j + (n-2j)w_i\otimes v_j \nonumber \\
  &= (n-2(i+j-1))w_i\otimes v_j.\label{eq:hontensor}
\end{align}
Hence $v_0\otimes w_0$ is up to scalar multiple the only vector of weight $n+2$ in $V(2)\otimes V(n)$, so it is necessarily a highest weight vector generating the direct summand isomorphic to $V(n+2)$. Note that by \cref{eq:V(2)tensorV(n)} we indeed have a direct summand isomorphic to $V(n+2)$ for all $n\in\N$. By $\slc$-theory, cf.\ \cite[36]{jantzen}, we know that this summand has a basis $(s_k\mid 0\leq k\leq n+2)$ satisfying equations as in \cref{eq:sl2modbasis}, where
\begin{align}\label{eq:s_kbasis}
  s_k &\coloneqq \dfrac{1}{k!} y^k.(w_0\otimes v_0).
\end{align}
By straightforward calculations, cf.\ \Cref{sec:basesofV(2)tensorV(n)}, we get for $n>0$ that 
\begin{align}\label{eq:s_kbasisres}
  \begin{aligned}
    s_0 &= w_0\otimes v_0, \\
    s_1 &= w_1\otimes v_0 + w_0\otimes v_1 \\
    s_k &= w_2\otimes v_{k-2} + w_1\otimes v_{k-1} + w_0\otimes v_k \\
    s_{n+1} &= w_2\otimes v_{n-1} + w_1\otimes v_n \\
    s_{n+2} &= w_2\otimes v_n.
  \end{aligned} &&
                   \begin{aligned}
                     &\phantom{a}\\ &\mbox{if }n>0, \\ &\mbox{for }2\leq k\leq n, \\ &\mbox{if }n>0,\\ &\phantom{a}
                   \end{aligned}
\end{align}

In case $n=0$ we likewise see that $s_1=w_1\otimes v_0$ and $s_2=w_2\otimes v_0$, and we note that $(s_0,s_1,s_2)$ is a basis for $V(2)\otimes V(0)\simeq V(2)$.

Suppose now that $n\geq 1$. Note that by \cref{eq:V(2)tensorV(n)} we have a direct summand isomorphic to $V(n)$, and by \cref{eq:hontensor} the weight space of weight $n$ is spanned by $w_0\otimes v_1$ and $w_1\otimes v_0$, so the vector of highest weight $n$ generating the summand corresponding to $V(n)$ must be of the form $aw_0\otimes v_1 + bw_1\otimes v_0$ for some $a,b\in \C$. Furthermore we know that for this vector generating the summand corresponding to $V(n)$, we must have that
\begin{align*}
  0 &= x.(aw_0\otimes v_1 + bw_1\otimes v_0) \\
    &= ax.w_0 \otimes v_1 + aw_0\otimes x.v_1 + bx.w_1\otimes v_0 + bw_1\otimes x.v_0 \\
    &= 0 + a(n-1+1)w_0\otimes v_0 + b(2-1+1)w_0\otimes v_0 + 0 \\
  &= (an+2b)w_0\otimes v_0, 
\end{align*}
i.e. $an+2b=0$ so $b = -\tfrac{n}{2}a$. This determines the vector generating the summand corresponding to $V(n)$ up to a scalar, so taking $a=1$, we see that we can take
\begin{align*}
  t_0 &\coloneqq w_0\otimes v_1 - \dfrac{n}{2}w_1\otimes v_0
\end{align*}
as our vector generating the summand corresponding to $V(n)$. As before $\slc$-theory now yields that this summand has a basis $(t_k \mid 0\leq k\leq n)$ satisfying equations as in \cref{eq:sl2modbasis}, where
\begin{align}\label{eq:t_kbasisdef}
  t_k \coloneqq \dfrac{1}{k!}y^k.t_0.
\end{align}
By straightforward calculations, cf.\ \Cref{sec:basesofV(2)tensorV(n)}, we get that
\begin{align}\label{eq:t_kbasisres}
  \begin{aligned}
    t_0 &= w_0\otimes v_1 - \dfrac{n}{2}w_1\otimes v_0, \\
    t_k &= (k+1)w_0\otimes v_{k+1} - \dfrac{n-2k}{2}w_1\otimes v_k \\
    &\phantom{{}={}}+ (k-1-n)w_2\otimes v_{k-1} \qquad\qquad\mbox{for }1\leq k\leq n-1,\\
    t_n &= \dfrac{n}{2}w_1\otimes v_n - w_2\otimes v_{n-1}.
  \end{aligned}
\end{align}

Suppose now that $n\geq 2$. By \cref{eq:V(2)tensorV(n)} we have a direct summand isomorphic to $V(n-2)$, and by \cref{eq:hontensor} the weight space of weight $n-2$ is spanned by $w_0\otimes v_2$, $w_1\otimes v_1$, and $w_2\otimes v_0$, so the vector of highest weight $n-2$ generating the summand corresponding to $V(n)$ must be of the form $aw_0\otimes v_2 + bw_1\otimes v_1 + cw_2\otimes v_0$ for some $a,b,c\in\C$. Furthermore we know that for this vector generating the summand corresponding to $V(n-2)$, we must have
\begin{align*}
  0 &= x.(aw_0\otimes v_2 + bw_1\otimes v_1 + cw_2\otimes v_0) \\
    &= aw_0\otimes x.v_2 + bx.w_1\otimes v_1 + bw_1\otimes x.v_1 + cx.w_2\otimes v_0 \\
    &= a(n-2+1)w_0\otimes v_1 + b(2-1+1)w_0\otimes v_1 + b(n-1+1)w_1\otimes v_0 \\
  &\phantom{{}={}}{} + c(2-2+1)w_1\otimes v_0 \\
    &= \bigl((n-1)a + 2b\bigr)w_0\otimes v_1 + \bigl(bn + c\bigr)w_1\otimes v_0,
\end{align*}
i.e. $a(n-1)+2b=0$ and $bn+c=0$. Giving us $c=-bn$ and $b=-\tfrac{n-1}{2}a$, so
\begin{align*}
  c=\dfrac{n(n-1)}{2}a.
\end{align*}
This determines the vector generating the summand corresponding to $V(n-2)$ up to a scalar, so taking $a=1$, we see that we can take
\begin{align*}
  u_0 \coloneqq w_0\otimes v_2 - \dfrac{n-1}{2}w_1\otimes v_1 + \dfrac{n(n-1)}{2}w_2\otimes v_0
\end{align*}
as our vector generating the summand corresponding to $V(n-2)$. Again $\slc$-theory now yields that this summand has a basis $(u_k \mid 0\leq k\leq n-2)$ satisfying equations as in \cref{eq:sl2modbasis}, where 
\begin{align}\label{eq:u_kbasisdef}
  u_k \coloneqq \dfrac{1}{k!}y^k.u_0.
\end{align}
By straightforward calculations, cf.\ \Cref{sec:basesofV(2)tensorV(n)}, we get that
\begin{align}\label{eq:u_kbasisres}
  \begin{aligned}
    u_k &= \dfrac{(k+1)(k+2)}{2}w_0\otimes v_{k+2} - \dfrac{(k+1)(n-k-1)}{2}w_1\otimes v_{k+1} \\
    &\phantom{{}={}}{} + \dfrac{(n-k)(n-k-1)}{2}w_2\otimes v_k
  \end{aligned}
\end{align}
for $0\leq k\leq n-2$. 

Now we want to express $w_1\otimes v_k$ for $0\leq k\leq n$ in terms of the bases $(s_k \mid 0\leq k\leq n+2)$, $(t_k \mid 0\leq k\leq n)$, and $(u_k \mid 0\leq k\leq n-2)$. A straightforward but long calculation, cf.\ \Cref{sec:w_1tensorv_k}, yields that
\begin{align}\label{eq:w_1tensorv_k}
  w_1\otimes v_k = \dfrac{2(k+1)(n+1-k)}{(n+1)(n+2)}s_{k+1} - \dfrac{2(n-2k)}{n(n+2)}t_k - \dfrac{4}{n(n+1)}u_{k-1}
\end{align}
for $0<k<n$, while
\begin{align}\label{eq:special_cases_w_1tensorv_k}
  w_1\otimes v_0 = \dfrac{2}{n+2}(s_1-t_0) \qquad \mbox{and} \qquad w_1\otimes v_n = \dfrac{2}{n+2}(s_{n+1}+t_n)
\end{align}
if $n\geq 1$. If $n=0$ we have already seen (just after \cref{eq:s_kbasisres}) that $w_1\otimes v_0 = s_1$. Note that \cref{eq:special_cases_w_1tensorv_k} is a special case of \cref{eq:w_1tensorv_k} if we set $u_{-1}=u_{n-1}=0$.

Now consider $V(2)$ and $V(n)$ as inner product spaces over $\C$ with inner products given by
\begin{align}\label{eq:inproddef1}
  \inner{w_k}{w_j} = \delta_{jk} \binom{2}{k} \qquad \mbox{and} \qquad \inner{v_k}{v_j} = \delta_{jk}\binom{n}{k}.
\end{align}
Then we can consider $V(2)\otimes V(n)$ an inner product space with inner product given by
\begin{align}\label{eq:inproddef2}
  \inner{w\otimes v}{w'\otimes v'} = \inner{w}{w'}\cdot\inner{v}{v'}
\end{align}
for $w,w'\in V(2)$ and $v,v'\in V(n)$. By straightforward calculations, cf.\ \Cref{sec:innerproductscalc}, we get that
\begin{align}\label{eq:inprodres}
  \inner{s_0}{s_0} = 1, \qquad \inner{t_0}{t_0} = \dfrac{n(n+2)}{2}, \qquad \inner{u_0}{u_0} = \dfrac{n^2(n+1)(n-1)}{4}.
\end{align}

Now set $\overline w_k = w_k/\norm{w_k}$, $\overline v_k = v_k/\norm{v_k}$, $\overline s_k = s_k/\norm{s_k}$, $\overline t_k = t_k/\norm{t_k}$, and $\overline s_k = s_k/\norm{s_k}$ for all possible $k$, where $\norm{}$ is given by $\norm{v}=\sqrt{\inner{v}{v}}$ as usually in an inner product space. Note, cf.\ \Cref{sec:innerproductscalc}, that
\begin{align}\label{eq:inprodres2}
  \begin{aligned}
    \inner{w_k}{w_k} &= \binom{2}{k} \\
    \inner{v_k}{v_k} &= \binom{n}{k} \\
    \inner{s_k}{s_k} &= \inner{s_0}{s_0}\binom{n+2}{k} = \binom{n+2}{k} \\
    \inner{t_k}{t_k} &= \inner{t_0}{t_0}\binom{n}{k} = \dfrac{n(n+2)}{2}\binom{n}{k} \\
    \inner{u_k}{u_k} &= \inner{u_0}{u_0}\binom{n-2}{k} = \dfrac{n^2(n+1)(n-1)}{4}\binom{n-2}{k}
  \end{aligned}
\end{align}
for $k$ where it makes sense, so we see that
\begin{align}\label{eq:normedbases1}
  w_k &= \sqrt{\binom{2}{k}}\overline w_k, & v_k &= \sqrt{\binom{n}{k}}\overline v_k, & s_k &= \sqrt{\binom{n+2}{k}}\overline s_k,
\end{align}
and
\begin{align}\label{eq:normedbases2}
  t_k &= \sqrt{\dfrac{n(n+2)}{2}\binom{n}{k}}\overline t_k, & u_k &= \sqrt{\dfrac{n^2(n+1)(n-1)}{4}\binom{n-2}{k}}\overline u_k.
\end{align}

\begin{remark}\label{rem:changeindex}
  Since
  \begin{align*}
    \overline v_k = \dfrac{1}{\sqrt{\binom{n}{k}}} v_k,
  \end{align*}
  we note that we just need to change indices to go to the basis $(e_m)$ from the basis of $(v_k)$ as in the work leading to \Cref{lem:modulebasis}.
\end{remark}

By a simple calculation, cf.\ \Cref{sec:barw_1tensorbarv_k}, we get that
\begin{align}\label{eq:barw_1tensorbarv_k}
  \begin{aligned}
    \overline w_1 \otimes \overline v_k &= \sqrt{\dfrac{2(k+1)(n+1-k)}{(n+1)(n+2)}}\overline s_{k+1} - \dfrac{n-2k}{\sqrt{n(n+2)}}\overline t_k \\*
    &\phantom{{}={}}{} - \sqrt{\dfrac{2k(n-k)}{n(n+1)}}\overline u_{k-1}.
  \end{aligned}
\end{align}
for $0\leq k\leq n$. Now changing indices as mentioned in \Cref{rem:changeindex} to $\ell = \tfrac{1}{2}n$ and $m=\tfrac{1}{2}(n-2k)=\ell-k$ as we did to get to \Cref{lem:modulebasis}, i.e.\ $n=2\ell$ and $k=\ell-m$, we get that
\begin{align*}
  \begin{aligned}
    \overline w_1\otimes e_m &= \overline w_1\otimes\overline v_k \\
    &= \sqrt{\dfrac{2(\ell-m+1)(2\ell+1)-(\ell-m)}{(2\ell+1)(2\ell+2)}}\overline s_{k+1} - \dfrac{(2\ell-2(\ell-m))}{\sqrt{2\ell(2\ell+2)}}\overline t_{k} \\*
    &\phantom{{}={}}{} - \sqrt{\dfrac{2(\ell-m)(2\ell-(\ell-m))}{2\ell(2\ell+1)}}\overline u_{k-1} \\
    &= \sqrt{\dfrac{(\ell-m+1)(\ell+1+m)}{(2\ell+1)(\ell+1)}}\overline s_{k+1} - \dfrac{m}{\sqrt{\ell(\ell+1)}}\overline t_{k} \\*
    &\phantom{{}={}}{} - \sqrt{\dfrac{(\ell-m)(\ell+m)}{\ell(2\ell+1)}}\overline u_{k-1},
  \end{aligned}
\end{align*}
where $e_m$ is as in the work we did to get \Cref{lem:modulebasis} except for the fact that we consider $\slc$-modules still. Now setting
\begin{align*}
  \widetilde D_+(\overline v_k) &= -\dfrac{\overline s_{k+1}}{\sqrt{(\ell+1)(2\ell+1)}}, & \widetilde D_0(\overline v_k) &= \dfrac{\overline t_k}{\sqrt{\ell(\ell+1)}}, & \widetilde D_-(\overline v_k) &= -\dfrac{\overline u_{k-1}}{\sqrt{\ell(2\ell+1)}},
\end{align*}
we see that
\begin{align}\label{eq:Drelstilde}
  \begin{aligned}
    \overline w_1 \otimes e_m &= \overline w_1\otimes\overline v_k \\
    &=\sqrt{(\ell+1)^2-m^2}\dfrac{\overline s_{k+1}}{\sqrt{(\ell+1)(2\ell+1)}} - m\dfrac{\overline t_k}{\sqrt{\ell(\ell+1)}} \\*
    &\phantom{{}={}}{} - \sqrt{\ell^2-m^2}\dfrac{\overline u_{k-1}}{\ell(2\ell+1)} \\
    &= \sqrt{\ell^2-m^2}\widetilde D_-(\overline v_k) - m\widetilde D_0(\overline v_k) - \sqrt{(\ell+1)^2-m^2}\widetilde D_+(\overline v_k).
  \end{aligned}
\end{align}
Note that for $m\in\Set{\pm\ell}$ the $\widetilde D_-$ term vanishes, so the formula works here although $D_-$ is not well-defined in these edge cases.\fancybreak{* \quad * \quad * \quad * \quad *}

Getting back to the problem at the end of \Cref{sec:formulae}, we want to give the maps $D_0$, $D_+$, and $D_-$ such that $D_0R_{\ell,m}\subset R_{\ell,m}$, $D_+R_{\ell,m}\subset R_{\ell+1,m}$, and $D_-R_{\ell,m}\subset R_{\ell-1,m}$, the diagrams of \cref{eq:Ddiagrams} commute, and we can describe $F_3$, $F_+$, $F_-$ by the maps $D_0$, $D_+$, $D_-$, $E_+$, and $E_-$. Now consider the $\slc$-modules $V(n)$ as $L_k$-modules $M(n)$ via the isomorphism of \cref{eq:sl2iso}, and note that since
\begin{align*}
  R_\ell = M(2\ell)^1 \oplus M(2\ell)^2 \oplus \dotsb \oplus M(2\ell)^r
\end{align*}
and each $M(2\ell)^{i}$ has a basis $(e_{-\ell}^{i},e_{-\ell+1}^{i},\dotsc,e_{\ell-1}^{i},e_{\ell}^{i})$ with $H_3e_m^{i} = m e_m^{i}$ for all $m$, we have that $R_{\ell,m}$ has basis $(e_m^1,e_m^2,\dotsc,e_m^r)$ by definition. So when describing the maps $D_0$, $D_+$, and $D_-$, we just need to describe what the maps should do to each $e_m^{i}$. We already know that $E_+e_m^{i}=e_{m+1}^{i}$ and $E_-e_m^{i}=e_{m-1}^{i}$ where it makes sense, so if the maps $D_0$, $D_+$, and $D_-$ do not depend on $m$ or $i$, we get the commutative diagrams of \cref{eq:Ddiagrams}, thus we want to describe what each map does to $M(2\ell)$ in general, so we will stop writing the superscripts.

Since we want to describe the maps $F_3$, $F_+$, and $F_-$, we are actually interested in the actions of $L_p$, so by using $\psi$ of \cref{eq:psi} and the considerations at the end of \Cref{sec:formulae}, we can start out by describing the $L_k$-module $M(2)\otimes M(2\ell)$, i.e.\ we can use the description of $V(2)\otimes V(n)$ from above. Note that we have already seen that $L_p\simeq M(2)$ as $L_k$-modules, but we would like to better understand how the basis $(f_+,f_3,f_-)$ of $L_p$ corresponds to the basis $(w_0,w_1,w_2)$ of $M(2)$ as in \cref{eq:L_kbasis1}. In the basis $(w_0,w_1,w_2)$ we have that $h_+ . w_0 = 0$ (since this is what corresponds to $x.w_0=0$ in $V(2)$ by \cref{eq:sl2iso}), so by checking \cref{eq:lierels} we see that $w_0$ must correspond to a multiple of $f_3$, but the basis is chosen up to scalar, so we can take $w_0$ to be $-\tfrac{\sqrt{2}}{2}f_3$. Now we get $w_1$ by taking $h_- . w_0$ (corresponding to $y.w_0$ in $V(2)$ by \cref{eq:sl2iso}), thus we get that
\begin{align*}
  w_1 = h_-.w_0 = -\tfrac{\sqrt{2}}{2}h_- . f_+ = -\tfrac{\sqrt{2}}{2}[h_-,f_+] = \sqrt{2}f_3.
\end{align*}
Likewise we get that $w_2=[h_-,\sqrt{2}f_3]=\sqrt{2}f_-$, so we can take our basis to be $(w_0,w_1,w_2)=(-\tfrac{\sqrt{2}}{2}f_+,\sqrt{2}f_3,\sqrt{2}f_-)$ when thinking of $L_p$ as the $L_k$-module $M(2)$. Normalizing as in \cref{eq:normedbases1}, we get that $(\overline w_0,\overline w_1,\overline w_2)=(-\tfrac{\sqrt{2}}{2}f_+,f_3,\sqrt{2}f_-)$. So by \cref{eq:Drelstilde}, we see that in $L_p\otimes M(2\ell)$
\begin{align*}
  f_3 \otimes e_m &= \sqrt{\ell^2-m^2}\widetilde D_-(e_m) - m\widetilde D_0(e_m) - \sqrt{(\ell+1)^2-m^2}\widetilde D_+(e_m),
\end{align*}
where $e_m=\overline v_k$ for $k=\ell-m$ and $f_3 = \overline w_1$.
\begin{remark}\label{rem:baseswithindexchange}
  Note that if we have bases $(e_{-\ell-1}^{(2\ell+2)},e_{-\ell}^{(2\ell+2)},\dotsc,e_{\ell}^{(2\ell+2)},e_{\ell+1}^{(2\ell+2)})$ for $M(2\ell+2)$, $(e_{-\ell}^{(2\ell)},e_{-\ell+1}^{(2\ell)},\dotsc,e_{\ell-1}^{(2\ell)},e_{\ell}^{(2\ell)})$ for $M(2\ell)$, and $(e_{-\ell+1}^{(2\ell-2)},e_{-\ell+2}^{(2\ell-2)},\dotsc,\\ e_{\ell-2}^{(2\ell-2)},e_{\ell-1}^{(2\ell-2)})$ for $M(2\ell-2)$ (if $\ell\geq 2$) as in \Cref{lem:modulebasis}, then as above changing indices with $k=\ell+1-m$ we see that $e_m^{(2\ell+2)}$ corresponds to $\overline s_k$. Likewise changing indices with $k=\ell-m$ we see that $e_m^{(2\ell)}$ corresponds to $\overline t_k$, and with $k=\ell-1-m$ we see that $e_m^{(2\ell-2)}$ corresponds to $\overline u_k$. 
\end{remark}
By this remark together with \Cref{rem:Edef}, we see that $E_+ \overline v_k = \overline v_{k-1}$ and $E_+ \overline s_k = \overline s_{k-1}$ where it makes sense, so by the definition of $\widetilde D_+$ it commutes with $E_+$ and $E_-$. Similarly we can see that $\widetilde D_0$ and $\widetilde D_-$ commute with $E_+$ and $E_-$. Note that it makes sense to talk about using $E_+$ and $E_-$ here by \Cref{rem:Edef}.

Now using $\psi$ from \cref{eq:psi}, we see that
\begin{align}\label{eq:F_3action}
  \begin{aligned}
    F_3e_m &= f_3 . e_m = \psi(f_3\otimes e_m) \\
    &= \sqrt{\ell^2-m^2}\psi \widetilde D_-(e_m) - m\psi \widetilde D_0(e_m) - \sqrt{(\ell+1)^2-m^2}\psi \widetilde D_+(e_m).
  \end{aligned}
\end{align}
So we can take $D_0=\psi\widetilde D_0$, $D_+ = \psi\widetilde D_+$, and $D_- = \psi\widetilde D_-$ to get three linear maps with which we can describe the map $F_3$. So far this is just maps on $M(2\ell)$, but we can expand to maps on $R_\ell$ by using the maps on each summand of $R_\ell=M(2\ell)^1 \oplus \dotsb \oplus M(2\ell)^r$, and likewise we can expand further to maps on $M=\bigoplus_\ell R_\ell$ by using the maps on each summand. Also indeed $D_0R_{\ell,m}\subset R_{\ell,m}$, $D_+R_{\ell,m}\subset R_{\ell+1,m}$, and $D_-R_{\ell,m}\subset R_{\ell-1,m}$, since for $\xi\in R_{\ell,m}$ we have that
\begin{align*}
  H_3D_0(\xi) &= h_3 . \psi\widetilde D_0(\xi) = \psi h_3 . \widetilde D_0(\xi) = \psi H_3 \widetilde D_0(\xi) = m\psi \widetilde D_0(\xi) \\
  &= m D_0(\xi),
\end{align*}
since $\psi$ is an $L_k$-module homomorphism and by \Cref{rem:baseswithindexchange} we see that $\widetilde D_0(e_m)$ is a scalar multiple of $\overline t_k = \overline t_{\ell-m} = e_m^{(2\ell)}$, and indeed $H_3e_m^{(2\ell)}=me_m^{(2\ell)}$. The same reasoning with $\overline s_{k+1}$ for $D_+$ and $\overline u_{k-1}$ for $D_-$ yields the other two inclusions. Also note that the diagrams of \cref{eq:Ddiagrams} commute by the defintion of $D_0$, $D_+$, and $D_-$, since we have already shown that $\widetilde D_0$, $\widetilde D_+$, and $\widetilde D_-$ commute with $E_+$ and $E_-$ where it makes sense, and $L_k$-module homomorphisms in general commute with $E_+$ and $E_-$, so also $\psi$ commutes with $E_+$ and $E_-$. This is the case since for an $L_k$-module homomorphism $\varphi\colon M\to M'$ with decompositions $M=\bigoplus_{\ell,m} R_{\ell,m}$ and $M'=\bigoplus_{\ell,m} R_{\ell,m}'$, we have that $\gamma R_{\ell,m} \subset R_{\ell,m}'$, so since $\gamma$ commutes with $H_+$ and $H_-$ this implies by \cref{eq:E+andE-} that $\gamma$ commutes with $E_+$ and $E_-$. To see that $\gamma R_{\ell,m}\subset R_{\ell,m}'$ first note that $R_\ell$ and $R_\ell'$ are direct sums of finitely many $L_k$-modules $M(2\ell)$, where each $M(2\ell)$ is generated by a maximal vector of weight $\ell$ (weight w.r.t. $h_3$ in $L_k$). So since $\gamma$ takes a maximal vector of weight $\ell$ to either zero or another maximal vector of weight $\ell$, we see that indeed $\gamma R_\ell \subset R_\ell'$. Finally since $\gamma\in\Hom_{L_k}(M,M')$ we have that $\gamma H_3 = H_3' \gamma$, so vectors of weight $m$ go to $0$ or vectors of weight $m$, and thus $\gamma R_{\ell,m} \subset R_{\ell,m}'$.

\begin{remark}\label{rem:D0andD-}
  Note that $D_0$ is not defined at $R_{0,0}$, but $D_0$ is defined on $R_{\ell,0}$ by using the relation $E_+D_0=E_+D_0$ repeatedly. Also note that $D_-$ is not defined at $R_{\ell,\pm \ell}$. When we sometimes later use $D_0$ on $R_{0,0}$, it is taken to be given by multiplication by some constant $\dd{0}{0}$, and also we will use the convention that $D_-$ is $0$ on $R_{\ell_0,m}$.
\end{remark}

Now simple calculations, cf.\ \Cref{sec:Factions}, gives us that
\begin{align}\label{eq:Factions}
  \begin{aligned}
    F_3 \xi &= \sqrt{\ell^2-m^2} D_-\xi - m D_0\xi - \sqrt{(\ell+1)^2-m^2}D_+\xi, \\
    F_+ \xi &= \sqrt{(\ell-m)(\ell-m-1)}D_-E_+\xi - \sqrt{(\ell-m)(\ell+m+1)}D_0E_+\xi \\*
    &\phantom{{}={}}{} + \sqrt{(\ell+m+1)(\ell+m+2)}E_+D_+\xi, \\
    F_- \xi &= -\sqrt{(\ell+m)(\ell+m-1)}D_-E_-\xi - \sqrt{(\ell+m)(\ell-m+1)}D_0E_-\xi \\*
    &\phantom{{}={}}{} - \sqrt{(\ell-m+1)(\ell-m+2)}E_-D_+\xi
  \end{aligned}
\end{align}
for $\xi\in R_{\ell,m}$. Note here that although $D_-$ is not defined on $R_{\ell,\ell}$ and $R_{\ell,-\ell}$ the above still makes sense since in these cases the terms with $D_-$ vanish, either by the coefficient being zero or by $E_+$ or $E_-$ mapping to zero.

We claim now that the formulae \cref{eq:Factions} for the linear operators $F_+$, $F_-$, and $F_3$ together with the formulae \cref{eq:H3eigen,eq:E+andE-} for $H_+$, $H_-$, and $H_3$ define a representation of $L$, i.e. they satisfy the commutation relations of \cref{eq:lierels}, if and only if $D_0$, $D_+$, and $D_-$ satisfy
\begin{align}\label{eq:Drels}
  \begin{aligned}
    \ell D_+D_0 \xi &= (\ell+2)D_0 D_+ \xi, \\
    (\ell+1)D_-D_0 \xi &= (\ell-1)D_0D_-\xi, \\
    \xi &= (2\ell-1)D_+D_- \xi - (2\ell+3)D_-D_+\xi - D_0^2\xi
  \end{aligned}
\end{align}
for $\xi\in R_{\ell,m}$, $\ell\neq 0$ except in the first equation. Here the term $D_+D_-$ should be replaced with $E_+D_+D_-E_-$ for $\xi\in R_{\ell,\ell}$ and $E_-D_+D_-E_+$ for $\xi \in R_{\ell,-\ell}$.

\subsection{\texorpdfstring{Simple Harish-Chandra modules for the pair $(L,L_k)$}{Simple Harish-Chandra modules for the pair (L,L\_k)}}\label{sec:simplemodules}

We are now almost ready to classify the simple Harish-Chandra modules for the pair $(L,L_k)$, but before most of the work we need some basic results.

Let $M$ be an simple Harish-Chandra module over $L$ and suppose that each non-trivial subspace $R_{\ell,m}$ in $M=\bigoplus_{\ell,m} R_{\ell,m}$ is one dimensional. In this case each $L_k$-module $R_\ell$ is either isomorphic to the simple module $M(2\ell)$ or $0$. We claim that actually all simple Harish-Chandra modules are of this kind, so we indeed get a classification of the simple Harish-Chandra modules in the following.

To see this assume that $V\subset R_\ell$ is a simple $L_k$-submodule in $M$, i.e.\ $V\simeq M(2\ell)$, and put
\begin{align*}
  \overline{D_+V} = D_+(V) + E_+(D_+(V)) + E_-(D_+(V)).
\end{align*}
We claim that $\overline{D_+V}$ is either a simple $L_k$-submodule or $0$. This is clear since by construction if $\overline{D_+V}$ is non-zero, it sits inside an $L_k$-submodule isomorphic to $M(2\ell+2)$, and here we have the weight $m$ elements for $m\in\Set{-\ell,-\ell+1,\dotsc,\ell-1,\ell}$ sitting in $D_+(V)$, the $-\ell-1$ weight elements in $E_-(D_+(V))$, and the $\ell+1$ elements in $E_+(D_+(V))$, so $\overline{D_+V}\simeq M(2\ell+2)$. Now if $D_0(V)\subset V$ then \cref{eq:Drels} imply that $D_0(\overline{D_+V})\subset \overline{D_+V}$ since $E_+$ and $E_-$ commute with $D_0$. Likewise if $D_+D_-(V)\subset V$ we get that $D_-(\overline{D_+V})\subset V$ and $D_+D_-(\overline{D_+V}) \subset \overline{D_+V}$.

Now consider an eigenvector $v\in R_{\ell_0,\ell_0}$ of $D_0$ (or of $\Delta_2$ in case $\ell_0=0$), and let $V_{\ell_0}$ be the subspace spanned by all $(E_-)^rv$. This is clearly a simple $L_k$-module isomorphic to $M(2\ell_0)$ (up to constants on the terms it is exactly the construction of $M(2\ell_0)$). Inductively define $V_\ell = \overline{D_+V_{\ell-1}}$ for all $\ell>\ell_0$. Then $\bigoplus_{\ell\geq \ell_0} V_\ell$ is an $L$-submodule of $M$ since by definition it is invariant under $D_+$, $E_+$, and $E_-$, and by \cref{eq:Drels} $D_0(V_{\ell_0})\subset V_{\ell_0}$ and $D_+D_-(V_{\ell_0})\subset V_{\ell_0}$ so $\bigoplus_{\ell\geq \ell_0} V_\ell$ is also invariant under $D_0$ and $D_-$ by the above. Hence if $M$ is simple, we get that $M=\bigoplus_{\ell\geq \ell_0} V_\ell$, and thus indeed each non-trivial subspace $R_{\ell,m}$ in $M$ is one dimensional.

Denote by $\ell_0$ the minimal index $\ell$ in the decomposition $M=\bigoplus_\ell R_\ell$. Note that
\begin{align*}
  M' = \bigoplus_{\ell'\in\Set{\ell_0,\ell_0+1,\dotsc}} R_{\ell'}
\end{align*}
is invariant under $E_+$, $E_-$, $D_0$, $D_+$, and $D_-$, so by the formulae \cref{eq:Factions} for $F_+$, $F_-$, and $F_3$, we see that $M'$ is a submodule since we already know that it is an $L_k$-submodule because $R_{\ell'}$ all are $L_k$-submodules. Thus $M'=M$ since $M$ is simple and hence the index $\ell$ in $M=\bigoplus_\ell R_\ell$ range over only integral values or only half-integral values.

Additionally we want to show that the kernel of the map $D_-\colon M\to M$ is $R_{\ell_0}$. To do this assume for contradiction that $D_-R_{\ell',m_0}=0$ for some index $\ell'>\ell_0$ and $m_0\in\Set{-\ell_0,-\ell_0+1,\dotsc,\ell_0-1,\ell_0}$. Then by the commutative diagram in \cref{eq:Ddiagrams} with $D_-$, i.e.\ $D_-E_+=E_+D_-$, and the fact that $E_+\colon R_{\ell',m}\to R_{\ell',m+1}$ is an isomorphism for $m<\ell'$, we see that $D_-R_{\ell',m}=0$ for all $m\in\Set{-\ell',-\ell'+1,\dotsc,\ell'-1,\ell'}$. But then
\begin{align*}
  M'' = \bigoplus_{\ell''\in\Set{\ell',\ell'+1,\dotsc}} R_{\ell''}
\end{align*}
is a proper $L$-submodule of $M$, which contradicts the simplicity of $M$. Thus indeed $\ker D_- = R_{\ell_0}$.

Likewise we see that if $M$ is infinite dimensional, then $D_+\colon M\to M$ has trivial kernel since if $D_+R_{\ell'}=0$, then $M=\bigoplus_{\ell\in\Set{\ell_0,\ell_0+1,\dotsc}} R_{\ell}$ is finite dimensional. This is the case since all terms with $\ell>\ell'$ must be trivial since otherwise
\begin{align*}
  M'' = \bigoplus_{\ell''\in\Set{\ell_0,\ell_0+1,\dotsc,\ell'}} R_{\ell''}
\end{align*}
is a proper $L$-submodule of $M$, which contradicts the simplicity of $M$.

\subsubsection{Infinite dimensional simple modules}

Assume that $M$ is a Harish-Chandra module of the above kind and is infinite dimensional. Because all $R_{\ell,m}$ are one dimensional, the diagram with $E_+$ and $D_+$ in \cref{eq:Ddiagrams} commute, i.e.\ $D_+E_+=E_+D_+$, and $D_+$ has trivial kernel, while $E_+$ is an isomorphism for $m\neq \ell$, we see that we can choose a basis $\Set{\xi_{\ell,m}}$ of $M$ such that $\xi_{\ell,m}\in R_{\ell,m}$ and
\begin{align*}
  E_+ \xi_{\ell,m} &= \xi_{\ell,m+1} && \mbox{for }-\ell\leq m<\ell,\\
  D_+ \xi_{\ell,m} &= \xi_{\ell+1,m} && \mbox{for }\ell\in\Set{\ell_0,\ell_0+1,\dotsc}.
\end{align*}
In this basis we get that
\begin{align}\label{eq:EandDnewbasis}
  \begin{aligned}
    E_- \xi_{\ell,m} &= \xi_{\ell,m-1} && \mbox{for }-\ell<m\leq \ell, \\
    D_0 \xi_{\ell,m} &= d_\ell^0 \xi_{\ell,m} && \mbox{for }\ell\in\Set{\ell_0,\ell_0+1,\dotsc},\\
    D_- \xi_{\ell,m} &= d_\ell^- \xi_{\ell-1,m} && \mbox{for }\ell\in\Set{\ell_0+1,\ell_0+2,\dotsc},\\
    D_- \xi_{\ell_0,m} &= 0,
  \end{aligned}
\end{align}
where the first equation comes from the fact that $E_-\colon R_{\ell,m}\to R_{\ell,m-1}$ for $m\neq -\ell$ is the inverse of $E_+\colon R_{\ell,m-1}\to R_{\ell,m}$, while the independence of $m$ in the other equations comes from the commutativity of the diagrams of \cref{eq:Ddiagrams}. Note that we here unlike normally define $D_0$ on $R_{0,0}$ by multiplication by some constant $\dd{0}{0}$.

Now \cref{eq:Drels,eq:EandDnewbasis} implies that
\begin{align}\label{eq:drelations}
  \begin{aligned}
    \ell \dd{\ell}{0} &= (\ell+2)\dd{\ell+1}{0}, \\
    (\ell+1)\dd{\ell}{-}\dd{\ell}{0} &= (\ell-1)\dd{\ell-1}{0}\dd{\ell}{-}, \\
    1 &= (2\ell-1)\dd{\ell}{-} - (2\ell+3)\dd{\ell+1}{-} - (\dd{\ell}{0})^2,\\
    \dd{\ell_0}{-} &= 0,
  \end{aligned}
\end{align}
for $\ell\in\Set{\ell_0,\ell_0+1,\dotsc}$, $\ell\neq 0$ except in the first equation, and in the second equation we also demand that $\ell>\ell_0$. We see that
\begin{align*}
  \dd{\ell+1}{0} = \dfrac{\ell}{\ell+2}\dd{\ell}{0}.
\end{align*}
So if $\ell_0\neq0$, then for some constant $c$ 
\begin{align*}
  \dd{\ell_0}{0} = \dfrac{c}{\ell_0(\ell_0+1)},
\end{align*}
so we see inductively that if
\begin{align}\label{eq:dl0}
  \dd{\ell}{0} = \dfrac{c}{\ell(\ell+1)},
\end{align}
then
\begin{align*}
  \dd{\ell+1}{0} &= \dfrac{\ell}{\ell+2}\dd{\ell}{0} = \dfrac{\ell}{\ell+2}\dfrac{c}{\ell(\ell+1)} \\
  &= \dfrac{c}{(\ell+1)(\ell+2)}.
\end{align*}
Thus if $\ell_0\neq 0$ \cref{eq:dl0} holds true in general for some constant $c$. 

If on the other hand $\ell_0=0$, then we see that
\begin{align*}
  2\dd{\ell_0+1}{0} = 0,
\end{align*}
so $\dd{\ell_0+1}{0}=0$, and thus
\begin{align*}
  \dd{\ell}{0} = \dfrac{\ell-1}{\ell+1}\dd{\ell-1}{0} = 0
\end{align*}
for all $\ell\in\Set{1,2,\dotsc}$. Also in this case have $\dd{0}{0}=c_1$, where $c_1$ is some constant. 

To unify these two cases we set $c=i\ell_0\ell_1$ and $c_1=i\ell_1$ for some complex constant $\ell_1$ such that
\begin{align}\label{eq:dl0res}
  \dd{\ell}{0} = \dfrac{i\ell_0\ell_1}{\ell(\ell+1)}
\end{align}
for $\ell\in\Set{\ell_0,\ell_0+1,\dotsc}$. Substituting this expression with $\dd{\ell}{0}$ in the third equation of \cref{eq:drelations} we get that
\begin{align*}
  (2\ell-1)\dd{\ell}{-} - (2\ell+3)\dd{\ell+1}{-} &= 1 - \dfrac{\ell_0^2\ell_1^2}{\ell^2(\ell+1)^2},
\end{align*}
and a simple calculation, cf.\ \Cref{sec:dl-calc}, yields that
\begin{align}\label{dl-res}
  \dd{\ell}{-} = -\dfrac{\bigl(\ell^2 - \ell_1^2\bigr)\bigl( \ell^2 - \ell_0^2 \bigr)}{\ell^2(4\ell^2-1)},
\end{align}
for $\ell>\ell_0$.

Since we showed in the beginning of this subsection that the kernel of $D_-$ is $R_{\ell_0}$, we must have that $\dd{\ell}{-}\neq 0$ for all $\ell>\ell_0$. Thus $\ell^2-\ell_1^2\neq 0$ for $\ell>\ell_0$, so $\abs{\ell_1}-\ell_0$ cannot be a positive integer if $\ell_1$ is real, because if that was the case then $\abs{\ell_1}= \ell_0+(\abs{\ell_1}-\ell_0) \in \Set{\ell_0+1,\ell_0+2,\dotsc}$, but $\abs{\ell_1}^2-\ell_1^2=0$ since $\ell_1\in\R$.

Hence altogether by \cref{eq:E+andE-,eq:Factions} in the basis $\Set{\xi_{\ell,m}}$ the operators $H_+$, $H_-$, $H_3$, $F_+$, $F_-$, and $F_3$ are given by the formulae
\begin{align}\label{eq:infdimactions}
  \begin{aligned}
    H_3\xi_{\ell,m} &= m\xi_{\ell,m}, \\
    H_+\xi_{\ell,m} &= \sqrt{(\ell+m+1)(\ell-m)} \xi_{\ell,m+1}, \\
    H_-\xi_{\ell,m} &= \sqrt{(\ell+m)(\ell-m+1)} \xi_{\ell,m-1}, \\
    F_3\xi_{\ell,m} &= \sqrt{\ell^2-m^2} \dd{\ell}{-}\xi_{\ell-1,m} - m \dd{\ell}{0}\xi_{\ell,m} - \sqrt{(\ell+1)^2-m^2}\dd{\ell}{+}\xi_{\ell+1,m}, \\
    F_+\xi_{\ell,m} &= \sqrt{(\ell-m)(\ell-m-1)}\dd{\ell}{-}\xi_{\ell-1,m+1} - \sqrt{(\ell-m)(\ell+m+1)}\dd{\ell}{0}\xi_{\ell,m+1} \\*
    &\phantom{{}={}}{} + \sqrt{(\ell+m+1)(\ell+m+2)} \dd{\ell}{+}\xi_{\ell+1,m+1}, \\
    F_-\xi_{\ell,m} &= -\sqrt{(\ell+m)(\ell+m-1)}\xi_{\ell-1,m-1} - \sqrt{(\ell+m)(\ell-m+1)}\xi_{\ell,m-1} \\*
    &\phantom{{}={}}{} - \sqrt{(\ell-m+1)(\ell-m+2)}\xi_{\ell+1,m-1},
  \end{aligned}
\end{align}
where
\begin{align}\label{eq:infdimds}
  \dd{\ell}{0} &= \dfrac{i\ell_0\ell_1}{\ell(\ell+1)}, & \dd{\ell}{-} &= -\dfrac{\bigl(\ell^2 - \ell_1^2\bigr)\bigl( \ell^2 - \ell_0^2 \bigr)}{\ell^2(4\ell^2-1)}, & \dd{\ell}{+} &= 1,
\end{align}
for $\ell\in\Set{\ell_0,\ell_0+1,\dotsc}$, and where $\ell_1$ is a complex number such that $\abs{\ell_1}-\ell_0$ is not a positive integer if $\ell_1$ is real. Here we use the convention that $\xi_{\ell',m'}=0$ for pairs $\ell',m'$ where there is no such basis element. Also since the formuale of \cref{eq:Drels} hold by definition, we see that given such maps we indeed get a simple $L$-module.

\subsubsection{Finite dimensional simple modules}

Assume that $M$ is a Harish-Chandra module of the above kind and that $M$ is finite dimensional, i.e.\ $M=\bigoplus_{\ell,m} R_{\ell,m}$ where $R_{\ell,m}$ are one dimensional subspaces for $\ell_0\leq \ell < \abs{\ell_1}$. Here $\ell_1$ is some real number such that $\abs{\ell_1}\geq \ell_0$ and $\abs{\ell_1}-\ell_0$ is integral. We can choose a basis $\Set{\xi_{\ell,m}}$ as in the infinite dimensional case and by similar considerations we still get the formulae \cref{eq:infdimactions,eq:infdimds} describing the actions of $H_+$, $H_-$, $H_3$, $F_+$, $F_-$, and $F_3$, though now in this basis we only consider $\ell\in\Set{\ell_0,\ell_0+1,\dotsc,\abs{\ell_1}-1}$. 

Here it is worth noting that we can actually describe the modules accurately using some algebraic results which we will not show in this paper. Firstly note that we know that all simple finite dimensional $\slc$-modules are the $V(n)$ we have described earlier, and from this it can be shown that all simple finite dimensional $\slpr$-modules are of the form $V(n)\otimes V(m)$, where the action of $\slpr$ on $V(n)\otimes V(m)$ is given by
\begin{align*}
  (x,y).v_1\otimes v_2 = (x.v_1)\otimes v_2 + v_1\otimes (y.v_2).
\end{align*}
Note that considering $V(n)\otimes V(m)$ as $\slc$-modules via the diagonal map $\slc\to\slpr$, $u\mapsto (u,u)$, we get the standard action of $\slc$ on $V(n)\otimes V(m)$ as described in \cref{eq:tensoraction}, and now Clebsch-Gordan's formula (which we proved in the special case of $V(2)\otimes V(n)$ earlier) implies that
\begin{align*}
  V(n)\otimes V(m) \simeq \bigoplus_{i=0}^{\min(n,m)} V(n+m-2i).
\end{align*}
Note that the last term of the direct sum if $V(\abs{n-m})$. Thus translating to the language of $L$ and $L_k$-modules, we have that all simple finite dimensional $L$-modules are of the form $\bigoplus_{i=0}^{\min(2r,2s)} M(2r+2s-2i)$ where $r,s\in \tfrac{1}{2}\Z_{\geq0}$. Here we see that $\ell_0 = \abs{r-s}$ and $\abs{\ell_1}-1=r+s$. 

%%% Local Variables:
%%% mode: latex
%%% TeX-master: "../main"
%%% End:
