\section{Representations of \texorpdfstring{$L_k$}{L\_k}}

Let $V$ be a $\C$ vector space and $\rho\colon L_k \to \Lie{gl}(V)$ a representation of $L_k$. We will use the notation $\rho(a)=A$ for $a\in L_k$ switching to upper case letters when we talk about the representation corresponding to a given element. Note that we will switch freely between the language of representations of $L_k$ and the language of $L_k$-modules.

We will start out by describing the finite dimensional simple\footnote{In \cite{indecompReprOfLorGr} the word irreducible is used instead of simple, but we will only use irreducible when talking about representations in this paper.} $L_k$-modules. Recall cf.\ \cite[36]{jantzen} that we know from $\slc$-theory that for integers $n\geq 0$ there exists a unique simple $\slc$-module $V(n)_\C$ of dimension $n+1$, and $V(n)_\C$ has a basis $(v_0,v_1,\dotsc,v_n)$ such that for all $i$, $0\leq i\leq n$
\begin{align*}
  h.v_i &= (n-2i)v_i, \\
  x.v_i &=
          \begin{dcases*}
            (n-i+1)v_{i-1} & if $i>0$, \\
            0 & if $i=0$,
          \end{dcases*} \\
  y.v_i &=
          \begin{dcases*}
            (i+1)v_{i+1} & if $i<n$, \\
            0 & if $i=n$.
          \end{dcases*}
\end{align*}

Now using the isomorphism from \cref{eq:sl2iso} we see that for integers $n\geq 0$ there exists a unique simple $L_k$-module $M(n)$ of dimension $n+1$, and $M(n)$ has a basis $(v_0,v_1,\dotsc,v_n)$ such that for all $i$, $0\leq i\leq n$
\begin{align*}
  H_3v_i &= (\tfrac{1}{2}n-i)v_i, \\
  H_+v_i &=
          \begin{dcases*}
            (n-i+1)v_{i-1} & if $i>0$, \\
            0 & if $i=0$,
          \end{dcases*} \\
  H_-v_i &=
          \begin{dcases*}
            (i+1)v_{i+1} & if $i<n$, \\
            0 & if $i=n$.
          \end{dcases*}
\end{align*}
From this we build a new basis by taking
\begin{align*}
  w_i &= \dfrac{1}{\sqrt{\binom{n}{i}}}v_i,
\end{align*}
Note that
\begin{align*}
  H_3w_i &= \dfrac{1}{\sqrt{\binom{n}{i}}}H_3v_i = \dfrac{1}{\sqrt{\binom{n}{i}}}(\tfrac{1}{2}n-i)v_i = (\tfrac{1}{2}n-i)w_i
\end{align*}
for all $i$, $0\leq i\leq n$, and clearly still
\begin{align*}
  H_+w_0 &= 0, \\
  H_-w_n &= 0.
\end{align*}
But for $i$, $0<i\leq n$
\begin{align*}
  H_+w_i &= \dfrac{1}{\sqrt{\binom{n}{i}}}H_+v_i = \dfrac{1}{\sqrt{\binom{n}{i}}}(n-i+1)v_{i-1} \\
         &= \sqrt{\dfrac{\binom{n}{i-1}}{\binom{n}{i}}}(n-i+1)\dfrac{1}{\sqrt{\binom{n}{i-1}}}v_{i-1} \\
  &= \sqrt{\dfrac{i}{n-i+1}}(n-i+1)w_{i-1} = \sqrt{(n-i+1)i}w_{i-1},
\end{align*}
and for $i$, $0\leq i<n$
\begin{align*}
  H_-w_i &= \dfrac{1}{\sqrt{\binom{n}{i}}}H_-v_i = \dfrac{1}{\sqrt{\binom{n}{i}}}(i+1)v_{i+1} \\
         &= \sqrt{\dfrac{\binom{n}{i+1}}{\binom{n}{i}}}(i+1)\dfrac{1}{\sqrt{\binom{n}{i+1}}}v_{i+1} \\
  &= \sqrt{\dfrac{n-i}{i+1}}(i+1)w_{i+1} = \sqrt{(n-i)(i+1)}w_{i+1}.
\end{align*}

Finally write $\ell = \tfrac{1}{2}n$. We will re-index with $m=\tfrac{1}{2}n-i=\ell-i$ by setting
\begin{align*}
  e_m &= w_{\ell-m}
\end{align*}
for $m\in\Set{-\ell,-\ell+1,\dotsc,\ell-1,\ell}$. Thus we get
\begin{align*}
  H_3e_m &= H_3w_{\ell-m} = (\ell-(\ell-m))w_{\ell-m} = me_m,
\end{align*}
and since $e_{\ell} = w_0$ and $e_{-\ell} = w_n$ also
\begin{align*}
  H_+e_{\ell} &= 0,\\
  H_-e_{-\ell} &= 0.
\end{align*}
And for $m\in\Set{-\ell,-\ell+1,\dotsc,\ell-2,\ell-1}$ we get
\begin{align*}
  H_+e_m &= H_+w_{\ell-m} = \sqrt{(n-(\ell-m)+1)(\ell-m)}w_{\ell-m-1} \\
  &= \sqrt{(\ell+m+1)(\ell-m)}e_{m+1},
\end{align*}
while for $m\in\Set{-\ell+1,-\ell+2,\dotsc,\ell-1,\ell}$ we get
\begin{align*}
  H_-e_m &= H_-w_{\ell-m} = \sqrt{(n-(\ell-m))(\ell-m+1)}w_{\ell-m+1} \\
  &= \sqrt{(\ell+m)(\ell-m+1)}e_{m-1}.
\end{align*}

Thus we get the following Lemma:
\begin{lemma}\label{lem:modulebasis}
  Every simple finite dimensional $L_k$-module is uniquely given by a number $\ell\in\tfrac{1}{2}\Z_{\geq 0}$. For such $\ell$ the unique simple $L_k$-module $M(2\ell)$ has dimension $2\ell+1$, and $M(2\ell)$ has a basis $(e_{-\ell},e_{-\ell+1},\dotsc,e_{\ell-1},e_\ell)$ such that for all $m\in\Set{-\ell,-\ell+1,\dotsc,\ell-1,\ell}$ we have
  \begin{align}
    \begin{split}
      H_3e_m &= me_m,\label{eq:modulerels} \\
      H_+e_m &=
      \begin{dcases*}
        \sqrt{(\ell+m+1)(\ell-m)}e_{m+1} & if $m\neq \ell$, \\
        0 & if $m=\ell$,
      \end{dcases*} \\
      H_-e_m &=
      \begin{dcases*}
        \sqrt{(\ell+m)(\ell-m+1)}e_{m-1} & if $m\neq -\ell$, \\
        0 & if $m=-\ell$.
      \end{dcases*}
    \end{split}
  \end{align}
\end{lemma}

\subsection{Formulae for the operators \texorpdfstring{$H_+,H_-,H_3,F_+,F_-,F_3$}{H\_+,H\_-,H\_3,F\_+,F\_-,F\_3}}

Let $M$ be a Harish-Chandra $L$-module. Then we have linear operators $H_+,H_-,H_3,F_+,F_-,F_3\colon M\to M$ satisfying commutation relations as in \cref{eq:lierels}, and we want to give expressions for these in terms of other linear operators $E_+,E_-,D_+,D_-,D_0\colon M\to M$. 

We will denote by $R_\ell$ a finite dimensional $L$-module which is a (finite) direct sum of $L_k$-modules $M(2\ell+1)$ for the same number $\ell\in\tfrac{1}{2}\Z_{\geq 0}$. Then $M$ is a direct sum of the subspaces $R_\ell$ since $M$ is Harish-Chandra, and from \cref{lem:modulebasis} we know that $R_\ell$ can be written as the direct sum of subspaces $R_{\ell,m}$, where $R_{\ell,m}$ are eigenspaces for $H_3$ such that
\begin{align*}
  H_3\xi &= m\xi
\end{align*}
for $m\in\Set{-\ell,-\ell+1,\dotsc,\ell-1,\ell}$ and $\xi\in R_{l,m}$. We will use the decomposition
\begin{align*}
  M &= \bigoplus_{\substack{\ell\in\tfrac{1}{2}\Z_{\geq 0} \\ m\in\Set{-\ell,-\ell+1,\dotsc,\ell-1,\ell}}} \!\!\!\!\!\!\!\!\!\!\!\! R_{\ell,m} = \bigoplus_{\ell,m} R_{\ell,m}
\end{align*}
throughout this paper.

By \cref{lem:modulebasis} we also have that $H_+$ and $H_-$ maps the $R_{\ell,m}$ into each other as follows:
\begin{align*}
  H_+\colon R_{\ell,m} &\to R_{\ell,m+1} && \mbox{if }-\ell\leq m<\ell, & H_+&\colon R_{\ell,\ell} \to 0, \\
  H_-\colon R_{\ell,m} &\to R_{\ell,m-1} && \mbox{if }-\ell< m\leq \ell, & H_-&\colon R_{\ell,-\ell} \to 0.
\end{align*}
Hence we have linear operators $H_+H_-,H_-H_+\colon R_{\ell,m}\to R_{\ell,m}$, and by \cref{eq:modulerels} we see that
\begin{align}
  \begin{split} \label{eq:H+H-}
    H_+H_-\xi &= \sqrt{(\ell+(m-1)+1)(\ell-(m-1))}\sqrt{(\ell+m)(\ell-m+1)} \xi \\
    &= (\ell+m)(\ell-m+1)\xi, \\
    H_-H_+\xi &= \sqrt{(\ell+(m+1))(\ell-(m+1)+1)}\sqrt{(\ell+m+1)(\ell-m)} \xi \\
    &= (\ell+m+1)(\ell-m)\xi.
  \end{split}
\end{align}
Note that this also covers the cases $m=\ell$ and $m=-\ell$. 

Now we define $E_+ \colon R_{\ell,m}\to R_{\ell,m+1}$ and $E_- \colon R_{\ell,m}\to R_{\ell,m-1}$ to be the linear maps satisfying
\begin{align}
  \begin{split} \label{eq:E+andE-}
    H_+ \xi &=
    \begin{dcases*}
      \sqrt{(\ell+m+1)(\ell-m)}E_+\xi & if $m\neq \ell$ \\
      0 & if $m=\ell$,
    \end{dcases*} \\
    H_- \xi &=
    \begin{dcases*}
      \sqrt{(\ell+m)(\ell-m+1)}E_-\xi & if $m\neq -\ell$ \\
      0 & if $m=\ell$
    \end{dcases*}
  \end{split}
\end{align}
for $\xi\in R_{\ell,m}$. Comparing \cref{eq:E+andE-} and \cref{eq:H+H-} we see that
\begin{align*}
  E_+E_-\xi &= \xi && \mbox{if }m\neq-\ell \\
  E_-E_+\xi &= \xi && \mbox{if }m\neq\ell.
\end{align*}
Thus $E_+\colon R_{\ell,m}\to R_{\ell,m+1}$ and $E_-\colon R_{\ell,m+1}\to R_{\ell,m}$ are isomorphisms for $m\neq\ell$ and they are each others inverse. 



%%% Local Variables:
%%% mode: latex
%%% TeX-master: "../main"
%%% End:
