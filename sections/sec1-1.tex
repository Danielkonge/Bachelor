\section{Representations of \texorpdfstring{$L_k$}{L\_k}}

Let $V$ be a $\C$ vector space and $\rho\colon L_k \to \Lie{gl}(V)$ a representation of $L_k$. We will use the notation $\rho(a)=A$ for $a\in L_k$ switching to upper case letters when we talk about the representation corresponding to a given element. Note that we will switch freely between the language of representations of $L_k$ and the language of $L_k$-modules.

We will start out by describing the finite dimensional simple\footnote{In \cite{indecompReprOfLorGr} the word irreducible is used instead of simple, but we will only use irreducible when talking about representations in this paper.} $L_k$-modules. Recall cf.\ \cite[36]{jantzen} that we know from $\slc$-theory that for integers $n\geq 0$ there exists a unique simple $\slc$-module $V(n)$ of dimension $n+1$, and $V(n)$ has a basis $(v_0,v_1,\dotsc,v_n)$ such that for all $i$, $0\leq i\leq n$
\begin{align}
  \begin{split}\label{eq:sl2modbasis}
    h.v_i &= (n-2i)v_i, \\
    x.v_i &=
    \begin{dcases*}
      (n-i+1)v_{i-1} & if $i>0$, \\
      0 & if $i=0$,
    \end{dcases*} \\
    y.v_i &=
    \begin{dcases*}
      (i+1)v_{i+1} & if $i<n$, \\
      0 & if $i=n$.
    \end{dcases*}
  \end{split}
\end{align}

Now using the isomorphism from \cref{eq:sl2iso} we see that for integers $n\geq 0$ there exists a unique simple $L_k$-module $M(n)$ of dimension $n+1$, and $M(n)$ has a basis $(v_0,v_1,\dotsc,v_n)$ such that for all $i$, $0\leq i\leq n$
\begin{align*}
  H_3v_i &= (\tfrac{1}{2}n-i)v_i, \\
  H_+v_i &=
          \begin{dcases*}
            (n-i+1)v_{i-1} & if $i>0$, \\
            0 & if $i=0$,
          \end{dcases*} \\
  H_-v_i &=
          \begin{dcases*}
            (i+1)v_{i+1} & if $i<n$, \\
            0 & if $i=n$.
          \end{dcases*}
\end{align*}
From this we build a new basis by taking
\begin{align*}
  w_i &= \dfrac{1}{\sqrt{\binom{n}{i}}}v_i,
\end{align*}
Note that
\begin{align*}
  H_3w_i &= \dfrac{1}{\sqrt{\binom{n}{i}}}H_3v_i = \dfrac{1}{\sqrt{\binom{n}{i}}}(\tfrac{1}{2}n-i)v_i = (\tfrac{1}{2}n-i)w_i
\end{align*}
for all $i$, $0\leq i\leq n$, and clearly still
\begin{align*}
  H_+w_0 &= 0, \\
  H_-w_n &= 0.
\end{align*}
But for $i$, $0<i\leq n$
\begin{align*}
  H_+w_i &= \dfrac{1}{\sqrt{\binom{n}{i}}}H_+v_i = \dfrac{1}{\sqrt{\binom{n}{i}}}(n-i+1)v_{i-1} \\
         &= \sqrt{\dfrac{\binom{n}{i-1}}{\binom{n}{i}}}(n-i+1)\dfrac{1}{\sqrt{\binom{n}{i-1}}}v_{i-1} \\
  &= \sqrt{\dfrac{i}{n-i+1}}(n-i+1)w_{i-1} = \sqrt{(n-i+1)i}w_{i-1},
\end{align*}
and for $i$, $0\leq i<n$
\begin{align*}
  H_-w_i &= \dfrac{1}{\sqrt{\binom{n}{i}}}H_-v_i = \dfrac{1}{\sqrt{\binom{n}{i}}}(i+1)v_{i+1} \\
         &= \sqrt{\dfrac{\binom{n}{i+1}}{\binom{n}{i}}}(i+1)\dfrac{1}{\sqrt{\binom{n}{i+1}}}v_{i+1} \\
  &= \sqrt{\dfrac{n-i}{i+1}}(i+1)w_{i+1} = \sqrt{(n-i)(i+1)}w_{i+1}.
\end{align*}

Finally write $\ell = \tfrac{1}{2}n$. We will re-index with $m=\tfrac{1}{2}n-i=\ell-i$ by setting
\begin{align*}
  e_m &= w_{\ell-m}
\end{align*}
for $m\in\Set{-\ell,-\ell+1,\dotsc,\ell-1,\ell}$. Thus we get
\begin{align*}
  H_3e_m &= H_3w_{\ell-m} = (\ell-(\ell-m))w_{\ell-m} = me_m,
\end{align*}
and since $e_{\ell} = w_0$ and $e_{-\ell} = w_n$ also
\begin{align*}
  H_+e_{\ell} &= 0,\\
  H_-e_{-\ell} &= 0.
\end{align*}
And for $m\in\Set{-\ell,-\ell+1,\dotsc,\ell-2,\ell-1}$ we get
\begin{align*}
  H_+e_m &= H_+w_{\ell-m} = \sqrt{(n-(\ell-m)+1)(\ell-m)}w_{\ell-m-1} \\
  &= \sqrt{(\ell+m+1)(\ell-m)}e_{m+1},
\end{align*}
while for $m\in\Set{-\ell+1,-\ell+2,\dotsc,\ell-1,\ell}$ we get
\begin{align*}
  H_-e_m &= H_-w_{\ell-m} = \sqrt{(n-(\ell-m))(\ell-m+1)}w_{\ell-m+1} \\
  &= \sqrt{(\ell+m)(\ell-m+1)}e_{m-1}.
\end{align*}

Thus we get the following Lemma:
\begin{lemma}\label{lem:modulebasis}
  Every simple finite dimensional $L_k$-module is uniquely given by a number $\ell\in\tfrac{1}{2}\Z_{\geq 0}$. For such $\ell$ the unique simple $L_k$-module $M(2\ell)$ has dimension $2\ell+1$, and $M(2\ell)$ has a basis $(e_{-\ell},e_{-\ell+1},\dotsc,e_{\ell-1},e_\ell)$ such that for all $m\in\Set{-\ell,-\ell+1,\dotsc,\ell-1,\ell}$ we have
  \begin{align}
    \begin{split}
      H_3e_m &= me_m,\label{eq:modulerels} \\
      H_+e_m &=
      \begin{dcases*}
        \sqrt{(\ell+m+1)(\ell-m)}e_{m+1} & if $m\neq \ell$, \\
        0 & if $m=\ell$,
      \end{dcases*} \\
      H_-e_m &=
      \begin{dcases*}
        \sqrt{(\ell+m)(\ell-m+1)}e_{m-1} & if $m\neq -\ell$, \\
        0 & if $m=-\ell$.
      \end{dcases*}
    \end{split}
  \end{align}
\end{lemma}

\subsection{Formulae for the operators \texorpdfstring{$H_+,H_-,H_3,F_+,F_-,F_3$}{H\_+,H\_-,H\_3,F\_+,F\_-,F\_3}}

Let $M$ be a Harish-Chandra $L$-module. Then we have linear operators $H_+,H_-,H_3,F_+,F_-,F_3\colon M\to M$ satisfying commutation relations as in \cref{eq:lierels}, and we want to give expressions for these in terms of other linear operators $E_+,E_-,D_+,D_-,D_0\colon M\to M$. 

We will denote by $R_\ell$ a finite dimensional $L$-module which is a (finite) direct sum of $L_k$-modules $M(2\ell+1)$ for the same number $\ell\in\tfrac{1}{2}\Z_{\geq 0}$. Then $M$ is a direct sum of the subspaces $R_\ell$ since $M$ is Harish-Chandra, and from \cref{lem:modulebasis} we know that $R_\ell$ can be written as the direct sum of subspaces $R_{\ell,m}$, where $R_{\ell,m}$ are eigenspaces for $H_3$ such that
\begin{align} \label{eq:H3eigen}
  H_3\xi &= m\xi
\end{align}
for $m\in\Set{-\ell,-\ell+1,\dotsc,\ell-1,\ell}$ and $\xi\in R_{l,m}$. We will use the decomposition
\begin{align*}
  M &= \bigoplus_{\substack{\ell\in\tfrac{1}{2}\Z_{\geq 0} \\ m\in\Set{-\ell,-\ell+1,\dotsc,\ell-1,\ell}}} \!\!\!\!\!\!\!\!\!\!\!\! R_{\ell,m} = \bigoplus_{\ell,m} R_{\ell,m}
\end{align*}
throughout this paper.

By \cref{lem:modulebasis} we also have that $H_+$ and $H_-$ maps the $R_{\ell,m}$ into each other as follows:
\begin{align*}
  H_+\colon R_{\ell,m} &\to R_{\ell,m+1} && \mbox{if }-\ell\leq m<\ell, & H_+&\colon R_{\ell,\ell} \to 0, \\
  H_-\colon R_{\ell,m} &\to R_{\ell,m-1} && \mbox{if }-\ell< m\leq \ell, & H_-&\colon R_{\ell,-\ell} \to 0.
\end{align*}
Hence we have linear operators $H_+H_-,H_-H_+\colon R_{\ell,m}\to R_{\ell,m}$, and by \cref{eq:modulerels} we see that
\begin{align}
  \begin{split} \label{eq:H+H-}
    H_+H_-\xi &= \sqrt{(\ell+(m-1)+1)(\ell-(m-1))}\sqrt{(\ell+m)(\ell-m+1)} \xi \\
    &= (\ell+m)(\ell-m+1)\xi, \\
    H_-H_+\xi &= \sqrt{(\ell+(m+1))(\ell-(m+1)+1)}\sqrt{(\ell+m+1)(\ell-m)} \xi \\
    &= (\ell+m+1)(\ell-m)\xi.
  \end{split}
\end{align}
Note that this also covers the cases $m=\ell$ and $m=-\ell$. 

Now we define $E_+ \colon R_{\ell,m}\to R_{\ell,m+1}$ and $E_- \colon R_{\ell,m}\to R_{\ell,m-1}$ to be the linear maps satisfying
\begin{align}
  \begin{split} \label{eq:E+andE-}
    H_+ \xi &=
    \begin{dcases*}
      \sqrt{(\ell+m+1)(\ell-m)}E_+\xi & if $m\neq \ell$ \\
      0 & if $m=\ell$,
    \end{dcases*} \\
    H_- \xi &=
    \begin{dcases*}
      \sqrt{(\ell+m)(\ell-m+1)}E_-\xi & if $m\neq -\ell$ \\
      0 & if $m=\ell$
    \end{dcases*}
  \end{split}
\end{align}
for $\xi\in R_{\ell,m}$. Comparing \cref{eq:E+andE-} and \cref{eq:H+H-} we see that
\begin{align*}
  E_+E_-\xi &= \xi && \mbox{if }m\neq-\ell \\
  E_-E_+\xi &= \xi && \mbox{if }m\neq\ell.
\end{align*}
Thus $E_+\colon R_{\ell,m}\to R_{\ell,m+1}$ and $E_-\colon R_{\ell,m+1}\to R_{\ell,m}$ are isomorphisms for $m\neq\ell$ and they are each others inverse. 

Now note that $H_+$, $H_-$, and $H_3$ are completely determined by \cref{eq:H3eigen} and \cref{eq:E+andE-}, so we just need to find maps to determine $F_+$, $F_-$, and $F_3$ now, while making sure that we get commutation relations as in \cref{eq:lierels}.

Consider maps $D_0$ and $D_+$ defined on $M=\bigoplus_{\ell,m} R_{\ell,m}$ and $D_-$ defined on the direct sum without the summands $R_{\ell,\ell}$ and $R_{\ell,-\ell}$ such that $D_0 R_{\ell,m}\subset R_{\ell,m}$, $D_+ R_{\ell,m} \subset R_{\ell+1,m}$, and $D_- R_{\ell,m} \subset R_{\ell-1,m}$ and the diagrams
\begin{center}
  \begin{tikzcd}
    R_{\ell-1,m+1} & R_{\ell,m+1} \ar[l,"D_-",swap] & R_{\ell,m+1} \ar[r,"D_0"] & R_{\ell,m+1}  \\
    R_{\ell-1,m} \ar[u,"E_+"] & R_{\ell,m} \ar[l,"D_-"] \ar[u,"E_+",swap] & R_{\ell,m} \ar[r,"D_0",swap] \ar[u,"E_+"] & R_{\ell,m} \ar[u,"E_+",swap] \\
    & R_{\ell,m+1} \ar[r,"D_+"] & R_{\ell+1,m+1} \\
    & R_{\ell,m} \ar[u,"E_+"] \ar[r,"D_+",swap] & R_{\ell+1,m+1} \ar[u,"E_+",swap]
  \end{tikzcd}
\end{center}
commute, when $-\ell+1\leq m < \ell-1$ in the top left diagram, $-\ell\leq m<\ell$ in the other two diagrams. We need quite a lot of work to find a way to describe $F_+$, $F_-$, and $F_3$ from these maps.

We already have that $L_k=\Span_\C(h_+,h_-,h_3)$, but now we will also consider $L_p=\Span_\C(f_+,f_-,f_3)$. \Cref{eq:lierels} gives us that $[L_k,L_p]\subset L_p$, so by the adjoint representation we can see $L_p$ as an $L_k$-module, and again by \cref{eq:lierels} we see that $L_p$ is a simple $L_k$-module: If $V$ is an $L_k$-submodule and we have a non-zero element $f=af_++bf_-+cf_3\in V$ for some $a,b,c\in\C$ not all zero. Then
\begin{align*}
  [h_+,af_++bf_-+cf_3] &= 2bf_3-cf_+, \\
  [h_-,af_++bf_-+cf_3] &= -2af_3+cf_-, \\
  [h_3,af_++bf_-+cf_3] &= af_+-bf_-.
\end{align*}
If $c\neq0$, we get that
\begin{align*}
  [h_3,[h_+,f]] &= [h_3,2bf_3-cf_+] = -cf_+, \\
  [h_3,[h_-,f]] &= [h_3,-2af_3+cf_-] = -cf_-,
\end{align*}
so we see that $f_+,f_-\in V$, and thus also $[h_+,\tfrac{1}{2}f_-]=f_3\in V$, so $V=L_p$. If on the other hand $c=0$, then
\begin{align*}
  [h_-,f] &= -2af_3, \\
  [h_+,f] &= 2bf_3,
\end{align*}
so since either $a\neq 0$ or $b\neq 0$, we see that $f_3\in V$, and thus also $[h_+,-f_3]=f_+\in V$ and $[h_-,f_3]=f_-\in V$, so $V=L_p$. Hence $L_p$ is indeed a simple $L_k$-module. Now since $L_p$ is a simple finite dimensional $L_k$-module of dimension 3, we have that $L_p\simeq M(2)$ as $L_k$-modules.

In general given two $L$-modules $V$ and $W$, we consider the tensor product $V\otimes W$ over $\C$ of the underlying vector spaces as an $L$-module via the action
\begin{align*}
  x.(v\otimes w) &= x.v\otimes w + v\otimes x.w,
\end{align*}
cf.\ \cite[26]{humphrey}.

Now we are interested in the $L_k$-module $L_p\otimes M$, where $M$ is a Harish-Chandra $L$-module as before. Specifically we will show that
\begin{align}
  \begin{split} \label{eq:psi}
    \psi \colon L_p\otimes M &\to M \\
    x\otimes v &\mapsto x.v
  \end{split}
\end{align}
is a homomorphism of $L_k$-modules. It is clear that $\psi$ is linear, and for $y\in L_k$ we see that
\begin{align*}
  y.(x\otimes v) &= y.x\otimes v + x\otimes y.v = [y,x]\otimes v + x\otimes y.v,
\end{align*}
for $x\otimes v\in L_p\otimes M$, since the action in $L_p$ is by the adjoint representation. So
\begin{align*}
  \psi(y.(x\otimes v)) &= \psi([y,x]\otimes v) + \psi(x\otimes y.v) = [y,x].v + x.(y.v) \\
  &= y.(x.v)-x.(y.v)+x.(y.v) = y.(x.v) = y.\psi(x\otimes v),
\end{align*}
i.e.\ $\psi$ is indeed a homomorphism of $L_k$-modules.

Now we note that $M=\bigoplus_{\ell} R_\ell$, so 
\begin{align*}
  L_p\otimes M &= L_p\otimes \bigl( \bigoplus_{\ell} R_\ell \bigr) \simeq \bigoplus_{\ell} (L_p\otimes R_\ell),
\end{align*}
as $L_k$-modules, and since also $R_\ell$ is direct sum of finitely many copies of $M(2\ell)$, we see that
\begin{align*}
  L_p\otimes R_\ell &\simeq M(2)\otimes \bigl( M(2\ell)^1 \oplus M(2\ell)^2 \oplus \dotsb \oplus M(2\ell)^r \bigr) \\
  &\simeq (M(2)\otimes M(2\ell)^1) \oplus (M(2)\otimes M(2\ell)^2) \oplus \dotsb \oplus (M(2)\otimes M(2\ell)^r),
\end{align*}
as $L_k$-modules, since $L_p\simeq M(2)$. Here the superscripts are just indices for the different $M(2\ell)$. Thus we want to describe the $L_k$-modules $M(2)\otimes M(2\ell)$, which we will do by first describing the $\slc$-modules $V(2)\otimes V(2\ell)$ and then translating back to a solution to our problem by the isomorphism of \cref{eq:sl2iso}. 

\subsection{\texorpdfstring{Describing $V(2)\otimes V(n)$}{Describing V(2) tensor product V(n)}}
Let $2\ell=n\in\N$. We want to show that
\begin{align}\label{eq:V(2)tensorV(n)}
  V(2)\otimes V(n) &\simeq
                     \begin{dcases*}
                       V(n-2) \oplus V(n) \oplus V(n+2) & if $n\geq 2$,\\
                       V(3)\oplus V(1) & if $n=1$, \\
                       V(2) & if $n=0$.
                     \end{dcases*}
\end{align}
Note that in all cases there is a summand $V(n+2)$. We can show the above by considerations using formal characters. We will use the notation of \cite[Chapter~8]{jantzen}, specifically we will do calculations with the functions $e(\lambda)\colon H^*\to \Z$ for $\lambda\in H^*$. Firstly note that in general
\begin{align*}
  \ch_V &= \sum_{\lambda\in H^*} (\dim V_\lambda) e(\lambda),
\end{align*}
and use the notation $V(n)_k$ for $V(\lambda)_\mu$ and $e(n)$ for $e(\lambda)$ with $\lambda,\mu\in H^*$ such that $\lambda(h)=n$ and $\mu(h)=k$. We get that
\begin{align*}
  \ch_{V(2)} &= e(-2) + e(0) + e(2)
\end{align*}
and 
\begin{align*}
  \ch_{V(n)} &= \sum_{i=0}^n e(n-2i),
\end{align*}
since
\begin{align*}
  \dim V(n)_k &=
                \begin{dcases*}
                  1 & if $k=n-2i$ for some $i\in\Set{0,1,\dotsc,n}$, \\
                  0 & otherwise.
                \end{dcases*}
\end{align*}
Now since $e(\lambda)*e(\mu)=e(\lambda+\mu)$ in general cf.\ \cite[93]{jantzen}, we see that for $n\geq 2$
\begin{align*}
  \ch_{V(2)\otimes V(n)} &= \ch_{V(2)} * \ch_{V(n)} = e(-2)*\ch_{V(n)} + e(0)*\ch_{V(n)} + e(2)*\ch_{V(n)} \\
                         &= \sum_{i=0}^n e(n-2-2i) + \ch_{V(n)} + \sum_{i=0}^n e(n+2-2i) \\
                         &= e(-n-2)+e(-n) + \sum_{i=0}^{n-2} e(n-2-2i) + \ch_{V(n)} \\*
                         &\phantom{{}={}}+ \sum_{i=0}^n e(n+2-2i) \\
                         &= \ch_{V(n-2)} + \ch_{V(n)} + \sum_{i=0}^{n+2} e(n+2-2i) \\
  &= \ch_{V(n-2)} + \ch_{V(n)} + \ch_{V(n+2)} = \ch_{V(n-2)\oplus V(n)\oplus V(n+2)},
\end{align*}
where the first equality follows from the fact that $\ch_{V\otimes W}=\ch_V * \ch_W$ in general, cf.\ \cite[125]{humphrey}. Thus since two $L$-modules $V$ and $V'$ are isomorphic if and only if $\ch_V=\ch_{V'}$, cf.\ \cite[90]{jantzen}, we see that $V(2)\otimes V(n) \simeq V(n-2)\oplus V(n)\oplus V(n+2)$ if $n\geq 2$.

Likewise we see that
\begin{align*}
  \ch_{V(2)\otimes V(1)} &= \ch_{V(2)}*\ch_{V(1)} \\
  &= \bigl(e(-2)+e(0)+e(2)\bigr)*e(-1) + \bigl(e(-2)+e(0)+e(2)\bigr)*e(1) \\
                         &= e(-3)+e(-1)+e(1)+e(-1)+e(1)+e(3) \\
                         &= \bigl(e(-3)+e(-1)+e(1)+e(3)\bigr) + \bigl(e(-1)+e(1)\bigr) \\
  &= \ch_{V(3)} + \ch_{V(1)} = \ch_{V(3)\oplus V(1)}
\end{align*}
and
\begin{align*}
  \ch_{V(2)\otimes V(0)} &= \ch_{V(2)}*\ch_{V(0)} = \ch_{V(2)}*e(0) = \ch_{V(2)},
\end{align*}
so indeed $V(2)\otimes V(1)\simeq V(3)\oplus V(1)$ and $V(2)\otimes V(0)\simeq V(2)$. 

Now consider $(w_0,w_1,w_2)$ a basis for $V(2)$ and $(v_i \mid 0\leq i\leq n)$ a basis for $V(n)$ such that both satisfies the conditions from \cref{eq:sl2modbasis}. Then for $w_i\otimes v_j \in V(2)\otimes V(n)$ with $i\in\Set{0,1,2}$ and $j\in\Set{0,1,\dotsc,n}$ we see that
\begin{align}
  h.(w_i\otimes v_j) &= h.w_i\otimes v_j + w_i\otimes h.v_j = (2-2i)w_i\otimes v_j + (n-2j)w_i\otimes v_j \nonumber \\
  &= (n-2(i+j-1))w_i\otimes v_j.\label{eq:hontensor}
\end{align}
Hence $v_0\otimes w_0$ is up to scalar multiple the only vector of weight $n+2$ in $V(2)\otimes V(n)$, so it is necessarily a highest weight vector generating the direct summand isomorphic to $V(n+2)$. Note that by \cref{eq:V(2)tensorV(n)} we indeed have a direct summand isomorphic to $V(n+2)$ for all $n\in\N$. By $\slc$-theory, cf.\ \cite[36]{jantzen}, we know that this summand has a basis $(s_k\mid 0\leq k\leq n+2)$ satisfying equations as in \cref{eq:sl2modbasis}, where
\begin{align}\label{eq:s_kbasis}
  s_k &\coloneqq \dfrac{1}{k!} y^k.(w_0\otimes v_0).
\end{align}
By straightforward calculations, cf.\ \Cref{sec:basesofV(2)tensorV(n)}, we get for $n>0$ that 
\begin{align}\label{eq:s_kbasisres}
  \begin{aligned}
    s_0 &= w_0\otimes v_0, \\
    s_1 &= w_1\otimes v_0 + w_0\otimes v_1 \\
    s_k &= w_2\otimes v_{k-2} + w_1\otimes v_{k-1} + w_0\otimes v_k \\
    s_{n+1} &= w_2\otimes v_{n-1} + w_1\otimes v_n \\
    s_{n+2} &= w_2\otimes v_n.
  \end{aligned} &&
                   \begin{aligned}
                     &\phantom{a}\\ &\mbox{if }n>0, \\ &\mbox{for }2\leq k\leq n, \\ &\mbox{if }n>0,\\ &\phantom{a}
                   \end{aligned}
\end{align}

In case $n=0$ we likewise see that $s_1=w_1\otimes v_0$ and $s_2=w_2\otimes v_0$, and we note that $(s_0,s_1,s_2)$ is a basis for $V(2)\otimes V(0)\simeq V(2)$.

Suppose now that $n\geq 1$. Note that by \cref{eq:V(2)tensorV(n)} we have a direct summand isomorphic to $V(n)$, and by \cref{eq:hontensor} the weight space of weight $n$ is spanned by $w_0\otimes v_1$ and $w_1\otimes v_0$, so the vector of highest weight $n$ generating the summand corresponding to $V(n)$ must be of the form $aw_0\otimes v_1 + bw_1\otimes v_0$ for some $a,b\in \C$. Furthermore we know that for this vector generating the summand corresponding to $V(n)$, we must have that
\begin{align*}
  0 &= x.(aw_0\otimes v_1 + bw_1\otimes v_0) \\
    &= ax.w_0 \otimes v_1 + aw_0\otimes x.v_1 + bx.w_1\otimes v_0 + bw_1\otimes x.v_0 \\
    &= 0 + a(n-1+1)w_0\otimes v_0 + b(2-1+1)w_0\otimes v_0 + 0 \\
  &= (an+2b)w_0\otimes v_0, 
\end{align*}
i.e. $an+2b=0$ so $b = -\tfrac{n}{2}a$. This determines the vector generating the summand corresponding to $V(n)$ up to a scalar, so taking $a=1$, we see that we can take
\begin{align*}
  t_0 &\coloneqq w_0\otimes v_1 - \dfrac{n}{2}w_1\otimes v_0
\end{align*}
as our vector generating the summand corresponding to $V(n)$. As before $\slc$-theory now yields that this summand has a basis $(t_k \mid 0\leq k\leq n)$ satisfying equations as in \cref{eq:sl2modbasis}, where
\begin{align}\label{eq:t_kbasisdef}
  t_k \coloneqq \dfrac{1}{k!}y^k.t_0.
\end{align}
By straightforward calculations, cf.\ \cref{sec:basesofV(2)tensorV(n)}, we get that
\begin{align}\label{eq:t_kbasisres}
  \begin{aligned}
    t_0 &= w_0\otimes v_1 - \dfrac{n}{2}w_1\otimes v_0, \\
    t_k &= (k+1)w_0\otimes v_{k+1} - \dfrac{n-2k}{2}w_1\otimes v_k \\
    &\phantom{{}={}}+ (k-1-n)w_2\otimes v_{k-1} \qquad\qquad\mbox{for }1\leq k\leq n-1,\\
    t_n &= \dfrac{n}{2}w_1\otimes v_n - w_2\otimes v_{n-1}.
  \end{aligned}
\end{align}

Suppose now that $n\geq 2$. By \cref{eq:V(2)tensorV(n)} we have a direct summand isomorphic to $V(n-2)$, and by \cref{eq:hontensor} the weight space of weight $n-2$ is spanned by $w_0\otimes v_2$, $w_1\otimes v_1$, and $w_2\otimes v_0$, so the vector of highest weight $n-2$ generating the summand corresponding to $V(n)$ must be of the form $aw_0\otimes v_2 + bw_1\otimes v_1 + cw_2\otimes v_0$ for some $a,b,c\in\C$. Furthermore we know that for this vector generating the summand corresponding to $V(n-2)$, we must have
\begin{align*}
  0 &= x.(aw_0\otimes v_2 + bw_1\otimes v_1 + cw_2\otimes v_0) \\
    &= aw_0\otimes x.v_2 + bx.w_1\otimes v_1 + bw_1\otimes x.v_1 + cx.w_2\otimes v_0 \\
    &= a(n-2+1)w_0\otimes v_1 + b(2-1+1)w_0\otimes v_1 + b(n-1+1)w_1\otimes v_0 \\
  &\phantom{{}={}}{} + c(2-2+1)w_1\otimes v_0 \\
    &= \bigl((n-1)a + 2b\bigr)w_0\otimes v_1 + \bigl(bn + c\bigr)w_1\otimes v_0,
\end{align*}
i.e. $a(n-1)+2b=0$ and $bn+c=0$. Giving us $c=-bn$ and $b=-\tfrac{n-1}{2}a$, so
\begin{align*}
  c=\dfrac{n(n-1)}{2}a.
\end{align*}
This determines the vector generating the summand corresponding to $V(n-2)$ up to a scalar, so taking $a=1$, we see that we can take
\begin{align*}
  u_0 \coloneqq w_0\otimes v_2 - \dfrac{n-1}{2}w_1\otimes v_1 + \dfrac{n(n-1)}{2}w_2\otimes v_0
\end{align*}
as our vector generating the summand corresponding to $V(n-2)$. Again $\slc$-theory now yields that this summand has a basis $(u_k \mid 0\leq k\leq n-2)$ satisfying equations as in \cref{eq:sl2modbasis}, where 
\begin{align}\label{eq:u_kbasisdef}
  u_k \coloneqq \dfrac{1}{k!}y^k.u_0.
\end{align}
By straightforward calculations, cf.\ \cref{sec:basesofV(2)tensorV(n)}, we get that
\begin{align}\label{eq:u_kbasisres}
  \begin{aligned}
    u_k &= \dfrac{(k+1)(k+2)}{2}w_0\otimes v_{k+2} - \dfrac{(k+1)(n-k-1)}{2}w_1\otimes v_{k+1} \\
    &\phantom{{}={}}{} + \dfrac{(n-k)(n-k-1)}{2}w_2\otimes v_k
  \end{aligned}
\end{align}
for $0\leq k\leq n-2$. 

Now we want to express $w_1\otimes v_k$ for $0\leq k\leq n$ in terms of the bases $(s_k \mid 0\leq k\leq n+2)$, $(t_k \mid 0\leq k\leq n)$, and $(u_k \mid 0\leq k\leq n-2)$. A straightforward but long calculation, cf.\ \Cref{sec:w_1tensorv_k}, yields that
\begin{align}\label{eq:w_1tensorv_k}
  w_1\otimes v_k = \dfrac{2(k+1)(n+1-k)}{(n+1)(n+2)}s_{k+1} - \dfrac{2(n-2k)}{n(n+2)}t_k - \dfrac{4}{n(n+1)}u_{k-1}
\end{align}
for $0<k<n$, while
\begin{align}\label{eq:special_cases_w_1tensorv_k}
  w_1\otimes v_0 = \dfrac{2}{n+2}(s_1-t_0) \qquad \mbox{and} \qquad w_1\otimes v_n = \dfrac{2}{n+2}(s_{n+1}+t_n)
\end{align}
if $n\geq 1$. If $n=0$ we have already seen (just after \cref{eq:s_kbasisres}) that $w_1\otimes v_0 = s_1$.

%%% Local Variables:
%%% mode: latex
%%% TeX-master: "../main"
%%% End:
