\section{\texorpdfstring{$F_3,F_+,F_-$ in terms of $E_+,E_-,D_0,D_+,D_-$}{F\_3,F\_+,F\_- in terms of E\_+,E\_-,D\_0,D\_+,D\_-}}\label{sec:Factions}

We have already seen that
\begin{align*}
  F_3 \xi = \sqrt{\ell^2-m^2} D_- \xi - m D_0 \xi - \sqrt{(\ell+1)^2-m^2} D_+ \xi
\end{align*}
for $\xi\in R_{\ell,m}$ by using \cref{eq:F_3action} and the definition of how we expanded $D_0$, $D_+$, and $D_-$ to maps on all of $M$. Now we get by \cref{eq:lierels,eq:E+andE-} and the commutative diagrams in \cref{eq:Ddiagrams} that
\begin{align*}
  F_+ \xi &= [F_3,H_+] \xi = F_3H_+ \xi - H_+F_3 \xi \\
          &= \sqrt{(\ell+m+1)(\ell-m)}F_3E_+ \xi - \sqrt{\ell^2-m^2} H_+D_- \xi + m H_+D_0 \xi \\*
          &\phantom{{}={}}{} + \sqrt{(\ell+1)^2-m^2} H_+D_+ \xi \\
          &= \sqrt{(\ell+m+1)(\ell-m)}\Bigl( \sqrt{\ell^2-(m+1)^2} D_-E_+ \xi - (m+1) D_0E_+ \xi \\*
          &\phantom{{}={}\sqrt{(\ell+m+1)(\ell-m)}()}{} - \sqrt{(\ell+1)^2-(m+1)^2} D_+E_+ \xi \Bigr) \\*
          &\phantom{{}={}}{} - \sqrt{\ell^2-m^2} \sqrt{((\ell-1)+m+1)((\ell-1)-m)} E_+D_- \xi \\*
          &\phantom{{}={}}{} + m \sqrt{(\ell+m+1)(\ell-m)}E_+D_0 \xi \\*
          &\phantom{{}={}}{} + \sqrt{(\ell+1)^2-m^2}\sqrt{((\ell+1)+m+1)((\ell+1)-m)} E_+D_+ \xi \\
          &= \sqrt{(\ell+m+1)(\ell-m)}\Bigl( \sqrt{\ell^2-(m+1)^2} D_-E_+ \xi - (m+1) D_0E_+ \xi \\*
          &\phantom{{}={}\sqrt{(\ell+m+1)(\ell-m)}()}{} - \sqrt{(\ell+1)^2-(m+1)^2} D_+E_+ \xi \Bigr) \\*
          &\phantom{{}={}}{} - \sqrt{\ell^2-m^2} \sqrt{(\ell+m)(\ell-m-1)} D_-E_+ \xi \\*
          &\phantom{{}={}}{} + m \sqrt{(\ell+m+1)(\ell-m)}D_0E_+ \xi \\*
          &\phantom{{}={}}{} + \sqrt{(\ell+1)^2-m^2}\sqrt{(\ell+m+2)(\ell-m+1)} D_+E_+ \xi \\
          &= \Bigl( \sqrt{(\ell+m+1)(\ell-m)(\ell^2-(m+1)^2)} \\*
          &\phantom{{}={}()}{} - \sqrt{(l^2-m^2)(l+m)(l-m-1)} \Bigr)D_-E_+ \xi \\*
          &\phantom{{}={}}{} - \sqrt{(\ell+m+1)(\ell-m)}D_0E_+\xi \\*
          &\phantom{{}={}}{} + \Bigl( \sqrt{((\ell+1)^2-m^2)(\ell+m+2)(\ell-m+1)} \\*
          &\phantom{{}={}()}{} - \sqrt{(\ell+m+1)(\ell-m)((\ell+1)^2-(m+1)^2)}\Bigr)D_+E_+\xi
\end{align*}
for $\xi\in R_{\ell,m}$ and $-\ell+1 \leq m < \ell-1$. In the case where $m=-\ell$ the only problem is at the term with $E_+D_-$, but this is not a problem because the term vanishes since there is $\ell+m$ as part of the coefficient, so the formula also holds true in this case. In case $m=\ell-1$ the only problem is at the term with $D_-E_+$, but here we have $\ell^2-(m+1)^2$ as part of the coefficient, so this term also vanishes, and the formula also hold true in this case. Finally in case $m=\ell$ the terms with $D_-E_+$, $D_0E_+$, $D_+E_+$, $E_+D_-$, and $E_+D_0$ all cause problems, but again all of these terms vanish, so the formula still holds true in this case. Now by noting that $\ell^2-m^2=(\ell+m)(\ell-m)$ and $\ell^2-(m+1)^2=(\ell+m+1)(\ell-m-1)$, we see that
\begin{align*}
  &\sqrt{(\ell+m+1)(\ell-m)(\ell^2-(m+1)^2)} - \sqrt{(l^2-m^2)(l+m)(l-m-1)} \\
  &=\sqrt{(\ell+m+1)(\ell-m)(\ell+m+1)(\ell-m-1)} \\
  &\phantom{{}={}}{} - \sqrt{(\ell+m)(\ell-m)(\ell+m)(\ell-m-1)} \\
  &= (\ell+m+1)\sqrt{(\ell-m)(\ell-m-1)} - (\ell+m)\sqrt{(\ell-m)(\ell-m-1)} \\
  &= \sqrt{(\ell-m)(\ell-m-1)}
\end{align*}
and similarly
\begin{align*}
  &\sqrt{((\ell+1)^2-m^2)(\ell+m+2)(\ell-m+1)} \\*
  &\phantom{{}={}}{} - \sqrt{(\ell+m+1)(\ell-m)((\ell+1)^2-(m+1)^2)} \\
  &= \sqrt{(\ell+m+1)(\ell+m+2)}.
\end{align*}
So we get that
\begin{align*}
  F_+ \xi &= \sqrt{(\ell-m)(\ell-m-1)} D_-E_+ \xi - \sqrt{(\ell+m+1)(\ell-m)} D_0E_+\xi \\*
  &\phantom{{}={}}{} - \sqrt{(\ell+m+1)(\ell+m+2)} D_+E_+\xi
\end{align*}
for $\xi\in R_{\ell,m}$ and $-\ell \leq m \leq \ell$.

Similarly we get that
\begin{align*}
  F_- \xi &= -\sqrt{(\ell+m)(\ell+m-1)} D_-E_- \xi - \sqrt{(\ell+m)(\ell-m+1)} D_0E_- \xi \\*
  &\phantom{{}={}}{} - \sqrt{(\ell-m+1)(\ell-m+2)} D_+E_- \xi
\end{align*}
for $\xi\in R_{\ell,m}$, and thus indeed we get \cref{eq:Factions}. 

% \begin{align*}
%   F_- \xi &= [H_-,F_3] \xi = H_-F_3 \xi - F_3H_- \xi \\
%           &= \sqrt{\ell^2-m^2} H_-D_- \xi - m H_-D_0 \xi - \sqrt{(\ell+1)^2-m^2} H_-D_+ \xi \\*
%           &\phantom{{}={}}{} - \sqrt{(\ell+m)(\ell-m+1)} F_3 E_-\xi \\
%           &= \sqrt{\ell^2-m^2} \sqrt{((\ell-1)+m)((\ell-1)-m+1)} E_-D_- \\*
%           &\phantom{{}={}}{} - m\sqrt{(\ell+m)(\ell-m+1)} E_-D_0\xi \\*
%           &\phantom{{}={}}{} - \sqrt{(\ell+1)^2-m^2}\sqrt{((\ell+1)+m)((\ell+1)-m+1)} E_-D_+ \\*
%           &\phantom{{}={}}{} - \sqrt{(\ell+m)(\ell-m+1)} \Bigl( \sqrt{\ell^2-(m-1)^2}D_-E_-\xi - (m-1)D_0E_-\xi \\*
%           &\phantom{{}={}}{} - \sqrt{(\ell+1)^2-(m-1)^2} D_+E_-\xi \Bigr) \\
%           &= \sqrt{\ell^2-m^2} \sqrt{(\ell+m-1)(\ell-m)} D_-E_- - m\sqrt{(\ell+m)(\ell-m+1)} D_0E_-\xi \\*
%           &\phantom{{}={}}{} - \sqrt{(\ell+1)^2-m^2}\sqrt{(\ell+m+1)(\ell-m+2)} D_+E_- \\*
%           &\phantom{{}={}}{} - \sqrt{(\ell+m)(\ell-m+1)} \Bigl( \sqrt{\ell^2-(m-1)^2}D_-E_-\xi - (m-1)D_0E_-\xi \\*
%           &\phantom{{}={}}{} - \sqrt{(\ell+1)^2-(m-1)^2} D_+E_-\xi \Bigr) \\
%           &= -\Bigl( \sqrt{(\ell+m)(\ell-m+1)(\ell^2-(m-1)^2)} \\*
%           &\phantom{{}={}()}{} - \sqrt{(\ell^2-m^2)(\ell+m-1)(\ell-m)}  \Bigr)D_-E_-\xi \\*
%           &\phantom{{}={}}{} - \sqrt{(\ell+m)(\ell-m+1)} D_0E_-\xi \\*
%           &\phantom{{}={}}{} - \Bigl( \sqrt{((\ell+1)^2-m^2)(\ell+m+1)(\ell-m+2)} \\*
%           &\phantom{{}={}()}{} - \sqrt{(\ell+m)(\ell-m+1)((\ell+1)^2-(m-1)^2)}\Bigr)D_+E_-\xi
% \end{align*}
% for $\xi\in R_{\ell,m}$ and $-\ell+1 < m\leq \ell-1$. Again note that by the problematic terms vanish in such a way that this formula holds true for all $m$ with $-\ell\leq m\leq \ell$. Also note that
% \begin{align*}
%   &\sqrt{(\ell+m)(\ell-m+1)(\ell^2-(m-1)^2)} - \sqrt{(\ell^2-m^2)(\ell+m-1)(\ell-m)} \\*
%   &= \sqrt{(\ell+m)(\ell+m-1)}
% \end{align*}
% and
% \begin{align*}
%   &\sqrt{((\ell+1)^2-m^2)(\ell+m+1)(\ell-m+2)} \\*
%   &\phantom{{}={}}{} - \sqrt{(\ell+m)(\ell-m+1)((\ell+1)^2-(m-1)^2)} \\*
%   &= \sqrt{(\ell-m+1)(\ell-m+2)},
% \end{align*}
% so we get that
% \begin{align*}
%   F_- \xi &= -\sqrt{(\ell+m)(\ell+m-1)} D_-E_- \xi - \sqrt{(\ell+m)(\ell-m+1)} D_0E_- \xi \\*
%   &\phantom{{}={}}{} - \sqrt{(\ell-m+1)(\ell-m+2)} D_+E_- \xi
% \end{align*}
% for $\xi\in R_{\ell,m}$. Thus indeed we get \cref{eq:Factions}. 

%%% Local Variables:
%%% mode: latex
%%% TeX-master: "../main"
%%% End:
