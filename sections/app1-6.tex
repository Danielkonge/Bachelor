\section{\texorpdfstring{Relations for $D_0$, $D_+$, $D_-$}{Relations for D\_0,D\_+,D\_-}}

We want to show that the formulae \cref{eq:Factions} for the linear operators $F_+$, $F_-$, and $F_3$ together with the formulae \cref{eq:H3eigen,eq:E+andE-} for $H_+$, $H_-$, and $H_3$ define a representation of $L$, i.e. they satisfy the commutation relations of \cref{eq:lierels}, if and only if $D_0$, $D_+$, and $D_-$ satisfy \cref{eq:Drels}.

We claim that all commutators containing $H$'s already satisfy the relations. We already have by construction that all relations with only $H$'s satisfy the relations, so we just need to check the relations for commutators with one $F$ and one $H$. We will here check that $[H_-,F_+] = -2F_3$ as we want, and then we just note that the rest follow by similar considerations.

To simplify the otherwise very long calculations, we will only check the equility on $R_{\ell,\ell}$ and just note that similar calculations show the result on general $R_{\ell,m}$. By \cref{eq:Factions,eq:H+H-} first note that for $\xi\in R_{\ell,\ell}$ with $\ell\neq 0$ we have that $F_+\xi = \sqrt{(2\ell+1)(2\ell+2)}E_+D_+\xi\in R_{\ell+1,\ell+1}$, so
\begin{align*}
  H_-F_+ \xi &= \sqrt{(2\ell+2)\cdot 1}\sqrt{(2\ell+1)(2\ell+2)}E_-E_+D_+\xi \\
           &= 2(\ell+1)\sqrt{2\ell+1}D_+\xi \\
           &= 2\ell\sqrt{2\ell+1}D_+\xi + 2\sqrt{2\ell+1}D_+\xi,
\end{align*}
while $H_-\xi = \sqrt{2\ell}E_-\xi \in R_{\ell,\ell-1}$, so
\begin{align*}
  F_+H_- \xi &= -\sqrt{1\cdot2\ell}\sqrt{2\ell}D_0E_+E_-\xi + \sqrt{2\ell(2\ell+1)}\sqrt{2\ell}E_+D_+E_-\xi \\
           &= -2\ell D_0 \xi + 2\ell\sqrt{2\ell+1}E_+E_-D_+\xi \\
           &= -2\ell D_0 \xi + 2\ell\sqrt{2\ell+1} D_+ \xi. \\
\end{align*}
Therefore we get that
\begin{align*}
  [H_-,F_+] \xi &= H_-F_+ \xi - F_+H_- \xi \\
  &= 2\ell D_0\xi + 2\sqrt{2\ell+1}D_+\xi.
\end{align*}
Also
\begin{align*}
  -2F_3 \xi &= 2\ell D_0\xi + 2\sqrt{(\ell+1)^2-\ell^2} D_+\xi \\
  &= 2\ell D_0\xi + 2\sqrt{2\ell+1} D_+\xi,
\end{align*}
and thus indeed $[H_-,F_+]\xi = -F_3\xi$ for $\xi\in R_{\ell,\ell}$. 

Now we claim that $\ell D_+D_0\xi = (\ell+2)D_0D_+\xi$ corresponds to the relation $[F_+,F_3]=F_+$, while $(\ell+1)D_-D_0\xi = (\ell-1)D_0D_-\xi$ corresponds to the relation $[F_-,F_3]=-H_-$, and $\xi=(2\ell-1)D_+D_- \xi - (2\ell+3)D_-D_+ \xi - D_0^2 \xi$ corresponds to the relation $[F_+,F_-]=-2H_3$. Here will only show the last claim and note that the others are similar (although a little simpler), and that the first and second cliam can be shown first without assuming the third.

Again to simplify the otherwise very long calculation, we will only check the result on $R_{\ell,\ell}$. Note that for $\xi\in R_{\ell,\ell}$ with $\ell\neq 0$ we have that $F_+\xi = \sqrt{(2\ell+1)(2\ell+2)}E_+D_+\xi \in R_{\ell+1,\ell+1}$, so
\begin{align*}
  F_-F_+ \xi &= -\sqrt{(2\ell+2)(2\ell+1)}\sqrt{(2\ell+1)(2\ell+2)} D_-E_-E_+D_+ \xi \\*
           &\phantom{{}={}}{} - \sqrt{2\ell+2}\sqrt{(2\ell+1)(2\ell+2)} D_0E_-E_+D_+ \xi \\*
           &\phantom{{}={}}{} - \sqrt{2}\sqrt{(2\ell+1)(2\ell+2)}E_-D_+E_+D_+ \xi \\
           &= -(2\ell+1)(2\ell+2) D_-D_+ \xi - (2\ell+2)\sqrt{2\ell+1} D_0D_+ \xi \\*
           &\phantom{{}={}}{} - 2\sqrt{(2\ell+1)(\ell+1)} E_+E_-D_+D_+ \xi \\
           &= -(2\ell+1)(2\ell+2) D_-D_+ \xi - (2\ell+2)\sqrt{2\ell+1} D_0D_+ \xi \\*
           &\phantom{{}={}}{} - 2\sqrt{(2\ell+1)(\ell+1)} D_+D_+ \xi.
\end{align*}
Likewise we have that
\begin{align*}
  F_- \xi &= -\sqrt{2\ell(2\ell-1)}D_-E_- \xi - \sqrt{2\ell}D_0E_-\xi - \sqrt{2}E_-D_+\xi,
\end{align*}
where $D_-E_-\xi \in R_{\ell-1,\ell-1}$, $D_0E_-\xi \in R_{\ell,\ell-1}$, and $E_-D_+\xi \in R_{\ell+1,\ell-1}$, so
\begin{align*}
  F_+F_- \xi &= -\sqrt{(2\ell-1)2\ell}\sqrt{2\ell(2\ell-1)}E_+D_+D_-E_- \xi \\*
           &\phantom{{}={}}{} + \sqrt{2\ell}\sqrt{2\ell}D_0E_+D_0E_- \xi - \sqrt{2\ell(2\ell+1)}\sqrt{2\ell}E_+D_+D_0E_-\xi \\*
           &\phantom{{}={}}{} - \sqrt{2}\sqrt{2}D_-E_+E_-D_+ \xi + \sqrt{2(2\ell+1)}\sqrt{2}D_0E_+E_-D_+ \xi \\*
           &\phantom{{}={}}{} - \sqrt{(2\ell+1)(2\ell+2)}\sqrt{2} E_+D_+E_-D_+ \xi \\
           &= -2\ell(2\ell-1) E_+D_+D_-E_- \xi + 2\ell D_0^2 \xi - 2\ell\sqrt{2\ell+1}D_+D_0 \xi \\*
           &\phantom{{}={}}{} - 2 D_-D_+\xi + 2\sqrt{2\ell+1}D_0D_+ - 2\sqrt{(2\ell+1)(\ell+1)} D_+^2 \xi.
\end{align*}
Hence we get that
\begin{align*}
  [F_+,F_-]\xi &= F_+F_-\xi - F_-F_+ \xi \\
             &= -2\ell(2\ell-1) E_+D_+D_-E_- \xi + 2\ell D_0^2 \xi - 2\ell\sqrt{2\ell+1}D_+D_0 \xi \\*
             &\phantom{{}={}}{} - \bigl(2-(2\ell+1)(2\ell+2)\bigr) D_-D_+\xi + (2+(2\ell+2))\sqrt{2\ell+1}D_0D_+ \\
             &= -2\ell(2\ell-1) E_+D_+D_-E_- \xi + 2\ell D_0^2 \xi - 2\ell\sqrt{2\ell+1}D_+D_0 \xi \\*
             &\phantom{{}={}}{} + 2\ell(2\ell+3) D_-D_+\xi + 2(\ell+2)\sqrt{2\ell+1}D_0D_+\xi \\
             &= -2\ell(2\ell-1) E_+D_+D_-E_- \xi + 2\ell D_0^2 \xi \\*
             &\phantom{{}={}}{} + 2\sqrt{2\ell+1}\bigl((\ell+2)D_0D_+\xi -\ell D_+D_0 \xi \bigr) \\*
             &\phantom{{}={}}{} + 2\ell(2\ell+3) D_-D_+\xi + 2(\ell+2)\sqrt{2\ell+1}D_0D_+\xi \\
             &= -2\ell(2\ell-1) E_+D_+D_-E_- \xi + 2\ell D_0^2 \xi + 2\ell(2\ell+3) D_-D_+\xi,
\end{align*}
since $2-(2\ell+1)(2\ell+2)=2-4\ell^2-6\ell-2=-4\ell^2-6\ell=-2\ell(2\ell+3)$ and $(\ell+2)D_0D_+ \xi = \ell D_+D_0 \xi$. Also
\begin{align*}
  -2H_3 \xi &= -2\ell \xi,
\end{align*}
so dividing by $-2\ell$ (noting that $\ell\neq 0$) we see that $[F_+,F_-]\xi = -2H_3 \xi$ for $\xi \in R_{\ell,\ell}$ if and only if
\begin{align*}
  \xi = (2\ell-1)E_+D_+D_-E_- \xi - (2\ell+3)D_-D_+ \xi - D_0^2 \xi.
\end{align*}
Here we note that for $\xi \in R_{\ell,m}$, $m\neq\pm \ell$, the term $E_+D_+D_-E_-\xi$ simplifies to $D_+D_-\xi$, while for $m=-\ell$ it becomes $E_-D_+D_-E_+\xi$. 

%%% Local Variables:
%%% mode: latex
%%% TeX-master: "../main"
%%% End:
