\section{\texorpdfstring{The singular category $C(\lambda_1,\lambda_2)$}{The singular category C(lambda\_1,lambda\_2)}}

Now we want to describe the singular category $C(\lambda_1,\lambda_2)$, i.e.\ Harish-Chandra modules for the pair $(L,L_k)$ with $\ell_1-\ell_0$ an integer. The description of such modules turns out to be quite a bit more complicated than in the non-singular case, where a finite dimensional vector space $P$ and a nilpotent linear map $a\colon P\to P$ describes the module.

We define at first a category $S_0$ as follows. The objects $\widetilde{A}$ of $S_0$ are finite dimensional vector spaces $P_1$ and $P_2$ with four linear maps
\begin{align}
  d_+ \colon P_1\to P_2, && d_- \colon P_2\to P_1, && \delta_1\colon P_1\to 0 && \delta_2\colon P_2\to P_2
\end{align}
that satisfy the conditions
\begin{align}
  \begin{aligned}
    &d_-\delta_2 = \delta_2d_+ = 0, \\
    &\delta_2 \text{ and } d_+d_- \text{ are nilpotent.}
  \end{aligned}
\end{align}
The morphisms $\gamma\colon \widetilde{A}\to \widetilde{A}'$ in $S_0$ are pairs of linear maps $\gamma=(\gamma_1,\gamma_2)$ with $\gamma_1\colon P_1\to P_1'$ and $\gamma_2\colon P_2\to P_2'$ such that the diagram
\begin{equation}\label{eq:S0diagram}
  \begin{tikzcd}
    P_1 \ar[d,"\gamma_1"] \ar[r,"d_+"] & P_2 \ar[d,"\gamma_2"] \ar[r,"\delta_2"] & P_2 \ar[d,"\gamma_2"] \ar[r,"d_-"] & P_1 \ar[d,"\gamma_1"] \\
    P_1' \ar[r,"d_+'",swap] & P_2' \ar[r,"\delta_2'",swap] & P_2' \ar[r,"d_-'",swap] & P_1'
  \end{tikzcd}
\end{equation}
commutes. Similarly to the non-singular case we now want to prove that the singular category $C(\lambda_1,\lambda_2)$ is equivalent to the category $S_0$, but before we can do that we need some lemmas.

\begin{lemma}
  In a singular module $M\in C(\lambda_1,\lambda_2)$ all the subspaces $R_{\ell,m}$ for $\ell_0\leq \ell\leq \abs{\ell_1}-1$ and all $m$ where it makes sense have the same dimension. The subspaces $R_{\ell,m}$ for $\ell\geq \abs{\ell_1}$ and all $m$ where it makes sense also have the same dimension. Furthermore the linear maps
  \begin{align*}
    D_+&\colon R_{\ell,m}\to R_{\ell+1,m} && \text{for }\ell\neq \abs{\ell_1}-1,\\
    D_-&\colon R_{\ell,m}\to R_{\ell-1,m} && \text{for }\ell\neq\ell_0,\abs{\ell_1}
  \end{align*}
  are isomorphisms.
\end{lemma}
\begin{proof}
  By \cref{eq:dformulaeonR_l}, we have that the eigenvalues $\dd{\ell}{-}$ and $\dd{\ell}{+}$ for $D_+D_-$ and $D_-D_+$ on $R_{\ell,m}$ are 
  \begin{align*}
    \dd{\ell}{+} &= \dfrac{((\ell+1)^2-\ell_0^2)(\ell_1^2-(\ell+1)^2)}{(4(\ell+1)^2-1)(\ell+1)^2}, & \dd{\ell}{-} &= \dfrac{(\ell^2-\ell_0^2)(\ell_1^2-\ell^2)}{(4\ell^2-1)\ell^2}
  \end{align*}
  for $\ell\neq \ell_0$ in the case of $\dd{\ell}{-}$, where $\dd{\ell_0}{-}=0$. Now $M$ is singular, so $\ell_1$ is real because $\ell_1-\ell_0$ is an integer and $\ell_0$ is real, and $\abs{\ell_1}-\ell_0$ is a positive integer by assumption, so we get that $\dd{\ell}{+}=0$ only for $\ell=\abs{\ell_1}-1=\ell_0+(\abs{\ell_1}-\ell_0)-1\in \Set{\ell_0,\ell_0+1,\dotsc}$ and $\dd{\ell}{-}=0$ only for $\ell=\ell_0$ and $\ell=\abs{\ell_1}=\ell_0+(\abs{\ell_1}-\ell_0)\in \Set{\ell_0+1,\ell_0+2,\dotsc}$. Hence the maps
  \begin{align*}
    D_-D_+&\colon R_{\ell,m}\to R_{\ell,m} && \text{for }\ell\neq \abs{\ell_1}-1,\\
    D_+D_-&\colon R_{\ell,m}\to R_{\ell,m} && \text{for }\ell\neq \ell_0,\abs{\ell_1} \text{ and }m\neq \pm \ell
  \end{align*}
  have diagonals without zeros in their Schur decomposition, so they are invertible, and thus the maps
  \begin{align*}
    D_+&\colon R_{\ell,m}\to R_{\ell+1,m} && \text{for }\ell\neq \abs{\ell_1}-1,\\
    D_-&\colon R_{\ell,m}\to R_{\ell-1,m} && \text{for }\ell\neq \ell_0,\abs{\ell_1} \text{ and }m\neq \pm \ell
  \end{align*}
  are injective. Since $E_+$ and $E_-$ are isomorphisms we already have that $R_{\ell,m}$ and $R_{\ell,m'}$ have the same dimension as we have already seen a few times, and therefore the above implies that the subspaces $R_{\ell,m}$ for $\ell=\ell_0,\dotsc,\abs{\ell_1}-1$ and $m$ where it makes sense all have the same dimension, and that the subspaces $R_{\ell,m}$ for $\ell\geq \abs{\ell_1}$ and $m$ where it makes sense all have the same dimension.
\end{proof}

Since $E_+$, $E_-$, $D_+$, and $D_-$ commute with $\Delta_1$ and $\Delta_2$ by \Cref{lem:commuteDeltas}, we get by the same method as in the proof of \Cref{lem:nonsingsimilar} that:

\begin{lemma}\label{lem:singmodDeltasimilar}
  In a singular module $M\in C(\lambda_1,\lambda_2)$ all the operators $(\Delta_i)_{\ell,m}$ with $\ell_0\leq \ell\leq \abs{\ell_1}-1$ are similar to $(\Delta_i)_{\ell_0,m_0}$ via the same matrix for $i=1,2$, and the operators $(\Delta_i)_{\ell,m}$ with $\ell\geq \abs{\ell_1}$ are similar to $(\Delta_i)_{\abs{\ell_1},m_0}$ via the same matrix for $i=1,2$.
\end{lemma}

\begin{lemma}\label{lem:Deltarelsing}
  Define in the singular module $M\in C(\lambda_1,\lambda_2)$ a linear operator $\delta$ by
  \begin{align}
    \delta = \dfrac{1}{\ell_0^2-\ell_1^2}\Bigl( \Delta_2 + \dfrac{\Delta_1^2}{\ell_0^2} - (\ell_0^2-1)\id \Bigr)
  \end{align}
  for $\ell_0\neq 0$. Then on the subspaces $R_{\ell,m}$ for which $\ell_0\leq \ell\leq \abs{\ell_1}-1$ the operator $\delta$ is zero, and on the remaining subspaces $R_{\ell,m}$ $\delta$ is nilpotent.
\end{lemma}
\begin{proof}
  That $\delta=0$ on the subspaces $R_{\ell,m}$ for which $\ell_0\leq \ell\leq \abs{\ell_1}-1$ follows by \Cref{lem:singmodDeltasimilar} and the same argument as in the proof of \Cref{lem:Deltarelnonsing}. And since $\Delta_1$ and $\Delta_2$ only have one eigenvalues $\lambda_1=-i\ell_0\ell1$ and $\lambda_2=\ell_0^2+\ell_1^2-1$ respectively we see that $\delta$ only has the eigenvalues
  \begin{align*}
    \dfrac{1}{\ell_0^2-\ell_1^2}\Bigl( \lambda_2 + \dfrac{\lambda_1^2}{\ell_0^2} - (\ell_0^2-1) \Bigr) &=\dfrac{1}{\ell_0^2-\ell_1^2}\Bigl( \ell_0^2+\ell_1^2-1 - \ell_1^2 - (\ell_0^2-1) \Bigr) \\
    &=0,
  \end{align*}
  so by Cayley-Hamilton Theorem $\delta$ is nilpotent in general which gives the result.
\end{proof}

\begin{remark}\label{rem:D+D-nilpotent}
  Using \cref{eq:dformulaeonR_l} the argument showing that $\delta$ is nilpotent can also be used to show that
  \begin{align*}
    D_-D_+\colon R_{\abs{\ell_1}-1,m} &\to R_{\abs{\ell_1}-1,m} \\
    D_+D_-\colon R_{\abs{\ell_1},m} &\to R_{\abs{\ell_1},m}
  \end{align*}
  for $m\neq \pm \ell_1$ in the latter case, are both nilpotent.
\end{remark}

\begin{remark}
  Comparing \Cref{lem:Deltarelsing} with \Cref{lem:Deltarelnonsing}, we see that this property of $\delta$ is one of the key differences between the singular and non-singular cases.
\end{remark}

We thus want to understand the map $\delta$ better for which we will need the lemma:

\begin{lemma}\label{lem:deltaDszero}
  We have that
  \begin{align}
    \begin{aligned}
      \delta D_+ \xi &= 0 && \text{for }\xi\in R_{\abs{\ell_1}-1,m},\\
      D_-\delta \xi &= 0 && \text{for }\xi\in R_{\abs{\ell_1},m}.
    \end{aligned}
  \end{align}
\end{lemma}
\begin{proof}
  The operator $\delta$ is a linear combination of $\Delta_1^2$, $\Delta_2$, and $\id$, so it commutes with $D_+$ and $D_-$. Hence since $\delta=0$ on $R_{\abs{\ell_1}-1,m}$ by \Cref{lem:Deltarelsing} $\delta D_+ \xi = D_+\delta \xi = 0$ for $\xi\in R_{\abs{\ell_1}-1,m}$ and $D_-\delta \xi = \delta D_-\xi = 0$ for $\xi \in R_{\abs{\ell_1},m}$ since then $D_-\xi\in R_{\abs{\ell_1}-1,m}$.
\end{proof}

Now we are ready to begin showing the equivalence of the singular category $C(\lambda_1,\lambda_2)$ and the category $S_0$. First we will show that we from objects $M\in C(\lambda_1,\lambda_2)$ can construct objects $\widetilde{A}\in S_0$.

Let $M\in C(\lambda_1,\lambda_2)$ be a singular module, and consider the maps
\begin{align*}
  D_+\colon R_{\abs{\ell_1}-1,m_0} &\to R_{\abs{\ell_1},m_0}, & D_-\colon R_{\abs{\ell_1},m_0}&\to R_{\abs{\ell_1}-1,m_0}, \\
  \delta\colon R_{\abs{\ell_1}-1,m_0} &\to 0, & \delta\colon R_{\abs{\ell_1},m_0} &\to R_{\abs{\ell_1},m_0}
\end{align*}
for some $m_0$ where it makes sense. Writing $P_1=R_{\abs{\ell_1}-1,m_0}$ and $P_2=R_{\abs{\ell_1},m_0}$ for the finite dimensional vector spaces and $d_+$, $d_-$, $\delta_1$, and $\delta_2$ for the maps above, we have that $(P_1,P_2,d_+,d_-,\delta_1,\delta_2)$ is an object of the category $S_0$, since $d_-\delta_2=\delta_2d_+=0$ by \Cref{lem:deltaDszero}, $\delta_2$ is nilpotent by \Cref{lem:Deltarelsing}, and $d_+d_-$ is nilpotent by \Cref{rem:D+D-nilpotent}.

Likewise from a morphism $\Gamma\colon M\to M'$ for $M,M'\in C(\lambda_1,\lambda_2)$ we get a corresponding morphism $\gamma=(\gamma_1,\gamma_2)$ from $\widetilde{A}$ to $\widetilde{A}'$. As in the proof of \Cref{thm:nonsingcateq} we have that $\Gamma R_{\ell,m} \subset R_{\ell,m}'$ for all $\ell$ and $m$ where it makes sense, so setting $\gamma_1=(\Gamma)_{\abs{\ell_1}-1,m_0}\colon P_1\to P_1'$ and $\gamma_2 = (\Gamma)_{\abs{\ell_1},m_0}\colon P_2\to P_2'$, we want to show that $\gamma=(\gamma_1,\gamma_2)$ gives a morphism in $S_0$. \todo{Show that the diagram \cref{eq:S0diagram} commutes}

Now we want to show that to each object $\widetilde{A}=(P_1,P_2,d_+,d_-,\delta_1,\delta_2)\in S_0$ there is a corresponding singular module $M\in C(\lambda_1,\lambda_2)$ with $\lambda_1=-i\ell_0\ell_1$ and $\lambda_2=\ell_0^2+\ell_1^2-1$. Let such an $\widetilde{A}$ be given, then we want to construct $M$. First choose an $m_0$ such that it makes sense to write $R_{\abs{\ell_1}-1,m_0}$ and put $R_{\abs{\ell_1}-1,m_0}=P_1$ and $R_{\abs{\ell_1},m_0}=P_2$, and define
\begin{align*}
  D_+\coloneqq d_+\colon R_{\abs{\ell_1}-1,m_0}&\to R_{\abs{\ell_1},m_0}, & D_-\coloneqq d_-\colon R_{\abs{\ell_1},m_0} &\to R_{\abs{\ell_1}-1,m_0}.
\end{align*}
Consider the space $M=\bigoplus_{\ell,m}R_{\ell,m}$ for $\ell=\ell_0,\ell_0+1,\dotsc,\abs{\ell_1}-1,\abs{\ell_1},\dotsc$ and $m$ where it makes sense, where $\dim R_{\ell,m}=\dim R_{\abs{\ell_1}-1,m}=\dim P_1$ for all $\ell\leq \abs{\ell_1}-1$, and $\dim R_{\ell,m}=\dim R_{\abs{\ell_1},m}=\dim P_2$ for all $\ell\geq \abs{\ell_1}$. We want to construct the maps $E_+$, $E_-$, $D_+$, $D_-$, and $D_0$. First take $E_+\colon R_{\ell,m}\to R_{\ell,m+1}$ to be any isomorphism for $m\neq \ell$ and $E_+\colon R_{\ell,\ell}\to 0$, and then take $E_-\colon R_{\ell,m}\to R_{\ell,m-1}$ for $m\neq -\ell$ to be the inverse isomorphism of the corresponding $E_+$ and take $E_-\colon R_{\ell,-\ell}\to 0$. Likewise take $D_+\colon R_{\ell,m_0}\to R_{\ell+1,m_0}$ for $\ell\neq \abs{\ell_1}-1$ to be any isomorphism, and for $\ell=\abs{\ell_1}-1$ we already have defined $D_+$ on $R_{\abs{\ell_1}-1,m_0}$. We expand to all $R_{\ell,m}$ and $R_{\abs{\ell_1}-1,m}$ by using $E_+$ either on the left or on the right such that
\[
  \begin{tikzcd}
    R_{\ell,m+1} \ar[r,"D_+"] & R_{\ell+1,m+1} \\
    R_{\ell,m}\ar[u,"E_+"] \ar[r,"D_+",swap] & R_{\ell+1,m}\ar[u,"E_+",swap]
  \end{tikzcd}
\]
commutes for $-\ell\leq m<\ell$, i.e.\ $(D_+)_{\ell,m+1}=(E_+)_{\ell+1,m} (D_+)_{\ell,m} (E_+)_{\ell,m}^{-1}$. 

Now before constructing $D_0$ and $D_-$ we want to construct $\Delta_1$ and $\Delta_2$. Since $D_-$ and $\delta=\delta_2$ are defined on $R_{\abs{\ell_1},m_0}$, we can define
\begin{align*}
  \Delta_2\xi = (\ell_1^2+\ell_0^2-1)\xi + \ell_1^2\dfrac{4\ell_1^2-1}{\ell_1^2-\ell_0^2} D_+D_-\xi + \ell_0^2\delta\xi
\end{align*}
for $\xi \in R_{\abs{\ell_1},m_0}$. Since $D_+D_-=d_+d_-$ and $\delta=\delta_2$ on $R_{\abs{\ell_1},m_0}$ are nilpotent by assumption, we see that $\Delta_2$ has only one eigenvalue and that is $\ell_1^2+\ell_0^2-1$. Now since we still want the equation
\begin{align*}
  \delta = \dfrac{1}{\ell_0^2-\ell_1^2}\Bigl( \Delta_2 + \dfrac{\Delta_1^2}{\ell_0^2} - (\ell_0^2-1)\id \Bigr)
\end{align*}
to hold true, we define
\begin{align*}
  \Delta_1^2\xi &= \ell_0^2\Bigl( (\ell_0^2-1)\xi - \Delta_2\xi + (\ell_0^2-\ell_1^2)\delta\xi   \Bigr) \\
         &= \ell_0^2\Bigl( (\ell_0^2-1)\xi - (\ell_1^2+\ell_0^2-1)\xi - \ell_1^2\dfrac{4\ell_1^2-1}{\ell_1^2-\ell_0^2} D_+D_- \xi - \ell_0^2\delta \xi \\*
         &\phantom{{}={}}{} + (\ell_0^2-\ell_1^2)\delta\xi   \Bigr) \\
         &= \ell_0^2\Bigl( -\ell_1^2 \xi - \ell_1^2\dfrac{4\ell_1^2-1}{\ell_1^2-\ell_0^2} D_+D_- \xi - \ell_1^2\delta\xi \Bigr)\\
         &= -\ell_0^2\ell_1^2\Bigl( \xi + \dfrac{4\ell_1^2-1}{\ell_1^2-\ell_0^2} D_+D_- \xi + \delta\xi \Bigr)
\end{align*}
for $\xi \in R_{\abs{\ell_1},m_0}$. Again since $D_+D_-$ and $\delta$ are nilpotent on $R_{\abs{\ell_1},m_0}$ by assumption, we see that $\Delta_1^2$ only has the eigenvalue $-\ell_0^2\ell_1^2$, which is non-zero, and thus $\Delta_1$'s only eigenvalue is non-zero. Therefore \Cref{sec:linearmapfromsquare} implies that the above determines $\Delta_1$ uniquely on $R_{\abs{\ell_1},m_0}$.

For $\ell\geq \abs{\ell_1}$ consider the map $J_{r,s}=(E_+)^s(D_+)^r$ as earlier, where $(E_+)^{-1}=E_-$, and put 
\begin{align*}
  (\Delta_i)_{\ell,m} = J_{\ell-\abs{\ell_1},m-m_0}(\Delta_i)_{\abs{\ell_1},m_0}J_{\ell-\abs{\ell_1},m-m_0}^{-1}
\end{align*}
for $i=1,2$, such that we have expanded $\Delta_1$ and $\Delta_2$ to all $R_{\ell,m}$. Now following the formulae of \cref{eq:D0fromDelta1,eq:D+D-fromDeltas} we put
\begin{align*}
  (D_0)_{\ell,m} &= - \dfrac{1}{\ell(\ell+1)}(\Delta_1)_{\ell,m},\\
  (D_+D_-)_{\ell,m} &= \dfrac{1}{4\ell^2-1}\Bigl((\Delta_2)_{\ell,m} - (\ell^2-1)(\id)_{\ell,m} + \dfrac{(\Delta_1)_{\ell,m}^2}{\ell^2} \Bigr).
\end{align*}
Since $(D_+)_{\ell,m}$ is an isomorphism for all $\ell\geq \abs{\ell_1}$, we get that this determines $(D_-)_{\ell,m}$ for $\ell\geq \abs{\ell_1}+1$, and we already have $(D_-)_{\abs{\ell_1},m}$ given.

Similarly we can find $D_0$ and $D_-$ for $\ell_0\leq \ell <\abs{\ell_1}$, and thus we get the linear maps $E_+$, $E_-$, $D_+$, $D_-$, and $D_0$ defined everywhere where it makes sense, and actually we get by our standard construction operators $H_+$, $H_-$, $H_3$, $F_+$, $F_-$, and $F_3$, which gives a representation, i.e. we indeed get a singular module $M\in C(\lambda_1,\lambda_2)$ for each object $\widetilde{A}=(P_1,P_2,d_+,d_-,\delta_1,\delta_2)\in S_0$. To see this we need only check that the equations of \cref{eq:Drels} hold. As in \Cref{thm:objectscor} we see that for $\ell\neq 0$
\begin{align*}
  \ell(D_+)_{\ell,m}(D_0)_{\ell,m} &= -\dfrac{\ell}{\ell(\ell+1)} (D_+)_{\ell,m}(\Delta_1)_{\ell,m} \\
                          &= - \dfrac{1}{\ell+1} (\Delta_1)_{\ell+1,m} (D_+)_{\ell,m} \\
                          &= - (\ell+2)\dfrac{1}{(\ell+1)(\ell+2)} (\Delta_1)_{\ell+1,m} (D_+)_{\ell,m} \\
                          &= (\ell+2)(D_0)_{\ell+1,m}(D_+)_{\ell,m},
\end{align*}
which gives us the first equation, and the rest can be checked similarly. Additionally it can be checked like in \Cref{thm:nonsingcateq} that there is a correspondence between the morphisms, and thus altogether we get the theorem:
\begin{theorem}
  The singular category $C(\lambda_1,\lambda_2)$ is equivalent to the category $S_0$. 
\end{theorem}

\begin{corollary}\label{cor:singindec}
  An indecomposable module $M$ in the singular category $C(\lambda_1,\lambda_2)$ corresponds to an indecomposable object $A$ in the category $S_0$.
\end{corollary}

Now we have reduced the problem of characterizing the singular indecomposable modules to a problem of linear algebra, or more precisely to working with linear relations, which is beyond the scope of this paper --- for the details of this see \cite{indecompReprOfLorGr} Chapter II. However we will now give the resulting description:

We divide the indecomposable objects of $S_0$ into two types that we call open or closed. A simple type of open object $A$ from which we build general open objects is called a strand, and it has the following properties: $P_1$ has a basis $(e_1,\dotsc,e_n)$ and $P_2$ has a basis $(f_0,f_1,\dotsc,f_n,f_1',f_2',\dotsc,f_m')$ such that the operators $d_+$, $d_-$, and $\delta$ satisfy
\begin{align*}
  d_-f_i &= e_{i+1} \quad (i<n), & d_-f_n &= 0, & d_-f_i' &= 0, & d_+e_i &= f_i, \\
  \delta f_i &= 0 \quad (i>0), & \delta f_0 &= f_1', & \delta f_i' &= f_{i+1}' \quad (i<m), & \delta f_m' &= 0.
\end{align*}
Here $f_n$ and $f_m'$ are called tail vectors in the strand, and the two numbers $(n,m)$ are invariants of a strand.

An open object is composed of strands and is given by the set of numbers $(s,n_1,m_1,n_2,m_2,\dotsc,n_k,m_k)$, where $s\in \Set{0,1}$,  $n_1\geq 0$, $n_i>0$ for $i\neq 1$, $m_i>0$ for $i\neq k$, $m_k\geq -1$. Here the $i$'th pair $(n_i,m_i)$ corresponds to the $i$'th strand, and the tail vectors of these strands must satisfy
\begin{align}\label{eq:taileqs}
  f_{m_1}' = f_{n_2}, \quad f_{m_2}' = f_{n_3}, \quad \dotsc \quad f_{m_{k-1}}' = f_{n_k}.
\end{align}
Given this one can show the following proposition:
\begin{proposition}
  Let $A$ and $A'$ be two open indecomposable objects in $S_0$ which are given respectively by $(s,n_1,m_1,\dotsc,n_k,m_k)$ and $(s',n_1',m_1'\dotsc,\\ n_k',m_k')$. Then $A$ and $A'$ are equivalent if and only if $s=s'$, $n_i=n_i'$, and $m_i=m_i'$ for all $i$.
\end{proposition}

A simple closed object is obtained from one open object with $s=0$, $n_1>0$, $m_k>0$, and $f_{n_1}$ and $f_{m_k}'$ non-zero. Here $f_{n_1}$ is called the starting vector and $f_{m_k}'$ the terminating vector of the open object. The closed simple object is defined simply by adding the relation $f_{m_k}'=\mu f_{n_1}$ to the relations of \cref{eq:taileqs}, where $\mu$ is some complex number. Thus in the simple case a closed object is given by numbers $(n_1,m_1,\dotsc,n_k,m_k,\mu)$, where $n_i,m_i>0$ and $\mu\in\C$.

In general a closed object is given by numbers $(n_1,m_1,\dotsc,n_k,m_k,\mu,N)$, where $n_i,m_i,N>0$ are integers and $\mu\in\C$, and it is made from $N$ open objects given by $(0,n_1,m_1,\dotsc,n_k,m_k)$ for the same $n_i$ and $m_i$ as above. If we denote for $j=1,\dotsc,N$ the starting object of the $j$'th open object by $f_j$ and the terminating object by $f_j'$, then the closed object is given by adding the relations 
\begin{align*}
  f_1' = \mu f_1, \qquad f_i' = \mu f_i + f_{i-1} \mbox{ for }i=2,3,\dotsc,N.
\end{align*}
Given this one can as in the open case show the following proposition:
\begin{proposition}
  Let $A$ and $A'$ be two closed indecomposable objects in $S_0$ which are given respectively by $(n_1,m_1,\dotsc,n_k,m_k,\mu,N)$ and $(n_1',m_1',\dotsc,n_k',m_k',\mu',N')$. Then $A$ and $A'$ are equivalent if and only if $\mu=\mu'$, $N=N'$, and the sequence $\Set{n_i',m_i'}$ is a cyclic permutation of the sequence $\Set{n_i,m_i}$. 
\end{proposition}

Now since all indecomposable objects in $S_0$ are either open or closed, this gives a complete description of the indecomposables of $S_0$, and thus by \Cref{cor:singindec} a description of the indecomposables of singular Harish-Chandra modules in $C(\lambda_1,\lambda_2)$. Altogether we end up with the theorem:
\begin{theorem}
  Let $M$ be an indecomposable Harish-Chandra module for the pair $(L,L_k)$ belonging to the singular category $C(\lambda_1,\lambda_2)$. Then either M can be described by some integers $(s,n_1,m_1,n_2,m_2,\dotsc,n_k,m_k)$ or by some integers and a complex number $\mu$ $(n_1,m_1,\dotsc,n_k,m_k,\mu,N)$, and two such modules $M$ and $M'$ are equivalent if and only if $\lambda_1=\lambda_1'$, $\lambda_2=\lambda_2'$, $\mu=\mu'$ (in the case with $\mu$ in the description), and all the corresponding integers agree.  
\end{theorem}

%%% Local Variables:
%%% mode: latex
%%% TeX-master: "../main"
%%% End:
