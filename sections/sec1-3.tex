\section{\texorpdfstring{The non-singular category $C(\lambda_1,\lambda_2)$}{The non-singular category C(lambda\_1,lambda\_2)}}

Let $M\in C(\lambda_1,\lambda_2)$ be an $L$-module, where $(\lambda_1,\lambda_2)$ is a non-singular pair, i.e.\ $\ell_1-\ell_0$ is not an integer. We now want to that this module $M$ is completely determined by a finite dimensional vector space and a nilpotent map $a$ on this vector space, where an isomorphism of the modules is equivalent to similarity of the linear map $a$.

Define on the finite dimensional linear subspace $R_{\ell_0,m_0}$ for some $m_0\in \Set{-\ell_0,-\ell_0+1,\dotsc,\ell_0-1,\ell_0}$ a linear map $a\colon R_{\ell_0,m_0}\to R_{\ell_0,m_0}$ by
\begin{align}\label{eq:adef}
  a\xi = D_0\xi - \dfrac{i\ell_1}{\ell_0+1}\xi
\end{align}
for $\xi\in R_{\ell_0,m_0}$. This map is nilpotent since by \Cref{prop:eigenvaluesforDs} the only eigenvalue of $D_0$ on $R_{\ell_0}$ is
\begin{align*}
  \dd{\ell_0}{0} = \dfrac{i\ell_1}{\ell_0+1},
\end{align*}
and
\begin{align*}
  \det(a-t\id) = \det\Bigl( D_0 - \Bigl(t+\dfrac{i\ell_1}{\ell_0+1} \Bigr)\id \Bigr),
\end{align*}
so the only eigenvalue of $a$ on $R_\ell$ is zero, and thus $a$ is clearly nilpotent by Cayley-Hamilton Theorem.

We want to show that the module $M$ is completely determined by the finite dimensional vector space $R_{\ell_0,m_0}$ and the linear map $a\colon R_{\ell_0,m_0}\to R_{\ell_0,m_0}$ when $C(\lambda_1,\lambda_2)$ is non-singular. To do this we first need some lemmas.

\begin{lemma}\label{lem:Disos}
  In a non-singular module $M\in C(\lambda_1,\lambda_2)$, the maps
  \begin{align*}
    D_+\colon R_{\ell,m} \to R_{\ell+1,m} \\
    D_-\colon R_{\ell+1,m} \to R_{\ell,m}
  \end{align*}
  for $\ell\geq \ell_0$ are isomorphisms. 
\end{lemma}
\begin{proof}
  By \Cref{prop:eigenvaluesforDs} the eigenvalues of $D_+D_-$ and $D_-D_+$ on $R_\ell$ for $\ell\neq 0,\tfrac{1}{2}$ are
  \begin{align*}
    \dd{\ell}{-} &= \dfrac{(\ell^2-\ell_0^2)(\ell_1^2-\ell^2)}{(4\ell^2-1)\ell^2} & \mbox{for }\ell>\ell_0,\\
    \dd{\ell}{+} &= \dfrac{((\ell+1)^2-\ell_0^2)(\ell_1^2-(\ell+1)^2)}{(4(\ell+1)^2-1)(\ell+1)^2}.
  \end{align*}
  Now by assumption $M$ is non-singular, i.e.\ $\ell_1-\ell_0$ is not an integer, and we want to show that $\ell_1^2-\ell^2\neq 0$ for all $\ell\in \Set{\ell_0,\ell_0+1,\dotsc}$.

  Assume that $\ell_1^2-\ell^2=0$ for some $\ell=\ell_0+k$, where $k$ is a non-negative integer. Since $\ell_1-\ell_0$ is not an integer, we also have that $\ell_1-(\ell_0+k)$ is not an integer and hence not equal to zero, so we must have that $\ell_1+\ell_0+k=0$ since $\ell_1^2-\ell^2=(\ell_1-\ell)(\ell_1+\ell)$. But this would imply that $\ell_1=-\ell_0-k$ and hence $\ell_1-\ell_0=-2\ell_0-k$ is an integer, which is a contradiction with the non-singularity of $M$.

  Thus we see that $\ell_1^2-\ell^2$ is non-zero for all $\ell\in \Set{\ell_0,\ell_0+1,\dotsc}$, and therefore the eigenvalues $\dd{\ell}{+}$ and $\dd{\ell}{-}$ are different from zero for all $\ell$ except $\ell_0$ in the case of $\dd{\ell}{-}$. Hence the maps $D_+D_-\colon R_{\ell,m}\to R_{\ell,m}$ for $\ell\neq\ell_0$ and $D_-D_+\colon R_{\ell,m}\to R_{\ell,m}$ have diagonals without zeros in the Schur decomposition, so they are invertible. Therefore $D_-\colon R_{\ell+1,m}\to R_{\ell,m}$ and $D_+\colon R_{\ell,m}\to R_{\ell+1,m}$ are injective, and thus $\dim R_{\ell,m} = \dim R_{\ell+1,m}$, which again implies that $D_-$ and $D_+$ as above are actually isomorphisms.
\end{proof}

\begin{lemma}\label{lem:nonsingsimilar}
  In a non-singular module $M\in C(\lambda_1,\lambda _2)$ the Laplace operators $\Delta_1$ and $\Delta_2$ are such that each operator $(\Delta_i)_{\ell,m}$ is similar to $(\Delta_i)_{\ell_0,m_0}$ via the same matrix for $i=1,2$. 
\end{lemma}
\begin{proof}
  Recall that the maps $E_+\colon R_{\ell_0,m} \to R_{\ell_0,m+1}$ for $-\ell_0\leq m<\ell_0$ and $E_-\colon R_{\ell_0,m} \to R_{\ell_0,m-1}$ for $-\ell_0<m\leq \ell_0$ are isomorphisms, and the Laplace operators $\Delta_1$ and $\Delta_2$ commute with these maps by \Cref{lem:commuteDeltas}, so $(\Delta_i)_{\ell_0,m}$ is similar to $(\Delta_i)_{\ell_0,m_0}$ for each $m$ and $i=1,2$.

  Likewise the $D_+\colon R_{\ell,m}\to R_{\ell+1,m}$ are also isomorphisms for all $\ell$ and commute with both $\Delta_1$ and $\Delta_2$, so the map $(\Delta_i)_{\ell_0+1,m}$ is similar to $(\Delta_i)_{\ell_0,m}$, and inductively $(\Delta_i)_{\ell,m}$ is similar to $(\Delta_i)_{\ell_0,m}$ for all $\ell\in \Set{\ell_0,\ell_0+1,\dotsc}$. Hence indeed $(\Delta_i)_{\ell,m}$ is similar to $(\Delta_i)_{\ell_0,m_0}$, $i=1,2$, and by construction it is clearly by the same matrix for $i=1,2$.
\end{proof}

\begin{lemma}\label{lem:Deltarelnonsing}
  If $M\in C(\lambda_1,\lambda_2)$ is a non-singular module, then the Laplace operators $\Delta_1$ and $\Delta_2$ are connected on the whole of $M$ by the relation
  \begin{align}\label{eq:Deltarelinnonsing}
    \Delta_1^2 + \ell_0^2\Delta_2 - \ell_0^2(\ell_0^2-1)\id = 0.
  \end{align}
\end{lemma}
\begin{proof}
  Suppose that $\ell_0\neq 0$. By \cref{eq:Deltastemp} we get that
  \begin{align}\label{eq:deltatempcalc}
    \begin{aligned}
      \Delta_2 \xi + \dfrac{\Delta_1^2}{\ell_0^2}\xi - (\ell_0^2-1)\id \xi &= (\ell_0^2-1)\xi - (\ell_0+1)^2D_0^2\xi \\*
      &\phantom{{}={}}{} + (\ell_0+1)^2D_0^2\xi - (\ell_0^2-1)\xi = 0,
    \end{aligned}
  \end{align}
  for $\xi\in R_{\ell_0,m_0}$, so multiplying by $\ell_0^2$ we get \cref{eq:Deltarelinnonsing} on $R_{\ell_0,m_0}$. By \Cref{lem:nonsingsimilar} $(\Delta_i)_{\ell,m}$ is similar to $(\Delta_i)_{\ell_0,m_0}$ via the same matrix for $i=1,2$, so the relation holds true for any $\xi\in R_{\ell,m}$, and thus on all of $M$.

  Suppose otherwise that $\ell_0=0$. Then \cref{eq:Deltastemp} implies that $(\Delta_1)_{0,0}$ is zero, and thus the relation follows easily on $R_{0,0}$, and we can expand to all of $M$ as above.
\end{proof}

\begin{remark}
  Note that \Cref{lem:Disos,lem:nonsingsimilar,lem:Deltarelnonsing} are not true in the singular case.
\end{remark}

We have seen above that to each non-singular module $M\in C(\lambda_1,\lambda_2)$ there is a corresponding finite dimensional vector space $P=R_{\ell_0,m_0}$ and a nilpotent linear map $a\colon P\to P$ given by
\begin{align}\label{eq:adef}
  a\xi = \Bigl( D_0 - \dfrac{i\ell_1}{\ell_0+1}\id\Bigr) \xi
\end{align}
for $\xi\in R_{\ell_0,m_0}=P$. Denote now by $\widetilde{A}$ the pair $(P,a)$ consisting of a finite dimensional vector space $P$ and a nilpotent mapping $a\colon P\to P$.

\begin{theorem}\label{thm:objectscor}
  To each pair $\widetilde{A}$ and non-singular pair $(\lambda_1,\lambda_2)$ of numbers there is a corresponding $L$-module $M\in C(\lambda_1,\lambda_2)$ such that $P=R_{\ell_0,m_0}$ and $a$ is related to $D_0$ by \cref{eq:adef}.
\end{theorem}
\begin{proof}
  Denote by $R_{\ell_0,m_0}$ the space $P$ and consider the linear transformation
  \begin{align*}
    D_0 \xi = a\xi + \dfrac{i\ell_1}{\ell_0+1}\xi
  \end{align*}
  for $\xi\in R_{\ell_0,m_0}$. Consider the space
  \begin{align*}
    M = \bigoplus_{\substack{\ell\in\Set{\ell_0,\ell_0+1,\dotsc} \\ m\in\Set{-\ell,-\ell+1,\dotsc,\ell-1,\ell}}} \!\!\!\!\!\!\!\!\!\!\!\! R_{\ell,m},
  \end{align*}
  which is a direct sum of vector spaces with $\dim R_{\ell,m} = \dim P$ for all $\ell$ and $m$.

  Now take an isomorphism $E_+\colon R_{\ell,m}\to R_{\ell,m+1}$ for $m\neq \ell$ and put $E_+\colon R_{\ell,\ell}\to 0$, which we can do since $\dim R_{\ell,m}=\dim R_{\ell,m+1}$. Define an isomorphism $E_-\colon R_{\ell,m+1}\to R_{\ell,m}$ such that it is inverse to $E_+\colon R_{\ell,m}\to R_{\ell,m+1}$ and put $E_-\colon R_{\ell,-\ell}\to 0$. Take now isomorphisms $D_+\colon R_{\ell,m_0}\to R_{\ell+1,m_0}$ for some fixed $m_0$, and define on all the remaining $R_{\ell,m}$ linear maps $D_+\colon R_{\ell,m}\to R_{\ell+1,m}$ such that the diagram
  \[
    \begin{tikzcd}
      R_{\ell,m+1} \ar[r,"D_+"] & R_{\ell+1,m+1} \\
      R_{\ell,m}\ar[u,"E_+"] \ar[r,"D_+",swap] & R_{\ell+1,m}\ar[u,"E_+",swap]
    \end{tikzcd}
  \]
  commutes for $-\ell\leq m<\ell$, i.e.\ $(D_+)_{\ell,m+1}=(E_+)_{\ell+1,m} (D_+)_{\ell,m} (E_+)_{\ell,m}^{-1}$, where $(E_+)_{\ell,m}^{-1}=(E_-)_{\ell,m+1}$. Now we only need to construct linear maps $D_0$ and $D_-$ on $M$ satisfying properties as we have seen earlier, but to do this we will first define linear maps $\Delta_1$ and $\Delta_2$ corresponding to the Laplace operators.

  On $R_{\ell_0,m_0}$ set 
  \begin{align}
    \begin{aligned}\label{eq:DeltasRell0m0}
      \Delta_1 \xi &= -\ell_0(\ell_0+1)D_0\xi \\
      &= -\ell_0(\ell_0+1)a\xi - i\ell_1\ell_0\xi, \\
      \Delta_2 \xi &= (\ell_0^2-1)\xi - (\ell_0+1)^2D_0^2\xi \\
      &= (\ell_0^2-1)\xi + \ell_1^2\xi - (\ell_0+1)i\ell_1a\xi - (\ell_0+1)^2a^2\xi \\
      &= (\ell_0^2+\ell_1^2-1)\xi - (\ell_0+1)^2\Bigl( a^2\xi + 2\dfrac{i\ell_1}{\ell_0+1}a\xi \Bigr)
    \end{aligned}
  \end{align}
  for $\xi \in R_{\ell_0,m_0}$. Now note that for arbitrary $R_{\ell,m}$ the linear map given by $J_{\ell,m}=(E_+)^{m-m_0}(D_+)^{\ell-\ell_0}\colon R_{\ell_0,m_0} \to R_{\ell,m}$, where $(E_+)^{-1} = E_-$, is a composition of isomorphisms and hence an isomorphism, so we can define 
  \begin{align*}
    (\Delta_i)_{\ell,m}\xi = J_{\ell,m}(\Delta_i)_{\ell_0,m_0} (J_{\ell,m})^{-1}\xi
  \end{align*}
  for $\xi\in R_{\ell,m}$ and $i=1,2$. Thus we have defined $\Delta_1$ and $\Delta_2$ on all of $M$.
  
  Now define $D_0\colon R_{\ell,m}\to R_{\ell,m}$ by 
  \begin{align}\label{eq:D0fromDelta1}
    D_0\xi = -\dfrac{1}{\ell(\ell+1)}\Delta_1 \xi
  \end{align}
  for $\xi\in R_{\ell,m}$, $\ell\neq 0$, and $D_+D_-\colon R_{\ell,m}\to R_{\ell,m}$ by 
  \begin{align}
    \begin{aligned}\label{eq:D+D-fromDeltas}
      D_+D_-\xi &= \dfrac{1}{4\ell^2-1}(\Delta_2\xi - (\ell^2-1)\xi + (\ell+1)^2D_0^2\xi) \\
      &= \dfrac{1}{4\ell^2-1}\Bigl(\Delta_2\xi - (\ell^2-1)\xi + \dfrac{\Delta_1^2}{\ell^2}\xi\Bigr)
    \end{aligned}
  \end{align}
  for $\xi \in R_{\ell,m}$, $\ell\neq \ell_0$, which we can do since $D_+$ is an isomorphism. Using this we define $D_-\colon R_{\ell,m}\to R_{\ell-1,m}$ to be the map $(D_+)^{-1}(D_+D_-)$ for $\ell\neq \ell_0$, and equal to zero for $\ell=\ell_0$. 

  Now the maps $E_+$, $E_-$, $D_0$, $D_+$, and $D_-$ constructed above are maps as in \Cref{sec:formulae} (by construction the $E$'s and $D$'s commute and satisfy the other properties we want), so the operators $F_+$, $F_-$, $F_3$, $H_+$, $H_-$, and $H_3$ constructed from these maps as in \cref{eq:H3eigen,eq:E+andE-,eq:Factions} gives $M$ an $L$-module structure if the equations of \cref{eq:Drels} hold. We see that for $\ell\neq 0$
\begin{align*}
  \ell(D_+)_{\ell,m}(D_0)_{\ell,m} &= -\dfrac{\ell}{\ell(\ell+1)} (D_+)_{\ell,m}(\Delta_1)_{\ell,m} \\
                          &= - \dfrac{1}{\ell+1} (\Delta_1)_{\ell+1,m} (D_+)_{\ell,m} \\
                          &= - (\ell+2)\dfrac{1}{(\ell+1)(\ell+2)} (\Delta_1)_{\ell+1,m} (D_+)_{\ell,m} \\
                          &= (\ell+2)(D_0)_{\ell+1,m}(D_+)_{\ell,m},
\end{align*}
which gives us the first equation, and the rest can be checked similarly, so we get an $L$-module structure on $M$.

  Finally we get $P=R_{\ell_0,m_0}$ and \cref{eq:adef} by construction, and we note that $M$ is non-singular since the pair $(\lambda_1,\lambda_2)$ with the corresponding $\ell_0$ and $\ell_1$ is non-singular by assumption.
\end{proof}

Looking at the construction of the $L$-module $M$ above, we get the following corollary:
\begin{corollary}
  For modules $M$ and $M'$ from the non-singular category $C(\lambda_1,\lambda_2)$ to be equivalent it is necessary and sufficient that the subspaces $R_{\ell_0,m_0}$ and $R_{\ell_0,m_0}'$ in these modules have the same dimension, and that their maps $D_0\colon R_{\ell_0,m_0}\to R_{\ell_0,m_0}$ and $D_0'\colon R_{\ell_0,m_0}'\to R_{\ell_0,m_0}'$ are similar. 
\end{corollary}

Now let $S$ be the category with objects pairs $A=(P,a)$, where $P$ is a finite dimensional vector space (over $\C$) and $a\colon P\to P$ is a nilpotent linear transformation, and with morphisms $\gamma\colon A\to A'$ given by linear maps $\gamma\colon P\to P'$ such that
\[
  \begin{tikzcd}
    P\ar[r,"a"] \ar[d,"\gamma",swap] & P \ar[d,"\gamma"] \\
    P' \ar[r,"a'",swap] & P'
  \end{tikzcd}
\]
commutes. We have already shown that there is a correspondence between non-singular modules $M\in C(\lambda_1,\lambda_2)$ and pairs $A=(P,a)\in S$, and now we want to show that this correspondence is functorial and actually gives an equivalence of categories.

\begin{theorem}\label{thm:nonsingcateq}
  The non-singular category $C(\lambda_1,\lambda_2)$ is equivalent to the category $S$.
\end{theorem}
\begin{proof}
  By \Cref{thm:objectscor} we have a correspondence between objects $M\in C(\lambda_1,\lambda_2)$ and the objects $A\in S$, so we just need to establish a correspondence between the morphisms. Let $\Gamma\colon M\to M'$ be a morphism between the two modules $M,M'\in C(\lambda_1,\lambda_2)$. We have already seen in the proof of \Cref{prop:modulehom} that $\Gamma R_\ell\subset R_\ell'$ since $\Gamma$ commutes with $H_3$, and this furthermore implies that $\Gamma R_{\ell,m}\subset R_{\ell,m}'$ since $\Gamma$ must map vectors of weight $m$ to either $0$ or vectors of weight $m$. Thus $\Gamma$ is the direct sum of morphisms $\gamma_{\ell,m}\colon R_{\ell,m}\to R_{\ell,m}'$, so choosing indices $\ell_0$ and $m_0$ we get a morphism $\gamma\coloneqq \gamma_{\ell_0,m_0}\colon R_{\ell_0,m_0}\to R_{\ell_0',m_0'}$. Here $\gamma$ gives a morphism in $S$ from $\widetilde{A}=(R_{\ell_0,m_0},a)$ to $\widetilde{A}'=(R_{\ell_0,m_0}',a')$. To see this by \cref{eq:adef} it is enough to show that $D_0'\gamma = \gamma D_0$, which is true for $\ell_0\neq0$\footnote{Recall that $D_0$ is not defined on $R_{0,0}$, so this is what we want to check.} since by the proof of \Cref{prop:modulehom} $\Delta_1'\Gamma=\Gamma\Delta_1$, so for $\ell\neq 0$ also $D_0'\Gamma=\Gamma D_0$ on $R_\ell$ because $\Delta_1=-\ell(\ell+1)D_0$ on $R_\ell$ and $\Delta_1'=-\ell(\ell+1)D_0'$ on $R_\ell'$, and therefore specifically $D_0' \gamma = \gamma D_0$. Hence $\gamma\colon \widetilde{A}\to \widetilde{A}'$ is indeed a morphism in $S$.

  Now suppose conversely that we are given a morphism $\gamma\colon A\to A'$ in $S$, i.e.\ a linear map $\gamma\colon P\to P'$ such that $\gamma a=a'\gamma$. Construct modules $M$ and $M'$ with $R_{\ell_0,m_0}=P$ and $R_{\ell_0,m_0}'=P'$ as in the proof of \Cref{thm:objectscor}, then we have a linear map $\gamma\colon R_{\ell_0,m_0}\to R_{\ell_0,m_0}'$ such that $\gamma a=a'\gamma$, which implies that $\gamma D_0=D_0'\gamma$ since $D_0=a+\tfrac{i\ell_1}{\ell_0+1}\id$. From this we will construct linear maps $\gamma_{\ell,m}\colon R_{\ell,m}\to R_{\ell,m}'$ by noting that $J_{\ell,m}=E_+^{m-m_0}D_+^{\ell-\ell_0}\colon R_{\ell_0,m_0}\to R_{\ell,m}$ is an isomorphism for $\ell$ and $m$ where it makes sense, so we get linear maps
  \begin{align}\label{eq:tempgammalm}
    \gamma_{\ell,m} = J_{\ell,m}' \gamma J_{\ell,m}^{-1},
  \end{align}
  where $J_{\ell,m}'\colon R_{\ell,m}'\to R_{\ell,m}'$ is as above also. We want to show that the direct sum $\Gamma$ of the $\gamma_{\ell,m}$ gives a morphism of $L$-modules from $M=\bigoplus_{\ell,m} R_{\ell,m}$ to $M'=\bigoplus_{\ell,m} R_{\ell,m}'$. 

  Recall, cf.\ \Cref{lem:nonsingsimilar}, that $(\Delta_i)_{\ell,m}$ is similar to $(\Delta_i)_{\ell_0,m_0}$ for all $\ell$ and $m$, $i=1,2$, with $(\Delta_i)_{\ell,m} = J_{\ell,m}(\Delta_i)_{\ell_0,m_0}J_{\ell,m}^{-1}$. Hence since $\gamma (\Delta_i)_{\ell_0,m_0} = (\Delta_i')_{\ell_0,m_0}\gamma$ by \cref{eq:Deltastemp} since $\gamma D_0 = D_0' \gamma$, we see that
  \begin{align*}
    \gamma_{\ell,m} (\Delta_i)_{\ell,m} &= J_{\ell,m}' \gamma J_{\ell,m}^{-1}J_{\ell,m}(\Delta_i)_{\ell_0,m_0}J_{\ell,m}^{-1} = J_{\ell,m}' \gamma (\Delta_i)_{\ell_0,m_0}J_{\ell,m}^{-1} \\
    &= J_{\ell,m}' (\Delta_i')_{\ell_0,m_0} \gamma J_{\ell,m}^{-1} = (\Delta_i')_{\ell,m} J_{\ell,m}'\gamma J_{\ell,m}^{-1} = (\Delta_i')_{\ell,m} \gamma_{\ell,m}.
  \end{align*}
  Now since
  \begin{align*}
    D_0 \xi &= -\dfrac{1}{\ell(\ell+1)} \Delta_1\xi,\\
    D_+D_- \xi &= \dfrac{1}{4\ell^2-1}\Bigl( \Delta_2\xi - (\ell^2-1)\xi + \dfrac{\Delta_1^2}{\ell^2}\xi \Bigr)
  \end{align*}
  for $\xi \in R_{\ell,m}$ with $\ell\neq 0$, cf.\ the proof of \Cref{thm:objectscor}, we get that
  \begin{align*}
    (D_0')_{\ell,m}\gamma_{\ell,m} = \gamma_{\ell,m}(D_0)_{\ell,m},\\
    (D_+'D_-')_{\ell,m}\gamma_{\ell,m} = \gamma_{\ell,m} (D_+D_-)_{\ell,m}.
  \end{align*}
  Also noting that since $E_+$ and $D_+$ commute where it makes sense, we have that $E_+J_{\ell,m}=J_{\ell,m}E_+=J_{\ell,m+1}$ and $D_+J_{\ell,m}=J_{\ell,m}D_+=J_{\ell+1,m}$, so
  \begin{align*}
    \gamma_{\ell,m} E_+ &= J_{\ell,m}' \gamma J_{\ell,m}^{-1} E_+ = E_+'J_{\ell,m-1}' \gamma J_{\ell,m-1}^{-1} = E_+'\gamma_{\ell,m-1},
  \end{align*}
  and likewise $\gamma_{\ell,m}D_+ = D_+'\gamma_{\ell-1,m}$ where it makes sense. Thus $\Gamma$ commutes with $D_+$ and $D_+D_-$, so by \Cref{lem:Disos} $\Gamma$ also commutes with $D_-$, and likewise since $E_-$ is the inverse of $E_+$ or zero, we get that $\Gamma$ commutes with $E_-$. Hence $\Gamma$ commutes with $E_+$, $E_-$, $D_0$, $D_+$, and $D_-$, so it commutes with $H_+$, $H_-$, $H_3$, $F_+$, $F_-$, and $F_3$, i.e.\ it is a morphism of $L$-modules.
\end{proof}

\begin{corollary}
  An indecomposable module $M$ in the non-singular category $C(\lambda_1,\lambda_2)$ corresponds to an indecomposable object $A$ in the category $S$.
\end{corollary}

Here it follows from linear algebra that the indecomposable objects $A\in S$ are finite dimensional vector spaces $P$ with nilpotent linear maps $a\colon P\to P$ whose matrices in a suitable basis have the form of a single Jordan block.

\begin{remark}
  Note that by the above an indecomposable non-singular Harish-Chandra module for the pair $(L,L_k)$ has one additional invariant when compared to the simple case. In the simple case we just need to numbers $\ell_0$ and $\ell_1$, but in the indecomposable case we additionally need a number $n$ giving the dimension of the Jordan block.
\end{remark}

Finally to end our description of the non-singular Harish-Chandra modules, we will give the explicit form of $E_+$, $E_-$, $D_+$, $D_-$, and $D_0$ in a non-singular Harish-Chandra module $M\in C(\lambda_1,\lambda_2)$. Here we denote by $[E_+]_{\ell,m}$ the matrix representation of the map $E_+\colon R_{\ell,m}\to R_{\ell,m+1}$ in a given basis, by $[E_-]_{\ell,m}$ the matrix representation of the map $E_-\colon R_{\ell,m}\to R_{\ell,m-1}$ in a given basis, and so on.

\begin{theorem}
  Let $M\in C(\lambda_1,\lambda_2)$ be a non-singular indecomposable Harish-Chandra module for the pair $(L,L_k)$. Then all the subspaces $R_{\ell,m}$, $\ell=\ell_0,\ell_0+1,\dotsc$, have the same dimension, and bases can be chosen in them such that $[E_+]_{\ell,m}$ for $m\neq\ell$, $[E_-]_{\ell,m}$ for $m\neq -\ell$, and $[D_+]_{\ell,m}$ are identity matrices. Furthermore the matrices $[D_0]_{\ell,m}$ and $[D_-]_{\ell,m}$ can be expressed in terms of the matrix
  \begin{align*}
    [a_0] =
    \begin{pmatrix}
      0 & 1 & 0 & \cdots & 0 \\
      0 & 0 & 1 & \cdots & 0 \\
      \vdots & \vdots & \vdots & \ddots & \vdots \\
      0 & 0 & 0 & \cdots & 1 \\
      0 & 0 & 0 & \cdots & 0 
    \end{pmatrix}
  \end{align*}
  by the formulae
  \begin{align*}
    [D_0]_{\ell,m} &= i \dfrac{\ell_0\ell_1}{\ell(\ell+1)}[\id] + \dfrac{\ell_0(\ell_0+1)}{\ell(\ell+1)}[a_0] \\
    [D_-]_{\ell,m} &= \dfrac{\ell_0^2-\ell^2}{\ell^2(4\ell^2-1)}\Bigl( (\ell^2-\ell_1^2)[\id] + 2i\ell_1(\ell_0+1)[a_0] + (\ell_0+1)^2[a_0]^2 \Bigr),
  \end{align*}
  where $[\id]$ is the identity matrix of the same dimension as $[a_0]$. 
\end{theorem}
\begin{proof}
  We follow the proof of \Cref{thm:objectscor}. First take $[D_0]$ to be the linear transformation
  \begin{align*}
    [D_0]\xi = [a_0]\xi + \dfrac{i\ell_1}{\ell_0+1}[\id]\xi
  \end{align*}
  for $\xi\in R_{\ell_0,m_0}$. Now following the construction in \Cref{thm:objectscor}, we get that all the subspaces $R_{\ell,m}$ have the same dimension, and we can find bases of $R_{\ell,m}$ such that $[E_+]_{\ell,m}$ for $m\neq\ell$, $[E_-]_{\ell,m}$ for $m\neq -\ell$, and $[D_+]_{\ell,m}$ are identity matrices. By \Cref{lem:commuteDeltas} $E_+$, $E_-$, and $D_+$ commute with $\Delta_i$, $i=1,2$, so $[\Delta_i]_{\ell,m}$ is independent of $\ell$ and $m$, so we can focus on $R_{\ell_0,m_0}$, where \cref{eq:DeltasRell0m0} gives us that
  \begin{align*}
    [\Delta_1]_{\ell_0,m_0} \xi &= -\ell_0(\ell_0+1)[a_0]\xi - i\ell_1\ell_0[\id]\xi, \\
    [\Delta_2]_{\ell_0,m_0} \xi &= (\ell_0^2+\ell_1^2-1)[\id]\xi - (\ell_0+1)^2\Bigl( [a_0]^2\xi + 2\dfrac{i\ell_1}{\ell_0+1}[a_0]\xi \Bigr) \\
    &= (\ell_0^2+\ell_1^2-1)[\id]\xi - 2i\ell_1(\ell_0+1)[a_0]\xi - (\ell_0+1)^2[a_0]^2\xi,
  \end{align*}
  for $\xi \in R_{\ell_0,m_0}$. Thus we get that
  \begin{align*}
    [\Delta_1]_{\ell,m} &= -\ell_0(\ell_0+1)[a_0] - i\ell_1\ell_0[\id], \\
    [\Delta_2]_{\ell,m} &= (\ell_0^2+\ell_1^2-1)[\id] - 2i\ell_1(\ell_0+1)[a_0] - (\ell_0+1)^2[a_0]^2,
  \end{align*}
  since $[\Delta_i]_{\ell,m}$ is independent of $\ell$ and $m$. Continuing as in \Cref{thm:objectscor}, we get by \cref{eq:D0fromDelta1,eq:D+D-fromDeltas} that
  \begin{align*}
    [D_0]_{\ell,m} &= - \dfrac{1}{\ell(\ell+1)}[\Delta_1]_{\ell,m} \\
                &= \dfrac{\ell_0(\ell_0+1)}{\ell(\ell+1)}[a_0] + \dfrac{i\ell_1\ell_0}{\ell(\ell+1)}[\id], \\
    [D_+D_-]_{\ell,m} &= \dfrac{1}{4\ell^2-1}\Bigl( [\Delta_2]_{\ell,m} - (\ell^2-1)[\id] + \dfrac{[\Delta_1]_{\ell,m}^2}{\ell^2} \Bigr) \\
                &= \dfrac{1}{(4\ell^2-1)\ell^2} \Bigl( (\ell_0^2+\ell_1^2-1)\ell^2[\id] - 2i\ell_1(\ell_0+1)\ell^2[a_0] \\*
                &\phantom{{}={}}{} - (\ell_0+1)^2\ell^2[a_0]^2 - (\ell^2-1)\ell^2[\id] + \ell_0^2(\ell_0+1)^2[a_0]^2  \\*
    &\phantom{{}={}}{} - \ell_1^2\ell_0^2[\id] + 2i\ell_0^2(\ell_0+1)\ell_1[a_0]  \Bigr) \\
    &= \dfrac{\ell_0^2-\ell^2}{(4\ell^2-1)\ell^2}\Bigl( (\ell^2-\ell_1^2)[\id] + 2i\ell_1(\ell_0+1)[a_0] + (\ell_0+1)^2[a_0]^2  \Bigr)
  \end{align*}
  since
  \begin{align*}
    (\ell_0^2+\ell_1^2-1)\ell^2 - (\ell^2-1)\ell^2 - \ell_1^2\ell_0^2 &= -\ell^4 + \ell_0^2\ell^2 + \ell_1^2\ell^2  - \ell_1^2\ell_0^2 \\
                                                 &= (\ell_0^2-\ell^2)(\ell^2-\ell_1^2), \\
    -2i\ell_1(\ell_0+1)\ell^2 + 2i\ell_0^2(\ell_0+1)\ell_1 &= (\ell_0^2-\ell^2)2i\ell_1(\ell_0+1), \\
    -(\ell_0+1)^2\ell^2 + \ell_0^2(\ell_0+1)^2 &= (\ell_0^2-\ell^2)(\ell_0+1)^2.
  \end{align*}
  Now since $[D_+]_{\ell,m}$ was just the identity matrix, we see that $[D_-]_{\ell,m}$ has the same expression as $[D_+D_-]_{\ell,m}$ and thus we get the theorem. 
\end{proof}

%%% Local Variables:
%%% mode: latex
%%% TeX-master: "../main"
%%% End:
