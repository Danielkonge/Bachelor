\section{\texorpdfstring{The non-singular category $C(\lambda_1,\lambda_2)$}{The non-singular category C(lambda\_1,lambda\_2)}}

Let $M\in C(\lambda_1,\lambda_2)$ be an $L$-module, where $(\lambda_1,\lambda_2)$ is a non-singular pair, i.e.\ $\ell_1-\ell_0$ is not an integer. We now want to that this module $M$ is completely determined by a finite dimensional vector space and a nilpotent map $a$ on this vector space, where an isomorphism of the modules is equivalent to similarity of the linear maps $a$.

Define on the finite dimensional linear subspace $R_{\ell_0,m_0}$ for some $m_0\in \Set{-\ell_0,-\ell_0+1,\dotsc,\ell_0-1,\ell_0}$ a linear map $a\colon R_{\ell_0,m_0}\to R_{\ell_0,m_0}$ by
\begin{align}\label{eq:adef}
  a\xi = D_0\xi - \dfrac{i\ell_1}{\ell_0+1}\xi
\end{align}
for $\xi\in R_{\ell_0,m_0}$. This map is nilpotent since by \Cref{prop:eigenvaluesforDs} the only eigenvalue of $D_0$ on $R_{\ell_0}$ is
\begin{align*}
  \dd{\ell_0}{0} = \dfrac{i\ell_1}{\ell_0+1}.
\end{align*}
We want to show that the module $M$ is completely determined by the finite dimensional vector space $R_{\ell_0,m_0}$ and the linear map $a\colon R_{\ell_0,m_0}\to R_{\ell_0,m_0}$ when $C(\lambda_1,\lambda_2)$ is non-singular. To do this we first need some lemmas.

\begin{lemma}\label{lem:Disos}
  In a non-singular module $M\in C(\lambda_1,\lambda_2)$, the maps
  \begin{align*}
    D_+\colon R_{\ell,m} \to R_{\ell+1,m} \\
    D_-\colon R_{\ell+1,m} \to R_{\ell,m}
  \end{align*}
  for $\ell\geq \ell_0$ are isomorphisms. 
\end{lemma}
\begin{proof}
  By \Cref{prop:eigenvaluesforDs} the eigenvalues of $D_+D--$ and $D_-D_+$ on $R_\ell$ for $\ell\neq 0,\tfrac{1}{2}$ are
  \begin{align*}
    \dd{\ell}{-} &= \dfrac{(\ell^2-\ell_0^2)(\ell_1^2-\ell^2)}{(4\ell^2-1)\ell^2} & \mbox{for }\ell>\ell_0,\\
    \dd{\ell}{+} &= \dfrac{((\ell+1)^2-\ell_0^2)(\ell_1^2-(\ell+1)^2)}{(4(\ell+1)^2-1)(\ell+1)^2}.
  \end{align*}
  Now by assumption $M$ is non-singular, i.e.\ $\ell_1-\ell_0$ is not an integer, and we want to show that $\ell_1^2-\ell^2\neq 0$ for all $\ell\in \Set{\ell_0,\ell_0+1,\dotsc}$.

  Assume that $\ell_1^2-\ell^2=0$ for some $\ell=\ell_0+k$, where $k$ is a non-negative integer. Since $\ell_1-\ell_0$ is not an integer, we also have that $\ell_1-(\ell_0+k)$ is not an integer and hence not equal to zero, so we must have that $\ell_1+\ell_0+k=0$ since $\ell_1^2-\ell^2=(\ell_1-\ell)(\ell_1+\ell)$. But this would imply that $\ell_1=-\ell_0-k$ and hence $\ell_1-\ell_0=-2\ell_0-k$ is an integer, which is a contradiction with the non-singularity of $M$.

  Thus we see that $\ell_1^2-\ell^2$ is non-zero for all $\ell\in \Set{\ell_0,\ell_0+1,\dotsc}$, and therefore the eigenvalues $\dd{\ell}{+}$ and $\dd{\ell}{-}$ are different from zero for all $\ell$ except $\ell_0$ in the case of $\dd{\ell}{-}$. Hence the maps $D_+D_-\colon R_{\ell,m}\to R_{\ell,m}$ for $\ell\neq\ell_0$ and $D_-D_+\colon R_{\ell,m}\to R_{\ell,m}$ correspond to multiplication by a non-zero constant, so $D_-\colon R_{\ell+1,m}\to R_{\ell,m}$ and $D_+\colon R_{\ell,m}\to R_{\ell+1,m}$ are injective, and thus the $\dim R_{\ell,m} = \dim R_{\ell+1,m}$, which again implies that $D_-$ and $D_+$ as above are actually isomorphisms.
\end{proof}

\begin{lemma}\label{lem:nonsingsimilar}
  In a non-singular module $M\in C(\lambda_1,\lambda_2)$ the Laplace operators $\Delta_1$ and $\Delta_2$ are such that each operator $(\Delta_i)_{\ell,m}$ is similar to $(\Delta_i)_{\ell_0,m_0}$ for $i=1,2$. 
\end{lemma}
\begin{proof}
  Recall that the maps $E_+\colon R_{\ell_0,m} \to R_{\ell_0,m+1}$ for $-\ell_0\leq m<\ell_0$ and $E_-\colon R_{\ell_0,m} \to R_{\ell_0,m-1}$ for $-\ell_0<m\leq \ell_0$ are isomorphisms, and the Laplace operators $\Delta_1$ and $\Delta_2$ commute with these maps by \Cref{lem:commuteDeltas}, so $(\Delta_i)_{\ell_0,m}$ is similar to $(\Delta_i)_{\ell_0,m_0}$ for each $m$ and $i=1,2$.

  Likewise the $D_+\colon R_{\ell,m}\to R_{\ell+1,m}$ are also isomorphisms for all $\ell$ and commute with both $\Delta_1$ and $\Delta_2$, so the map $(\Delta_i)_{\ell_0+1,m}$ is similar to $(\Delta_i)_{\ell_0,m}$, and inductively $(\Delta_i)_{\ell,m}$ is similar to $(\Delta_i)_{\ell_0,m}$ for all $\ell\in \Set{\ell_0,\ell_0+1,\dotsc}$. Hence indeed $(\Delta_i)_{\ell,m}$ is similar to $(\Delta_i)_{\ell_0,m_0}$.
\end{proof}

\begin{lemma}\label{lem:Deltarelnonsing}
  If $M\in C(\lambda_1,\lambda_2)$ is a non-singular module, then the Laplace operators $\Delta_1$ and $\Delta_2$ are connected on the whole of $M$ by the relation
  \begin{align}\label{eq:Deltarelinnonsing}
    \Delta_1^2 + \ell_0^2\Delta_2 - \ell_0^2(\ell_0^2-1)\id = 0.
  \end{align}
\end{lemma}
\begin{proof}
  Suppose that $\ell_0\neq 0$. By \cref{eq:Deltastemp} we get that
  \begin{align*}
    \Delta_2 \xi + \dfrac{\Delta_1^2}{\ell_0^2}\xi - (\ell_0^2-1)\id \xi &= (\ell_0^2-1)\xi - (\ell_0+1)^2D_0^2\xi \\*
    &\phantom{{}={}}{} + (\ell_0+1)^2D_0^2\xi - (\ell_0^2-1)\xi = 0,
  \end{align*}
  for $\xi\in R_{\ell_0,m_0}$, so multiplying by $\ell_0^2$ we get \cref{eq:Deltarelinnonsing} on $R_{\ell_0,m_0}$. By \Cref{lem:nonsingsimilar} $(\Delta_i)_{\ell,m}$ is similar to $(\Delta_i)_{\ell_0,m_0}$, so the relation holds true for any $\xi\in R_{\ell,m}$, and thus on all of $M$.

  Suppose otherwise that $\ell_0=0$. Then \cref{eq:Deltastemp} implies that $(\Delta_1)_{0,0}$ is zero, and thus the relation follows easily on $R_{0,0}$, and we can expand to all of $M$ as above.
\end{proof}

\begin{remark}
  Note that \Cref{lem:Disos,lem:nonsingsimilar,lem:Deltarelnonsing} are not true in the singular case.
\end{remark}

We have seen above that to each non-singular module $M\in C(\lambda_1,\lambda_2)$ there corresponds a finite dimensional vector space $P=R_{\ell_0,m_0}$ and a nilpotent linear map $a\colon P\to P$ given by
\begin{align*}
  a\xi = \Bigl( D_0 - \dfrac{i\ell_1}{\ell_0+1}\id\Bigr) \xi
\end{align*}
for $\xi\in R_{\ell_0,m_0}=P$. Denote now by $\tilde{A}$ the pair $(P,a)$ consisting of a finite dimensional vector space $P$ and a nilpotent mapping $a\colon P\to P$.

\begin{proposition}
  To each pair $\tilde{A}$ and non-singular pair $(\lambda_1,\lambda_2)$ of numbers there is a corresponding $L$-module $M\in C(\lambda_1,\lambda_2)$ such that $P=R_{\ell_0,m_0}$ and $a$ is related to $D_0$ by \cref{eq:adef}.
\end{proposition}
\begin{proof}
  Denote by $R_{\ell_0,m_0}$ the space $P$ and consider the linear transformation
  \begin{align*}
    D_0 \xi = a\xi + \dfrac{i\ell_1}{\ell_0+1}\xi
  \end{align*}
  for $\xi\in R_{\ell_0,m_0}$. Consider the space
  \begin{align*}
    M = \bigoplus_{\substack{\ell\in\Set{\ell_0,\ell_0+1,\dotsc} \\ m\in\Set{-\ell,-\ell+1,\dotsc,\ell-1,\ell}}} \!\!\!\!\!\!\!\!\!\!\!\! R_{\ell,m},
  \end{align*}
  which is a direct sum of vector spaces with $\dim R_{\ell,m} = \dim P$ for all $\ell$ and $m$.

  Now take an isomorphism $E_+\colon R_{\ell,m}\to R_{\ell,m+1}$ for $m\neq \ell$ and put $E_+\colon R_{\ell,\ell}\to 0$, which we can do since $\dim R_{\ell,m}=\dim R_{\ell,m+1}$. Define an isomorphism $E_-\colon R_{\ell,m+1}\to R_{\ell,m}$ such that it is inverse to $E_+\colon R_{\ell,m}\to R_{\ell,m+1}$ and put $E_-\colon R_{\ell,-\ell}\to 0$. Take now isomorphisms $D_+\colon R_{\ell,m_0}\to R_{\ell+1,m_0}$ for some fixed $m_0$, and define on all the remaining $R_{\ell,m}$ linear maps $D_+\colon R_{\ell,m}\to R_{\ell+1,m}$ such that the diagram
  \[
    \begin{tikzcd}
      R_{\ell,m+1} \ar[r,"D_+"] & R_{\ell+1,m+1} \\
      R_{\ell,m}\ar[u,"E_+"] \ar[r,"D_+",swap] & R_{\ell+1,m}\ar[u,"E_+",swap]
    \end{tikzcd}
  \]
  commutes for $-\ell\leq m<\ell$. Now we only need to construct linear maps $D_0$ and $D_-$ on $M$ satisfying properties as we have seen earlier, but to do this we will first define linear maps $\Delta_1$ and $\Delta_2$ corresponding to the Laplace operators.

  On $R_{\ell_0,m_0}$ set 
  \begin{align}
    \begin{aligned}
      \Delta_1 \xi &= -\ell_0(\ell_0+1)D_0\xi \\
      &= -\ell_0(\ell_0+1)a\xi - i\ell_1\ell_0\xi, \\
      \Delta_2 \xi &= (\ell_0^2-1)\xi - (\ell_0+1)^2D_0^2\xi \\
      &= (\ell_0^2-1)\xi + \ell_1^2\xi - (\ell_0+1)i\ell_1a\xi - (\ell_0+1)^2a^2\xi \\
      &= (\ell_0^2+\ell_1^2-1)\xi - (\ell_0+1)^2\Bigl( a^2\xi + 2\dfrac{i\ell_1}{\ell_0+1}a\xi \Bigr)
    \end{aligned}
  \end{align}
  for $\xi \in R_{\ell_0,m_0}$. Now note that for arbitrary $R_{\ell,m}$ the linear map $J_{\ell,m}=(E_+)^{m-m_0}(D_+)^{\ell-\ell_0}$ is a composition of isomorphisms and hence an isomorphism, so we can define 
  \begin{align*}
    (\Delta_i)_{\ell,m}\xi = J_{\ell,m}(\Delta_i)_{\ell_0,m_0} (J_{\ell,m})^{-1}
  \end{align*}
  for $\xi\in R_{\ell,m}$ and $i=1,2$. Thus we have defined $\Delta_1$ and $\Delta_2$ on all of $M$.
  
  Now define $D_0\colon R_{\ell,m}\to R_{\ell,m}$ by 
  \begin{align*}
    D_0\xi = -\dfrac{1}{\ell(\ell+1)}\Delta_1 \xi
  \end{align*}
  for $\xi\in R_{\ell,m}$, and $D_+D_-\colon R_{\ell,m}\to R_{\ell,m}$ by 
  \begin{align*}
    D_+D_-\xi = \dfrac{1}{4\ell^2-1}(\Delta_2\xi - (\ell^2-1)\xi + (\ell+1)^2D_0^2\xi) = \dfrac{1}{4\ell^2-1}\Bigl(\Delta_2\xi - (\ell^2-1)\xi + \dfrac{\Delta_1^2}{\ell^2}\xi)
  \end{align*}
  for $\xi \in R_{\ell,m}$, $\ell\neq \ell_0$, which we can do since $D_+$ is an isomorphism. Using this we define $D_-\colon R_{\ell,m}\to R_{\ell-1,m}$ to be the map $(D_+)^{-1}(D_+D_-)$ for $\ell\neq \ell_0$, and equal to zero for $\ell=\ell_0$. 

  Now the maps $E_+$, $E_-$, $D_0$, $D_+$, and $D_-$ constructed above satisfy the relations of \Cref{sec:formulae}, \todo{Write this more explicitely} so the operators $F_+$, $F_-$, $F_3$, $H_+$, $H_-$, and $H_3$ constructed from these maps as in \cref{eq:H3eigen,eq:E+andE-,eq:Drels} gives $M$ an $L$-module structure. Finally we get $P=R_{\ell_0,m_0}$ and \cref{eq:adef} by construction, and we note that $M$ is non-singular since the pair $(\lambda_1,\lambda_2)$ with the corresponding $\ell_0$ and $\ell_1$ is non-singular by assumption.
\end{proof}

\begin{corollary}
  For modules $M$ and $M'$ from the non-singular category $C(\lambda_1,\lambda_2)$ to be equivalent it is necessary and sufficient that the subspaces $R_{\ell_0,m_0}$ and $R_{\ell_0,m_0}'$ in these modules have the same dimension, and that their maps $D_0\colon R_{\ell_0,m_0}\to R_{\ell_0,m_0}$ and $D_0'\colon R_{\ell_0,m_0}$ are similar. 
\end{corollary}\todo{Write a little before this}

%%% Local Variables:
%%% mode: latex
%%% TeX-master: "../main"
%%% End:
