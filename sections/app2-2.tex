\section{Determining a linear map from its square and eigenvalue}

Let $A\colon V\to V$ be a linear operator on a finite dimensional vector space $V$ over $\C$, and assume that $A$ only has only one eigenvalue, $\lambda\neq 0$. We claim then that $A$ is uniquely determined by $A^2$ and $\lambda$. To see this first note that
\begin{align*}
  \ker(A-\lambda \id_V)^r = \ker(A^2-\lambda^2\id_V)^r
\end{align*}
for all integers $r>0$. This is the case since $(A^2-\lambda^2\id_V)^r = (A+\lambda\id_V)^r(A-\lambda\id_V)^r$, and since $A+\lambda\id_V$ is bijective for $\lambda\neq 0$ because $\det(A+\lambda\id_V)$ cannot be $0$ since $-\lambda$ is not an eigenvalue of $A$.

Now choose a basis of $\ker(A^2-\lambda^2\id_V)$, expand to a basis of $\ker(A^2-\lambda^2\id_V)^2$, expand further to a basis of $\ker(A^2-\lambda^2\id_V)^3$, and so on. Since $A-\lambda\id_V$ takes $\ker(A^2-\lambda^2\id_V)^r=\ker(A-\lambda\id_V)^r$ to $\ker(A^2-\lambda^2\id_V)^{r-1}=\ker(A-\lambda\id_V)^{r-1}$, we see that the matrix of $A$ with respect to this basis is upper triangular with all diagonal entries equal to $\lambda$. To see this more clearly suppose that $(v_1,\dotsc,v_s)$ is the basis of $\ker(A^2-\lambda^2\id_V)^{r-1}=\ker(A-\lambda\id_V)^{r-1}$ and that $(v_1,\dotsc,v_s,\dotsc,v_n)$ is the basis of $\ker(A^2-\lambda^2\id_V)^r = \ker(A-\lambda\id_V)^r$. Then we have that $(A-\lambda\id_V)v_\ell \in \ker(A-\lambda\id_V)^{r-1}$ for all $\ell\in\Set{1,\dotsc,n}$, i.e. $Av_\ell - \lambda v_\ell = \sum_{k=1}^s \beta_k v_k$ for some $\beta_k\in\C$, so for $\ell>s$ we have that $Av_\ell = \sum_{k=1}^s \beta_k v_k + \lambda v_\ell$, so by induction in $r$ we have the claim since for $v\in \ker(A-\lambda\id_V)$ we have that $Av = \lambda v$ which gives the base case. Write $A=(a_{ij})$ in this basis, where we now know that $a_{ij}=0$ if $i>j$, and note that writing $m=\dim V$ we have that
\begin{align*}
  (A^2)_{ij} &= \sum_{k=1}^m a_{ik}a_{kj} = \sum_{k=i}^j a_{ik}a_{kj} = \lambda a_{ij} + \sum_{k=i+1}^{j-1} a_{ik}a_{kj},
\end{align*}
since for $k<i$ $a_{ik}=0$ and for $k>j$ $a_{kj}=0$, and since $a_{ii}=a_{jj}=\lambda$. Hence by induction in $i-j$ we get that $A$ is determined by $A^2$ and $\lambda$, since clearly $a_{ii}=\lambda$ satisfies this and inductively we can find $a_{ij}$ from the above formuala knowing $A^2$, $\lambda$, and $a_{k\ell}$ with $k-\ell<i-j$.

%%% Local Variables:
%%% mode: latex
%%% TeX-master: "../main"
%%% End:
