\section{Decomposition of modules into indecomposables}

Now we want to continue our work using our knowledge of the classification of simple Harish-Chandra modules for the pair $(L,L_k)$ to begin our classification of indecomposable Harish-Chandra modules for the pair $(L,L_k)$. To do this we will first need to do some work with Laplace operators.

\subsection{Laplace operators}

Let $U(L)$ be the universal enveloping algebra of $L$, cf.\ \cite[Appendix~E]{jantzen}. We know, cf.\ \cite[p.~E-9]{jantzen}, that $M$ is an $L$-module if and only if it is an $U(L)$-module, so we can describe $L$-modules by describing $U(L)$-modules. To do this we will first need to have an explicit description of the center $Z(U(L))$ of $U(L)$. We will begin this description by first noting that $Z(U(\slpr))\simeq Z(U(\slc))\otimes Z(U(\slc))$, which follows from the fact that $Z(U(L_1\times L_2))\simeq Z(U(L_1))\otimes Z(U(L_2))$ for Lie algebras $L_1$ and $L_2$ in general cf.\ \Cref{sec:centeriso}.

It is a result in Lie algebra that $Z(U(\slc))$ is the algebra of polynomials in $C=h^2+2h+4yx$, i.e.\ $Z(U(\slc))=\C[C]$ (we saw this in Exercise~11 in the Lie algebra course). Thus we see that $Z(U(\slc))\otimes Z(U(\slc))$ is the algebra of polynomials in $C\otimes 1$ and $1\otimes C$, or equivalently the algebra of polynomials in $C\otimes 1 - 1\otimes C$ and $C\otimes 1 + 1\otimes C$. So we want to describe $C\otimes 1 - 1\otimes C$ and $C\otimes 1 + 1\otimes C$ in terms of our basis $h_+,h_-,h_3,f_+,f_-,f_3$. We note that $(u,u')\in L=\slpr$ in $U(L)=U(\slpr)$ is identified with $u\otimes 1 + 1\otimes u'$ in $U(\slc)\otimes U(\slc)$, so 
\begin{align*}
  &\tfrac{1}{2}(h_-f_++f_-h_+)+h_3f_3+f_3\\
  &= \tfrac{1}{2}\bigl( (y\otimes 1 + 1\otimes y)(ix\otimes 1 - 1\otimes ix) + (iy\otimes 1 - 1\otimes iy)(x\otimes 1 + 1\otimes x) \bigr) \\
  &\phantom{{}={}}{} + \tfrac{1}{4}(h\otimes 1+ 1\otimes h)(ih\otimes 1- 1\otimes ih) + \tfrac{1}{2}(ih\otimes 1- 1\otimes ih) \\
  &= \tfrac{1}{2}\bigl( iyx\otimes 1 - iy\otimes x + ix\otimes y - i\otimes yx + iyx\otimes 1 + iy\otimes x - ix\otimes y - i\otimes yx \bigr) \\
  &\phantom{{}={}}{} + \tfrac{1}{4}(ih^2\otimes 1 - ih\otimes h + ih\otimes h - i\otimes h^2) + \tfrac{1}{2}(ih\otimes 1- 1\otimes ih) \\
  &= \tfrac{1}{2}\bigl( 2iyx\otimes 1 - 2i\otimes yx \bigr) + \tfrac{1}{4}(ih^2\otimes 1 - i\otimes h^2) + \tfrac{1}{2}(ih\otimes 1- i\otimes h) \\
  &= iyx\otimes 1 + \tfrac{1}{4}ih^2\otimes 1 + \tfrac{1}{2}ih\otimes 1 - i\otimes yx - \tfrac{1}{4}i\otimes h^2 - \tfrac{1}{2}i\otimes h \\
  &= \tfrac{i}{4}(h^2+2h+yx)\otimes 1 - \tfrac{i}{4}1\otimes (h^2+2h+yx) \\
  &= \tfrac{i}{4}(C\otimes 1 - 1\otimes C),
\end{align*}
and likewise $h_-h_+ - f_-f_+ + h_3^2 - f_3^2 + 2h_3 = \tfrac{1}{2}(C\otimes 1 + 1\otimes C)$.

% Translating back to $Z(U(L))$ with the isomorphism $\psi$ from \cref{eq:U(L)iso}, we see that
% \begin{align*}
%   \psi(C\otimes 1 - 1\otimes C) &= \psi_1(C)\psi_2(1) - \psi_1(1)\psi_2(C) \\
%   &= (C,0) - (0,C) = (C,-C)
% \end{align*}
% since
% \begin{align*}
%   \psi_1(C)\psi_2(1) &= \psi_1(C)(1,1)\\
%                &=\psi_1(\iota_1(h)^2+2\iota_1(h)+4\iota_1(y)\iota_1(x)) \\
%          &=\psi_1\iota_1(h)^2+2\psi_1\iota_1(h)+4\psi_1\iota_1(y)\psi_1\iota_1(x) \\
%          &= \iota i_1(h)^2+2\iota i_1(h)+4\iota i_1(y)\iota i_1(x) \\
%          &= \iota(h,0)^2 + 2\iota(h,0) + 4\iota(y,0)\iota(x,0) \\
%          &= (C,0)
% \end{align*}
% and likewise $\psi_1(1)\psi_2(C)=(0,C)$. Similarly we get that $\psi(C\otimes 1+1\otimes C)=(C,C)$, so we want to describe $(C,-C)=(h^2+2h+4yx,-h^2-2h-4yx)$ and $(C,C)=(h^2+2h+4yx,h^2+2h+4yx)$ in terms of our basis $h_+,h_-,h_3,f_+,f_-,f_3$. We note that
% \begin{align*}
%   &\tfrac{1}{2}(h_-f_++f_-h_+)+h_3f_3+f_3\\
%   &= \tfrac{1}{2}\bigl( (y,y)(ix,-ix) + (iy,-iy)(x,x) \bigr) + \tfrac{1}{4}(h,h)(ih,-ih) + \tfrac{1}{2}(ih,-ih) \\
%   &= \tfrac{1}{2}(2iyx,-2iyx) + \tfrac{1}{4}(ih^2,-ih^2) + \tfrac{1}{2}(ih,-ih) \\
%   &= \tfrac{i}{4}(h^2+4yx+2h,-h^2-4yx-2h) \\
%   &= \tfrac{i}{4}(C,-C)
% \end{align*}
% and
% \begin{align*}
%   &h_-h_+ - f_-f_+ + h_3^2 - f_3^2 + 2h_3 \\
%   &= (y,y)(x,x) - (iy,-iy)(ix,-ix) + \tfrac{1}{4}(h,h)^2 - \tfrac{1}{4}(ih,-ih)^2 + (h,h) \\
%   &= (yx,yx) + (yx,yx) + \tfrac{1}{4}(h^2,h^2) + \tfrac{1}{4}(h^2,h^2) + (h,h) \\
%   &= \tfrac{1}{2}(h^2+2h+4yx,h^2+2h+4yx) \\
%   &= \tfrac{1}{2}(C,C).
% \end{align*}
Thus since the constants don't matter when we look at the algebra of polynomials in $C\otimes 1 - 1\otimes C$ and $C\otimes 1 + 1\otimes C$, we see that setting
\begin{align}\label{eq:DeltasaU(L)def}
  \begin{aligned}
  \Delta_1 &= \tfrac{1}{2}(h_-f_++f_-h_+)+h_3f_3+f_3, \\
  \Delta_2 &= h_-h_+ - f_-f_+ + h_3^2 - f_3^2 + 2h_3,
\end{aligned}
\end{align}
we have that $Z(U(L))$ is the algebra of polynomials in $\Delta_1$ and $\Delta_2$. Thus in term of the corresponding linear operators on a Harish-Chandra module $M$ for the pair $(L,L_k)$, we define linear operators
\begin{align}\label{eq:Deltasdef}
  \begin{aligned}
    \Delta_1 &\coloneqq \tfrac{1}{2}(H_-F_++F_-H_+) + H_3F_3 + F_3 \\
    \Delta_2 &\coloneqq H_-H_+ - F_-F_+ + H_3^2 - F_3^2 + 2H_3,
  \end{aligned}
\end{align}
which are called Laplace operators. Note that by \cref{eq:H3eigen,eq:H+H-,eq:Factions}, cf.\ \Cref{sec:Deltacalc}, we get that
\begin{align}\label{eq:Deltas}
  \begin{aligned}
    \Delta_1 \xi &= -\ell(\ell+1)D_0\xi \\
    \Delta_2 \xi &= (\ell^2-1)\xi - (\ell+1)^2D_0^2\xi + (4\ell^2-1)D_+D_-\xi
  \end{aligned}                 
\end{align}
for $\xi\in R_\ell$, $\ell\neq 0$. Here as in the relations for the $D$'s earlier, we replace $D_+D_-$ with $E_+D_+D_-E_-$ for $\xi \in R_{\ell,\ell}$ and $E_-D_+D_-E_+$ for $\xi\in R_{\ell,-\ell}$. Alternatively by \cref{eq:Drels}, cf.\ \Cref{sec:Deltacalc}, we also get that
\begin{align}\label{eq:altDelta2}
  \Delta_2 \xi = \bigl((\ell+1)^2-1\bigr)\xi  + \ell^2D_0^2\xi + \bigl(4(\ell+1)^2-1\bigr)D_-D_+\xi
\end{align}
for $\xi\in R_\ell$, which will sometimes be more useful.

Now by noting that $D_0$, $D_+D_-$, and $D_0^2$ all preserve $R_{\ell,m}$ \cref{eq:Deltas} gives us the following Lemma:
\begin{lemma}\label{lem:R_lmDeltainvariant}
  Each subspace $R_{\ell,m}$ is invariant under the Laplace operators $\Delta_1$ and $\Delta_2$. 
\end{lemma}

Additionally we are ready to prove the Lemma:
\begin{lemma}\label{lem:commuteDeltas}
  The linear operators $D_+$, $D_-$, $D_0$, $E_+$, and $E_-$ commute with the Laplace operators $\Delta_1$ and $\Delta_2$. 
\end{lemma}
\begin{proof}
  Denote by $(\Delta_i)_{\ell,m}$ the restriction of $\Delta_i$ to $R_{\ell,m}$ for $i=1,2$. \Cref{lem:R_lmDeltainvariant} implies that $\Delta_i=\bigoplus_{\ell,m} (\Delta_i)_{\ell,m}$ for $i=1,2$, so it is enough to check that $(\Delta)_{\ell,m}$ commutes with the operators for all $\ell$ and $m$. Therefore \cref{eq:Deltas,eq:altDelta2} implies that $\Delta_i$ commute with $E_+$ and $E_-$ since $D_+$, $D_-$, and $D_0$ commute with $E_+$ and $E_-$ where it makes sense and using \cref{eq:altDelta2} for $\Delta_2$ it makes sense for all $R_{\ell,m}$.

  Now multiplying the first equation of \cref{eq:Drels} with $\ell+1$, we see that
  \begin{align*}
    \ell(\ell+1)D_+D_0\xi = (\ell+1)(\ell+2)D_0D_+ \xi
  \end{align*}
  for $\xi\in R_{\ell,m}$, so by \cref{eq:Deltas}, we see that
  \begin{align*}
    D_+\Delta_1 \xi &= -\ell(\ell+1)D_+D_0 \xi = -(\ell+1)(\ell+2)D_0D_+ \xi = \Delta_1 D_+ \xi
  \end{align*}
  for $\xi\in R_{\ell,m}$. Thus $\Delta_1$ indeed commutes with $D_+$. Similarly the second equation of \cref{eq:Drels} imply that $\Delta_1$ commutes with $D_-$, and also it is obvious from \cref{eq:Deltas} that $\Delta_1$ commutes with $D_0$.

  Likewise the first equation of \cref{eq:Drels} together with \cref{eq:Deltas,eq:altDelta2} implies that
  \begin{align*}
    \Delta_2 D_+ \xi &= \bigl((\ell+1)^2-1\bigr)D_+\xi - (\ell+2)^2D_0^2D_+\xi + \bigl(4(\ell+1)^2-1\bigr)D_+D_-D_+\xi \\
                     &= \bigl((\ell+1)^2-1\bigr)D_+\xi - \ell^2D_+D_0^2\xi + \bigl(4(\ell+1)^2-1\bigr)D_+D_-D_+\xi \\
                     &= D_+\Delta_2 \xi
  \end{align*}
  for $\xi\in R_{\ell,m}$. Thus $\Delta_2$ commutes with $D_+$, and similarly using the second equation of \cref{eq:Drels} we get that $\Delta_2$ commutes with $D_-$. Finally it is clear that $D_0$ commutes with the first two terms of $\Delta_2$, so we just need to show that $D_0(D_+D_-)\xi=(D_+D_-)D_0\xi$ for $\xi\in R_{\ell,m}$ where it makes sense. But now the first and second equation of \cref{eq:Drels} imply that
  \begin{align*}
    (\ell+1)D_0D_+D_-\xi &= (\ell-1)D_+D_0D_-\xi = (\ell+1) D_+D_-D_0\xi
  \end{align*}
  for $\xi\in R_{\ell,m}$, so since $\ell\geq 0$ and thus $\ell\neq -1$, we get that $D_0(D_+D_-)\xi=(D_+D_-)D_0\xi$. Hence $\Delta_2$ indeed commutes with $D_0$ also.
\end{proof}

\subsection{Properties of the Laplace operators in indecomposable modules}

Now we are finally ready to begin considering the properties of $\Delta_1$ and $\Delta_2$ in indecomposable Harish-Chandra modules, which will end up being an important part of our characterization of indecomposable Harish-Chandra modules for the pair $(L,L_k)$.

\begin{proposition}
  A Harish-Chandra module $M$ for the pair $(L,L_k)$ is decomposable into the direct sum of a countable number of indecomposable modules such that on each indecomposable module the Laplace operators $\Delta_1$ and $\Delta_2$ have each one eigenvalue, $\lambda_1$ and $\lambda_2$ respectively.
\end{proposition}
\begin{proof}
  Since each of the subspaces $R_{\ell,m}$ is invariant under $\Delta_1$ and $\Delta_2$ by \Cref{lem:R_lmDeltainvariant} and since these operators commute with each other, we get that $R_{\ell,m}$ can be written as a direct sum of subspaces $R_{\ell,m}(\lambda_1^{i},\lambda_2^{i})$ on each of which each of the operators $\Delta_1$ and $\Delta_2$ has one eigenvalue $\lambda_1^{i}$ and $\lambda_2^{i}$ respectively. Note that here the index set of $i$ is finite since $R_{\ell,m}$ is finite dimensional.

  Consider now fixed $\lambda_1$ and $\lambda_2$ and the set $S$ of those $(\ell,m)$ for which there exists subspaces $R_{\ell,m}(\lambda_1^{i},\lambda_2^{i})$ with $\lambda_1=\lambda_1^{i}$ and $\lambda_2=\lambda_2^{i}$. Denote by $M(\lambda_1,\lambda_2)$ the subspace of $M$ with $M(\lambda_1,\lambda_2)=\bigoplus_{(\ell,m)\in S} R_{\ell,m}(\lambda_1,\lambda_2)$ such that in $M(\lambda_1,\lambda_2)$ each of the operators $\Delta_1$ and $\Delta_2$ has one eigenvalue, $\lambda_1$ and $\lambda_2$ respectively. We want to show that $M(\lambda_1,\lambda_2)$ is a submodule of $M$, i.e.\ that it is invariant under $H_+$, $H_-$, $H_3$, $F_+$, $F_-$, and $F_3$, but this is clearly the case since $M(\lambda_1,\lambda_2)$ is invariant under $E_+$, $E_-$, $D_+$, $D_-$, and $D_0$ because $\Delta_1$ and $\Delta_2$ commute with these operators by \Cref{lem:commuteDeltas}. Finally note that the number of $M(\lambda_1,\lambda_2)$ in the decomposition of $M$ is countable since the number of $R_{\ell,m}$ is countable and the number of $R_{\ell,m}(\lambda_1^{i},\lambda_2^{i})$ in a given $R_{\ell,m}$ is finite, so decomposing each $M(\lambda_1,\lambda_2)$ we get the result.
\end{proof}

\begin{proposition}\label{prop:indecomposabledecomposition}
  Let $M$ be a Harish-Chandra module in which each of the Laplace operators $\Delta_1$ and $\Delta_2$ has one eigenvalue. Then there exists an integral or half-integral number $\ell_0\geq 0$ and a complex number $\ell_1$ such that the eigenvalues $\lambda_1$ and $\lambda_2$ have the form
  \begin{align}\label{eq:eigenvaluesformulae}
    \lambda_1 &= -i\ell_0\ell_1, & \lambda_2 &= \ell_0^2 + \ell_1^2 - 1.
  \end{align}
\end{proposition}
\begin{proof}
  Denote by $\ell_0$ the minimal index in the decomposition $M=\bigoplus_\ell R_\ell$ of $M$ into $L_k$-submodules of $R_\ell$. By the definition of $D_-$ it maps $R_{\ell_0}$ to zero, so by \cref{eq:Deltas} we get that
  \begin{align}\label{eq:Deltastemp}
    \begin{aligned}
      \Delta_1 \xi &= -\ell_0(\ell_0+1)D_0\xi \\
      \Delta_2 \xi &= (\ell_0^2-1)\xi - (\ell_0+1)D_0^2\xi
    \end{aligned}
  \end{align}
  for $\xi\in R_{\ell_0}$. Now if $\ell_0\neq 0$ the subspace $R_{\ell_0}$ is invariant under $D_0$, so we can find an eigenvector $\xi_0$ for $D_0$ such that $D_0\xi_0 = \mu\xi_0$ for some $\mu\in\C$. Thus we see that
  \begin{align*}
    \Delta_1 \xi_0 &= -\ell_0(\ell_0+1)\mu\xi_0 \\
    \Delta_2 \xi_0 &= (\ell_0^2-1)\xi_0 - (\ell_0+1)\mu^2\xi_0,
  \end{align*}
  so we get eigenvalues $\lambda_1$ and $\lambda_2$ of $\Delta_1$ and $\Delta_2$ with
  \begin{align*}
    \lambda_1 &= -\ell_0(\ell_0+1)\mu, & \lambda_2 &= (\ell_0^2-1) - (\ell_0+1)\mu^2.
  \end{align*}
  Hence putting $(\ell_0+1)\mu = i\ell_1$, we get that
  \begin{align*}
    \lambda_1 &= -i\ell_0\ell_1, & \lambda_2 &= \ell_0^2 + \ell_1^2 - 1.
  \end{align*}
  Now by assumption each of $\Delta_1$ and $\Delta_2$ has only one eigenvalue on $M$, and thus these eigenvalues are expressed in terms of the $\ell_0$ and $\ell_1$ as in \cref{eq:eigenvaluesformulae}. Also if $\ell_0=0$ it is clear that we can still find a complex number $\ell_1$ such that the above equations are satisfied.
\end{proof}

Note that all such eigenvalues $\lambda_1$ and $\lambda_2$ for $\Delta_1$ and $\Delta_2$ are possible, since in the case of finite dimensional simple modules $N$ as in \Cref{sec:simplemodules} we have by \cref{eq:infdimds,eq:Deltas} that
\begin{align*}
  \Delta_1 \xi_{\ell,m} &= -\ell(\ell+1)\dd{\ell}{0}\xi_{\ell,m} = -i\ell_0\ell_1, \\
  \Delta_2 \xi_{\ell,m} &= (\ell^2-1)\xi_{\ell,m} - (\ell+1)^2(\dd{\ell}{0})^2\xi_{\ell,m} + (4\ell^2-1)\dd{\ell+1}{+}\dd{\ell}{-}\xi_{\ell,m} \\*
                        &= (\ell^2-1)\xi_{\ell,m} + \dfrac{\ell_0^2\ell_1^2}{\ell^2}\xi_{\ell,m} - \dfrac{(\ell^2-\ell_1^2)(\ell^2-\ell_0^2)}{\ell^2} \xi_{\ell,m} \\*
                        &= (\ell^2-1)\xi_{\ell,m} - (\ell^2-\ell_0^2-\ell_1^2)\xi_{\ell,m} \\*
                        &= (\ell_0^2+\ell_1^2-1)\xi_{\ell,m},
\end{align*}
for $\xi_{\ell,m}\in R_{\ell,m}$, where $\ell_0\geq 0$ is an integral or half-integral number and $\ell_1$ is a complex number. Here we can construct $N$ such that $\ell_0$ and $\ell_1$ are as we want.

\begin{proposition}\label{prop:eigenvaluesforDs}
  Let $M$ be a Harish-Chandra module in which the Laplace operators $\Delta_1$ and $\Delta_2$ have only one eigenvalue $\lambda_1$ and $\lambda_2$ respectively. Then on each subspace $R_\ell$ the operators $D_+D_-$, $D_-D_+$, and $D_0$ have only one eigenvalue $\dd{\ell}{-}$, $\dd{\ell}{+}$, and $\dd{\ell}{0}$ respectively. Here the numbers $\dd{\ell}{-}$, $\dd{\ell}{+}$, and $\dd{\ell}{0}$ are expressed in terms of $\ell_0$ and $\ell_1$ in the following way:
  \begin{align}\label{eq:dformulaeonR_l}
    \begin{aligned}
      \dd{0}{-} &= \dd{1/2}{-}=0, \\
      \dd{\ell}{-} &= \dfrac{(\ell^2-\ell_0^2)(\ell_1^2-\ell^2)}{(4\ell^2-1)\ell^2} && \mbox{if } \ell\neq 0,\tfrac{1}{2}, \\
      \dd{\ell}{+} &= \dfrac{((\ell+1)^2-\ell_0^2)(\ell_1^2-(\ell+1)^2)}{(4(\ell+1)^2-1)(\ell+1)^2} \\
      \dd{0}{0} &= i\ell_1, \\
      \dd{\ell}{0} &= \dfrac{i\ell_0\ell_1}{\ell(\ell+1)} && \mbox{if }\ell\neq 0.
    \end{aligned}
  \end{align}
\end{proposition}
\begin{proof}
  By \cref{eq:Deltas} we see that
  \begin{align*}
    (4\ell^2-1)D_+D_-\xi &= \Delta_2\xi -(\ell^2-1)\xi + (\ell+1)^2D_0^2\xi, \\
    (\ell+1)D_0 \xi &= -\dfrac{\Delta_1}{\ell}\xi
  \end{align*}
  for $\xi\in R_\ell$ with $\ell>\ell_0$ such that $\ell\neq 0$. Thus
  \begin{align*}
    (4\ell^2-1)D_+D_-\xi &= \Delta_2\xi -(\ell^2-1)\xi + \dfrac{\Delta_1^2}{\ell^2}\xi
  \end{align*}
  for $\xi\in R_\ell$ with $\ell>\ell_0$. Hence since $\Delta_1$ and $\Delta_2$ each only have one eigenvalue on $R_\ell$ so thus $D_+D_-$, and we see by \cref{eq:eigenvaluesformulae} that
  \begin{align*}
    \dd{\ell}{-} &= \dfrac{1}{(4\ell^2-1)}\Bigl( \lambda_2 - (\ell-1)^2 + \dfrac{\lambda_1^2}{\ell^2}  \Bigr) \\
                 &= \dfrac{(\ell_0^2+\ell_1^2-1)\ell^2 - (\ell-1)^2\ell^2 - \ell_0^2\ell_1^2}{(4\ell^2-1)\ell^2} \\
                 &= \dfrac{(\ell_0^2+\ell_1^2-\ell^2)\ell^2 - \ell_0^2\ell_1^2}{(4\ell^2-1)\ell^2} \\
                 &= \dfrac{(\ell^2-\ell_0^2)(\ell_1^2-\ell^2)}{(4\ell^2-1)\ell^2} \\
  \end{align*}
  for $\ell\neq 0,\tfrac{1}{2}$. Now since $D_-$ by definition maps $R_0$ and $R_{1/2}$ to zero if they occur in the decomposition of $M$, we see that $\dd{0}{-}=0$ and $\dd{1/2}{-}=0$, and likewise we know that $D_-$ maps $R_{\ell_0}$ to zero so $\dd{\ell_0}{-}=0$, which also holds true with the formula above. Thus we have proven the formulae for $\dd{\ell}{-}$, and the other formulae are proven similarly. 
\end{proof}

Now denote by $C_s(\lambda_1,\lambda_2)$ for $s=1$ or $s=\tfrac{1}{2}$ the set of all Harish-Chandra modules for the pair $(L,L_k)$ in which the Laplace operators have the eigenvalues $\lambda_1$ and $\lambda_2$, and in which if $s=1$ every $M\in C_1(\lambda_1,\lambda_2)$ has only integral numbers as indices in the decomposition $M=\bigoplus_\ell R_\ell$, and if $s=\tfrac{1}{2}$ every $M\in C_{1/2}(\lambda_1,\lambda_2)$ has only half-integral numbers as indices in the decomposition $M=\bigoplus_\ell R_\ell$. 

\begin{proposition}\label{prop:modulehom}
  Let $M\in C_s(\lambda_1,\lambda_2)$, $M'\in C_{s'}(\lambda_1',\lambda_2')$, where $(s,\lambda_1,\lambda_2)\neq (s',\lambda_1',\lambda_2')$. Then $\Hom_L(M,M')=0$.
\end{proposition}
\begin{proof}
  Let $\gamma\in\Hom_L(M,M')$ and assume that $\gamma\neq 0$. First we will show that $\gamma R_\ell \subset R_\ell'$, where $R_\ell$ comes from the decomposition of $M$ and $R_\ell'$ from the decomposition of $M'$. To see this note that $R_\ell$ and $R_\ell'$ are direct sums of finitely many $L_k$-modules $M(2\ell)$, where each $M(2\ell)$ is generated by a maximal vector of weight $\ell$ (weight w.r.t. $h_3$ in $L_k$). So since $\gamma$ takes a maximal vector of weight $\ell$ to either zero or another maximal vector of weight $\ell$, we see that indeed $\gamma R_\ell \subset R_\ell'$.
  
  Denoting by $\Delta_i$ the Laplace operators in $M$ and by $\Delta_i'$ the Laplace operators in $M'$, we also have that $\Delta_i' \gamma = \gamma \Delta_i$ for $i=1,2$, since in both cases $\Delta_i$ and $\Delta_i'$ correspond to the actions in $U(L)$ of \cref{eq:DeltasaU(L)def} and $\gamma\in\Hom_L(M,M')=\Hom_{U(L)}(M,M')$. Now since $\gamma\neq 0$ we get that $\gamma R_\ell \subset R_\ell'$ implies that $s=s'$ and $\Delta_i'\gamma=\gamma\Delta_i$ implies that $\lambda_i=\lambda_i'$ for $i=1,2$, but this is a contradiction since $(s,\lambda_1,\lambda_2)\neq (s',\lambda_1',\lambda_2')$. Hence we must have that $\gamma=0$, and thus indeed $\Hom_L(M,M')=0$. 
\end{proof}

This implies that the study of the category of Harish-Chandra modules for the pair $(L,L_k)$ can be reduced to the study of the category $C_s(\lambda_1,\lambda_2)$ of modules in which the Laplace operators have exactly one eigenvalue.

\begin{remark}
  From now on in places where the index $s$ is not important, we will simply denote $C_s(\lambda_1,\lambda_2)$ by $C(\lambda_1,\lambda_2)$. 
\end{remark}

\begin{definition}
  The category of modules $C(\lambda_1,\lambda_2)$ is called singular if the numbers $\ell_0$ and $\ell_1$ constructed from $\lambda_1$ and $\lambda_2$ as in \cref{eq:eigenvaluesformulae} are such that $\ell_1-\ell_0$ is an integer. Otherwise it is called non-singular.
\end{definition}

We will see that the study of the non-singular categories $C(\lambda_1,\lambda_2)$ is simpler than that of the singular ones, and we are now ready to begin our description of the category of the singular categories $C(\lambda_1,\lambda_2)$. 

%%% Local Variables:
%%% mode: latex
%%% TeX-master: "../main"
%%% End:
