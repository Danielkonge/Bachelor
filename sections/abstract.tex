\abstractintoc

\pagestyle{plain}
\null\vspace{\fill}
\begin{abstract}
  This paper investigates a special class of modules over the Lie algebra $L=\slpr$ called Harish-Chandra modules, which are defined as follows: Given a Lie algebra $L$ and a Lie subalgebra $L_k$ an $L$-module $M$ is a Harish-Chandra module for the pair $(L,L_k)$ if it can be written as direct sum $M=\bigoplus_i M_i$ of finite dimensional simple $L_k$-modules, where for each $M_{i_0}$ there are only finitely many $M_i$ equivalent to $M_{i_0}$ in the decomposition. In this paper we consider the pair $(L,L_k)$ with $L=\slpr$ and $L_k=\Set{(u,u)\given u\in \slc} \subset L$. We start out by giving a general description of $L$-modules, and then move on to give a characterization of all simple Harish-Chandra modules for the pair $(L,L_k)$. After this we begin working towards the main goal of this paper: A characterization of the indecomposable Harish-Chandra modules for the pair $(L,L_k)$. We show that overall these can be split into two kinds of modules, singular and non-singular, where the singular further can be split into two different types, open and closed. Here the non-singular modules are completely determined by numbers $(\ell_0,\ell_1,n)$, where $\ell_0\in\tfrac{1}{2}\Z_{\geq0}$, $\ell_1\in\C$, and $n\in\N$, while the singular modules are completely determined by some integers $(s,n_1,m_1,n_2,m_2,\dotsc,n_k,m_k)$ (the open type) or by some integers and a complex number $\mu$ $(n_1,m_1,\dotsc,n_k,m_k,\mu,N)$ (the closed type).
\end{abstract}
\vspace{\fill}
\newpage
%%% Local Variables:
%%% mode: latex
%%% TeX-master: "../main"
%%% End:
